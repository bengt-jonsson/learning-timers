%\newcommand{\natplus}{\nat^{>0}}
%\newcommand{\realsplus}{\bbbr^{\geq 0}}
%\newcommand{\stoptimer}{\mathit{kill}}
%\newcommand{\tosymbol}{\mathit{to}}
%\newcommand{\toevent}[1]{\mathit{to}[#1]}
%\newcommand{\toevents}{\mbox{\sl TO}}
%\newcommand{\extinputs}{\hat{I}}
\newcommand{\acttimers}{\mathit{active}}
%\newcommand{\expirable}{\mathit{expirable}}
%\newcommand{\tvals}{\kappa}
%\newcommand{\delay}[2]{t_{[#1:#2]}}
%\newcommand{\timerof}[2]{x_{#1}^{#2}}
\newcommand{\constrof}[1]{\phi_{#1}}
\newcommand{\post}{\mathit{post}}

%\newcommand{\conc}{\cdot}
%\newcommand{\tuple}[1]{\langle #1\rangle}
%\newcommand{\set}[1]{\lbrace #1\rbrace}
%\newcommand{\vect}[2]{{#1}_1 , \ldots , {#1}_{#2}}
%\newcommand{\setcomp}[2]{\set{#1 ~:~ #2}}
%\newcommand{\domof}[1]{\dom(#1)}
%\newcommand{\ranof}[1]{\ran(#1)}
%\newcommand{\vars}{\mathcal{X}}
%\newcommand{\varsof}[1]{\vars(#1)}
\newcommand{\ctimers}{X}
%\newcommand{\remap}{\pi}
%\newcommand{\remapinst}{\rho}
\newcommand{\normalize}{\gamma}
\newcommand{\normalizeof}[2]{\normalize_{#2}^{#1}}
\newcommand{\timerbij}{\gamma}
\newcommand{\timerequiv}{\pi}
\newcommand{\extendedby}{\lhd}
\newcommand{\uttrace}{\textsf{tr}}
\newcommand{\uttraceof}[1]{\uttrace(#1)}
\newcommand{\uttracesof}[1]{\textsf{Tr}(#1)}
\newcommand{\strace}{\textsf{tr}_s}
\newcommand{\ssuffix}{v_s}
\newcommand{\suftraces}{\textsf{Tr}_s}
\newcommand{\pinpof}[1]{\textit{inp}_p(#1)}
\newcommand{\sinpof}[1]{\textit{inp}_s(#1)}
\newcommand{\symbinpof}[1]{\textit{symbinp}(#1)}
\newcommand{\word}{w}
\newcommand{\smap}{{\cal O}}
\newcommand{\smappre}{{\cal O_p}}
\newcommand{\smapsuf}{{\cal O_s}}
\newcommand{\obspre}{{\cal O_U}}

\section{General Treatment for Arbitrary Number of Timers}

It is now time to consider the general case of
Mealy machines with an arbitrary number of timers (MMnT). 
These just generalize MM1Ts in that they have a set of timers that
can be manipulated in the same way as the timer of a MM1T.
We assume an unbounded set $X$ of {\em timers};
we use $x$, $x_1$, $x_2$, etc.\ to range over timers.
Let $\toevents$ be the set of {\em timeout events} of form
$\toevent{x}$ for $x \in X$.
For a set $I$, let $\extinputs$ be $I \cup \toevents$.

\begin{definition}
\label{def:MMnT}
A \emph{Mealy machine with timers (MMnT)} is a tuple
\\
$\M = (I, O, Q, q_0, \vars, \delta, \lambda, \remap)$, where
\begin{itemize}
\item
  $I$ is a finite set of inputs,
\item
$O$ is a finite set of outputs, containing a special default output $\Lambda$,
\item
$Q$ is a finite set of states,
\item
$q_0 \in Q$ is the initial state,
\item
  $\vars$ maps each state $q \in Q$ to a finite set
$\varsof{q}$ of timers,
\item
$\delta: Q \times \extinputs \rightarrow Q$ is a transition function, 
\item
$\lambda: Q \times \extinputs \rightarrow O$ is an output function, 
\item
  $\remap$ maps each $q \in Q$ and $i \in \extinputs$ to an injective mapping from
  $\varsof{\delta(q,i)}$ to $(\varsof{q} \cup \nat^{>0})$, where also
  $y \not\in \ran(\remap(q,i))$ if $i$ is $\toevent{y}$ for timer $y$
\end{itemize}
\end{definition}
The mapping $\remap$ specifies how timers are affected when an input or timeout event $i$ occurs in state $q$.
Consider a timer $x \in \varsof{\delta(q,i)}$ in the state reached after
supplying $i$.
If $\remap(q,i)(x) \in\nat^{>0}$ then we say that $i$ \emph{starts} the timer $x$ in state $q$ and sets the timer to $\remap(q,i)(x)$.
If $\remap(q,i)(x)$ is a timer $y$ in $\varsof{q}$, then $i$ does not
affect the timer $y$, it is merely renamed to timer $x$.
If timer $y \in \varsof{q}$ is not in the range of $\remap(q,i)$, then
input $i$ \emph{kills} the timer $y$ (before it has expired).
The renaming of timers is included in order to make MMnTs independent of
actual names of timers. Two MMnTs which differ only in the names
of their timers can be made equivalent by suitable renaming of their
timers. Renaming also allows two states that differ only by the names
of their timers to be identified: this is crucial for the construction of
canonical MMnTs. The renaming of timers is analogous to the renaming of
registers in~\cite{CasselHJS16}.

%% We require that, for each state $q$, $\tau(q,\mathit{to}_x)(x) \neq\perp$, that is, when $x$ times out then $x$ is either started or stopped.


\paragraph{Note:}
Definition~\ref{def:MMnT}
does not reflect that a timeout event $\toevent{x}$ can
occur in state $q$ only if $x \in \varsof{q}$. In future updates, we  should
refine the definition to make this clear. Later in the paper (when we
derive canonical MMnTs), we will also
make sure that $\toevent{x}$ is considered if and only if $\toevent{x}$ can
actually occur in $q$ in some actual behavior.
%% if $\remap$ maps
%% $(q,i)$ to the partial function $\nu$ from timers to 
%% $\nat^{>0} \cup \{ \stoptimer \}$. If the domain of $\nu$ is empty,
%% we write $q \xrightarrow{i/o} q'$

%% The above definition of MMnT does not define any canonical form, for
%% two reasons.
%% \begin{itemize}
%% \item
%%   Wheter or not a timeout event $\toevent{x}$ can arrive in a certain state
%%   $q$ depends, in general, not only on $q$, but also on the previous
%%   sequence of events and their timing.
%% \item
%%   Equivalent MMnTs may differ only by the names of their timers.
%% \end{itemize}
%% We will later present a modified definition of a canonical form of MMnT,
%% which addresses these two problems. For now, let us ignore them and
%% get on with the definition of the semantics of MMnTs.

\paragraph{Semantics.}
We give two semantics for MMnTs, an untimed and a timed one. Let us
first consider the untimed semantics.

Suppose $\delta(q,i) = q'$ and $\lambda(q,i)= o$.
Then we write $q \xrightarrow{i/o,\remap(q,i)} q'$.
A \emph{trace} of $\M$ is a sequence of
triples
$\tuple{i_1,o_1,\remapinst_1}\tuple{i_2,o_2,\remapinst_2}
\cdots\tuple{i_n,o_n,\remapinst_n}$ such that there is a sequence 
\[
q_0 \xrightarrow{i_1/o_1,\remapinst_1} q_1
\xrightarrow{i_2/o_2,\remapinst_2}
\cdots
q_{n-1} \xrightarrow{i_n/o_n,\remapinst_n} q_n
\]
of $\M$.
Note that the set of timers $\varsof{q_j}$ can be obtained from the trace,
since it is the same as $\domof{\remapinst_j}$.
We would, intuitively, like to let the untimed semantics of $\M$ be the set
of its traces. However, it would then depend heavily on the names of
timers. Therefore, we define a normalization operation on traces, which
renames timers to canonical names. This definition is found in
Section XXX.

The timed semantics, which is slightly more involved, describes the real-time behavior of a MMnT.
This semantics is defined by associating an infinite state transition system to a MMnT that describes all possible
configurations and transitions between them.
A \emph{configuration} of a MMnT is a pair $(q,\tvals)$, where $q \in Q$ is a state and $\tvals : \varsof{q} \rightarrow \realsplus$ 
specifies the values of the timers of $q$. The initial configuration is
the pair $(q_0, \tvals_0)$, where $\tvals_0$ is the empty function.
%% If $t = \stoptimer$ then we say that the timer has been \emph{stopped}, otherwise it is \emph{running}.
Using three rules we define a transition relation that describes how one configuration may evolve into another.

For a partial function $\tvals$ from $X$ to $\realsplus$, define
%% \begin{itemize}
%% \item
  $\tvals - d$ as the function mapping each $x \in \domof{\tvals}$ to $\tvals(x) - d$.
%% \item
%%   $\tvals \setminus x$ as the function obtained from $\tvals$ by removing $x$ from the
%%   domain,
%% \item
%%   $\nu(\tvals)$ as the function obtained by applying $\nu$ to the mapping $\tvals$,
%%   i.e.,
%%   \begin{itemize}
%%     \item
%%   $\domof{\nu(\tvals)} = (\domof{\nu} \cup \domof{\tvals}) \setminus
%%   \setcomp{x}{\nu(x) = \stoptimer}$, and
%% \item
%%   $\nu(\tvals)(x) = \nu(x)$ if $\nu(x) \in \natplus$ else $\tvals(x)$.
%%   \end{itemize}
%% \end{itemize}
For all $q, q' \in Q$, $i \in I$, $o \in O$, $\tvals$ a
mapping from $\varsof{q}$ to $\realsplus$, and $d \in \realsplus$:
\[
\frac{\forall x \in \varsof{q}: \ \ d \leq \tvals(x)}{(q,\tvals) \xrightarrow{d} (q,\tvals-d)}
\quad
  \frac{\tvals(x) = 0 \quad q \xrightarrow{\toevent{x}/o,\remapinst} q'}{(q,\tvals) \xrightarrow{\toevent{x}/o} (q',\tvals\circ\remapinst)}
  \quad
  \frac{i \in I ,~  q \xrightarrow{i/o,\remapinst} q'}{(q,\tvals) \xrightarrow{i/o} (q',\tvals\circ\remapinst)}
\]
The first rule states that the value of each timer decreases when time advances, and that
time can advance as long as the values of all timers are nonzero.
The second rule says that a timeout event $\toevent{x}$ may occur
when the timer $x$ has reached value $0$.
The third rule describes the transitions in response to a regular input.
In the two last rules, the values of timers  in the successor configuration
is given by 
$\tvals\circ\remapinst$, i.e., each timer $x \in \varsof{q'}$ is either
set to a value $\remapinst(x) \in \natplus$ or takes the value of the
timer $\remapinst(x) \in \varsof{q}$ in the configuration
$(q,\tvals)$.


A \emph{timed word} over inputs $\extinputs$ and outputs $O$ is a sequence
\\
$w = (i_1, o_1, t_1), (i_2, o_2, t_2) \cdots (i_k, o_k, t_k)$, where each $i_j \in \extinputs$, each $o_j \in O$, and each $t_j \in \realsplus$.
For a timed word $w = (i_1, o_1, t_1), (i_2, o_2, t_2) \cdots (i_k, o_k, t_k)$, a \emph{run} of MM1T $\M$ over $w$ is
a sequence 
\begin{eqnarray*}
\alpha & = & C_0 \xrightarrow{t_1} C'_0 \xrightarrow{i_1/o_1} C_1 \xrightarrow{t_2} C'_1 \xrightarrow{i_2/o_2} C_2 \cdots
\xrightarrow{t_k} C'_{k-1} \xrightarrow{i_k/o_k} C_{k}
\end{eqnarray*}
of transitions of $\M$ such that each $C_j, C'_j$ is a configuration of $\M$ and $C_0$ is the initial configuration.
Note that, since MMnTs are deterministic (if we allow to observe the
identities of timers in timeout events),
for each timed word $w$ there exists at most one run over $w$.
We say $w$ is a timed word of $\M$ if there exists a run of $\M$ over $w$.
Two MMnTs $\M$ and $\N$ with the same sets of inputs are \emph{timed equivalent}, denoted $\M \approx_{\mathit{timed}} \N$, iff 
they have the same sets of timed words.

\paragraph{Note:}
We will assume the following semantic requirements on MMnTs. At this point,
we just formulate them informally, and leave the precise formalization to
a later section.
\begin{itemize}
\item
    {\em Timer-liveness}, which says that whenever $x \in \varsof{q}$, then
  there is a sequence of transitions from some reachable $(q,\tvals)$ to some
  transition where $x$ expires.
\item
  {\em Robustness}, which says that any sequence of transitions does not
  depend on precise timing. Roughtly, it means that we can slightly change
  some delay and still obtain the same untimed behavior.
\end{itemize}
Also, A design decision has to be made regarding priorities between timers. I.e.,
to determine what should happen when several timers reach $0$ simultaneously.
We take the point that the choice of timer is nondeterministic, and that
expiration of one timer may still leave the other timers unaffected so that
they can also expire. A consequence is that the behavior is then deterministic
only if we can observer the identity of timers. This issue will be resolved
by assuming that MMnTs are robust, which implies that wee need not consider
timed behaviors where two timers may expire at the same time.

\subsection{A Closer Look at Traces, and Nerode Equivalence}  
In this section, we will more precisely define a semantics of MMnT,
defined in terms of traces, which is canonical. We first need to look more
closely at traces.

A {\em suffix trace}
is a sequence
$\strace = \tuple{i_1,o_1,\remapinst_1}\cdots\tuple{i_n,o_n,\remapinst_n}$,
such that for $j = 1, \ldots , n$, we have $i_j \in \extinputs$, $o_j \in O$ and
$\remapinst_j: X \rightarrow (X \cup \natplus)$ is a mapping from timers to
timers and natural numbers, such that 
$\ranof{\remapinst_j} \subseteq (\domof{\remapinst_{j-1}} \cup \natplus)$ and
$y \not\in \ran(\remapinst_j)$ if $i_j$ is $\toevent{y}$ for
$j = 2, \ldots, n$. Define
$\domof{\strace} = \domof{\remapinst_n}$ and
  $\ranof{\strace} = \ranof{\remapinst_1}$.
For two suffix traces 
$\strace = \tuple{i_1,o_1,\remapinst_1}\cdots\tuple{i_n,o_n,\remapinst_n}$ and
$\strace' = \tuple{i_1,o_1,\remapinst_1'}\cdots\tuple{i_n,o_n,\remapinst_n'}$
with the same sequences of (extended) inputs and outputs, which
are consistent (i.e., $\remapinst_j$ and $\remapinst_j'$ agree on the intersection of their domains), we define 
$\strace \sqcup \strace' = \tuple{i_1,o_1,\remapinst_1 \cup \remapinst_1'}\cdots\tuple{i_n,o_n,\remapinst_n\cup\remapinst_n'}$.

A {\em trace} is a suffix trace $\uttrace$ with $\ranof{\uttrace} = \natplus$.

We would, intuitively, like to let the untimed semantics of $\M$ be the set
of its traces. However, it would then depend heavily on the names of
timers. Therefore, we define a normalization operation on traces, which
renames timers to canonical names.
Intuitively, a timer which is set to $p$ at the $j$th element of the trace
(i.e., in response to input $i_j$) is given the name $\timerof jp$.
A trace is normalized by consistently renaming timers in this way.
More precisly, 
consider a trace 
$\uttrace = \tuple{i_1,o_1,\remapinst_1}\cdots\tuple{i_n,o_n,\remapinst_n}$.
%% For a mapping $\remapinst_k$ and timer $x \in \domof{\remapinst_k}$, let
%% $j$ be the element in $\uttrace$ where $x$ is set, and let $p$ be the
%% value that it is set to.
%% More precisely, let $\remapinst_{[j:k]}$ denote
%% $\remapinst_j \circ \remapinst_{j+1} \circ \cdots \circ \remapinst_k$.
For $k=1, \ldots , n$,
define the injective mapping $\normalizeof{\uttrace}{k}$ on $\domof{\remapinst_k}$ by
$\normalizeof{\uttrace}{k}(x) = \timerof jp$, where 
$j$, $p \in \natplus$ are such that
$(\remapinst_j \circ \remapinst_{j+1} \circ \cdots \circ \remapinst_k)(x) = p$.
%% $\remapinst_{[j:k]}(x) = m$.
Let $\normalizeof{\uttrace}{0}$ be the empty mapping.
Then the normalization of trace
$\tuple{i_1,o_1,\remapinst_1}\cdots\tuple{i_n,o_n,\remapinst_n}$ is
the trace
\[
\tuple{i_1,o_1,(\normalizeof{\uttrace}{0} \circ \remapinst_1 \circ (\normalizeof{\uttrace}{1})^{-1})}\cdots
\tuple{\normalizeof{\uttrace}{n-1}(i_n),o_n,(\normalizeof{\uttrace}{n-1} \circ \remapinst_n \circ (\normalizeof{\uttrace}{n})^{-1})}
,
\]
where $\normalizeof{\uttrace}{k-1}(\toevent{x}) = \toevent{\normalizeof{\uttrace}{k-1}(x)}$ and
$\normalizeof{\uttrace}{k-1}(i) = i$ for $i \in I$.
%% For a trace $\uttrace = \tuple{i_1,o_1,\remapinst_1}\cdots\tuple{i_n,o_n,\remapinst_n}$ we let $\remapinst_{\uttrace}$ denote $\remapinst_n$.

Let $\uttracesof{\M}$ denote the set of normalized traces of $\M$.
The untimed semantics of a MMnT $\M$ is the set of its normalized traces.

This untimed semantics can also be used to define a Nerode equivalence on
MMnTs.

For a suffix trace $\strace$ and a bijection $\timerbij$ on $X$ let
$\timerbij(\strace)$ denote the result of renaming timers in $\uttrace$
according to $\timerbij$.
Extend $\timerbij$ to sets of traces in the natural way.
Let $X_A, X_B$ be finite sets of timers
For a bijection $\timerequiv: X_A \rightarrow X_B$, let
$\timerequiv \extendedby\timerbij$ denote that $\timerequiv$ is the restriction
of $\timerbij$ to $X_A$.

Let $\uttrace$ and $\uttrace'$ be two normalized traces of $\M$.  Define the
following adaptation of the Nerode congruence.
Let $\uttrace \equiv^{\timerequiv} \uttrace'$ denote that
$\timerequiv: \domof{\uttrace} \rightarrow \domof{\uttrace'}$ is a bijection
such that there is a bijection $\timerbij$ on $X$ with
$\timerequiv \extendedby\timerbij$ such that
\[
\uttrace\conc \strace \in \uttracesof{\M}
\qquad \mbox{if and only if} \qquad
\uttrace'\conc \timerequiv(\strace) \in \uttracesof{\M}
\]
Let $\uttrace \equiv_{\M} \uttrace'$ denote that 
$\uttrace \equiv^{\timerequiv} \uttrace'$ for some bijection
$\timerequiv: \domof{\uttrace} \rightarrow \domof{\uttrace'}$.
The equivalence $\equiv_{\M}$ can now be used to minimize $\M$, define
canonical forms, equivalence between MMnTs, etc., using standard techniques
(see, e.g.,~\cite{LeeY96}).
We would also want to use it as a basis for an automata learning algorithm.

\section{Automata Learning for MMnTs}
In this section, we present an algorithm for automata learning for MMnTs.
It is based on the Nerode equivalence $\equiv_{\M}$, which is used to adapt
the standard paradigm for automata learing, using $L^*$, which is also adapted
to register automata in~\cite{CasselHJS16}. 

Thus, two tasks must be accomplised in order to design the algorithm.
\begin{itemize}
  \item
    One task is to infer untimed traces of am MMnT.
    This is done by supplying membership queries, in the form of inputs
    with specific timing.
\item
  One task is to define suitable approximations of $\equiv_\M$, which are
  based on finite sets of suffixes. We must determine how to define suitable
  finite sets of suffix traces, which are used to define overapproximations
  of $\equiv_\M$.
\end{itemize}
We assume that we can observe
exact timing of events and that we can observe whenever a timeout event
happens. We cannot directly observe the setting of timers, or the identity
of the timer in a timeout event. However, whenever a timeout event, we
can infer where it was set, by supplying a small sequence of membership queries
with slightly perturbed timing. This is a consequence of the fact that we
assume MMnTs to be robust, implying that we can always include ``slack'' in
the timing of inputs. The normalized name of the timer follows after inferring
in which transition it was set.

Let us begin with the second task.
Intuitively, we are looking for a mechanism for characterizing a finite
set of suffixes, so that we can adapt the definition of
$\uttrace \equiv^{\timerequiv} \uttrace'$ to consider specific subsets of
suffixes. The natural way to do this is, like~\cite{Nie03}, to define
finite sets of input sequences. In our setting, a complication is that
input sequences include timer events, of form $\toevent{x}$, and that
after different traces, timer events with different timers may or may
not be enabled. We solve this complication by omitting the
identity of the timer $x$ in input sequences that characterize suffixes.

Let an \emph{input prefix} be a sequence of extended inputs.
For a trace $\uttrace$, let $\pinpof{\uttrace}$ be the sequence of its
inputs an timer events.
Let a \emph{timeout symbol} be of form $\tosymbol$, i.e., a
timer event without any clock.
Let an \emph{input suffix} be a sequence of inputs and timeout symbols.
For a suffix trace $\strace$, let $\sinpof{\strace}$ be the sequence
of inputs and timeout symbols in $\strace$, i.e., $\sinpof{\strace}$ is the
sequence of extended inputs, but with timers removed from timer events.

A central mechanism in our active learning algorithm is a procedure which
takes  an input prefix $u$ and an input suffix $v$ and return the set of
untimed traces of form $\uttrace\conc\strace$ such that
$\pinpof{\uttrace} = u$ and $\sinpof{\strace} = v$. We must here take into
consideration that we may not be able to observe all timer settings of the
SUT; only those timer settings that can actually expire sometime during
$\uttrace\conc\strace$ can actually be observed.
This may further depend on the timing of transitions
in $\uttrace\conc\strace$.

%% Let us now consider the issue that timeout events are not always
%% enabled.
%% Consider a trace $u$ of $\M$, and assume that timer $x$ is set in the
%% $i$th event of $u$. Then, in each state the corresponding timeout even
%% $\toevent{x}$ may or may not be enabled; this may
%% further depend on the timing of events in $u$, including
%% the timing of events that precede the $i$th event.
%% This motivates the following definitions.

Motivated by these considerations,
let $\uttrace = \tuple{i_1,o_1,\remapinst_1}\cdots\tuple{i_n,o_n,\remapinst_n}$.
We say that $\uttrace$ is {\em feasible} if 
there is a corresponding timed word
$\tword = (i_1, o_1, t_1) \cdots (i_n, o_n, t_n)$ for some $\vect tn$.
We then say that $\tword$ is {\em derived} from $\uttrace$.
Let $\expirable(\uttrace)$ be the set of timers $x$ such that
$\tuple{i_1,o_1,\remapinst_1}\cdots\tuple{i_n,o_n,\remapinst_n}
\tuple{\toevent{x},o_{n+1},\remapinst_{n+1}}$
is feasible for some $o_{n+1}$, $\remapinst_{n+1}$.
For $u = \pinpof{\uttrace}$ we say that $u$ is feasible iff $\uttrace$ is
feasible, and define $\expirable(u) = \expirable(\uttrace)$.

%% For a sequence $u = i_1 \cdots i_n$, define $\uttraceof{u}$ as the normalized
%% trace $\tuple{i_1,o_1,\remapinst_1}\cdots\tuple{i_n,o_n,\remapinst_n}$
%% if it exists, otherwise $\uttraceof{u} = \bot$. We say that $u$ is {\em feasible}
%% if $\uttraceof{u}$ is defined and is feasible. Define
%% $\expirable(u)$ to be $\expirable(\uttraceof{u})$.
%% It follows that $u = i_1 \cdots i_n$ is feasible if and only if whenever
%% $i_k$ is a timeout event $\toevent{x}$ for some $k$ with $1 < k \leq n$, then
%% $x \in \expirable(i_1 \cdots i_{k-1})$.
%% We will later present a technique for inferring the set $\expirable(u)$.

\paragraph{Membership Queries}
The procedure for inferring untimed traces will utilize membership queries.
We will slightly adapt the notion of membership query to make it convenient
in our current setting.

A {\em membership query} is an alternating sequence
$t_1i_2t_2i_2 \cdots t_ni_nt_{n+1}$ of delays and inputs in $\extinputs$.
The response to a membership query is either
\begin{itemize}
\item
  a sequence $o_1 o_2 \cdots o_n$ of outputs in $O$, meaning that there is
  a corresponding timed word $(i_1, o_1, t_1) \cdots (i_n, o_n, t_n)(i_{n+1}, o_{n+1}, t_{n+1})$ for some $i_{n+1}$, $o_{n+1}$, $t_{n+1}'$ with $t_{n+1} \leq t_{n+1}'$,
  or
\item
  a timeout event after some prefix $t_1i_2t_2i_2 \cdots t_{k-1}i_{k-1}t_{k}'$
  where $k \leq n+1$ and $t_k' \leq t_k$, implying that  there is
  a corresonding timed word $(i_1, o_1, t_1) \cdots (\toevent{x}, o_k, t_k')$
  for some timer $x$. By the assumption of robustness, it is easy to
  determine where the timer $x$ was set. Therefore the response to the
  membership query is the index $k$ and the normalized timer name
  $\timerof jp$, or
\item
  a {\em missing timer event} after some prefix
  $t_1i_2t_2i_2 \cdots t_{k-1}i_{k-1}t_{k}$, implying that input $i_k$, of
  form $\toevent{x}$, does not occur at the intended time.
\end{itemize}

\paragraph{Inferring traces}
We can now present a central procedure of our learning algorithm,
which is one that 
takes a feasible input prefix $u$ and an input suffix $v$ and returns the set of
feasible (normalized) untimed traces of form $\uttrace\conc\strace$ such that
$\pinpof{\uttrace} = u$ and $\sinpof{\strace} = v$.
%% restricted to the set of timers that are in
%% $\expirable(i_1 \cdots i_{k})$ for some $k \leq n$.
We assume that names of timers are normalized.

%% Consider a trace 
%% $\tuple{i_1,o_1,\remapinst_1}\cdots\tuple{i_n,o_n,\remapinst_n}$.
%% and let 
%% $w = (i_1, o_1, t_1) \cdots (i_n, o_n, t_n)$ be a corresonding timed word.
%% We introduce the following canonical naming convention for timers.
%% A timer which is set to $p$ at transition $j$ (i.e., in response to
%% input $i_j$ is given the name $\timerof jp$. This means that
%% for a timer $\timerof jp$ we have
%% $\remapinst_j(\timerof jp) = p$, and that 
%% $\timerof jp \in \remapinst_k$ only if $j \leq k$ and $\timerof jp$ has
%% not expired at or before $i_k$, and then 
%% $\remapinst_k(\timerof jp) = \timerof jp$ if $j < k$.
%% We also extend $w$ by a delay $t_{n+1} \in \realsplus$, which intuitively
%% represents that time $t_{n+1}$ elapses after the trace; such a delay
%% may be observed only if it is not interrupted by a timeout event.


%% For each timer setting in the trace, i.e., each occurrence of
%% a timer  $x \in \domof{\remapinst_i}$ with  $\remapinst_i(x) \in \natplus$,
%% defined the constraint $\constrof{\word}$ as
%% \begin{itemize}
%% \item
%%   if $i_k$ is the timeout event $\toevent{x}$, then
%%   $\constrof{\word}$ is $\delay{i}{k}= \remapinst_i(x)$,
%% \item
%%   if the trace includes no timeout event $\toevent{x}$, then
%%   $\constrof{\word}$ is $\delay{i}{(n+1)} \leq \remapinst_i(x)$.
%% \end{itemize}
%% Define $\constrof{\word}$ as
%% the conjunction of $\constrof{\word}$ for all timer settings in the trace.

The procedure must find all ways to instantiate the timeout symbols in $v$
with actual timers to obtain a feasible instantiation of $u\conc v$, and for
each such instantiation determine the set of timers that may expire after some
prefix, and how they are manipulated.

Our procedure considers increasing prefixes $z$ of extended inputs that
instantiate some prefix of $u \conc v$. For each such prefix, it
determines the set $\expirable(z)$, the constraint $\constrof{z}$
under which $z$ can be performed without any succeeding timer event, 
the timers that are in $\expirable(z')$ for some prefix $z'$ of $z$, and
their manipulation.

The procedure performs repeated membership queries for selected values
of $\vect{t}{n+1}$, and gradually adds new timers to the word together
with constraints on $\vect{t}{n+1}$ induced by these timers.

Let $\delay{j}{k}$ denote $t_{j+1} + t_{j+2} + \cdots + t_k$.
%% This means that $\remapinst_j(\timerof jp) = p$, and that 
%% $\remapinst_l(\timerof jp) = \timerof jp$ if $j < l < k$.
%% This means that $\timerof jp \in \domof{\remapinst_l}$ iff
%% $j \leq l < k$, where $\remapinst_j(\timerof jp) = p$ and
%% $\remapinst_l(\timerof jp) = \timerof jp$ for $j < l < k$.
%% Initially, $\constrof{\word}$ is the conjunction of the constraints of form
%% $\delay{j}{k}=m$ for each $i_k$ of form is $\toevent{\timerof jp}$.

%% The procedure considers prefixes in increasing order. For each input prefix
%% $z$, it
%% \begin{itemize}
%% \item determines all timers that are in $\expirable(z)$,
%% \item gradually builds a constraint $\constrof{z}$ under which $z$ can
%%   be performed without interruption by unexpected timer events,
%% \item
%%   gradually builds the mappings $\remapinst_{z'}$ for prefixes $z'$ of $z$.
%% \end{itemize}

\medskip
\noindent{\bf Algorithm} \textsl{Infer-Traces.}
\begin{description}
\item[Input:] (normalized) feasible input prefix $u$ and input suffix $v$,
\item[Returns:] set of (normalized) traces
$\uttrace\conc\strace$ such that
$\pinpof{\uttrace} = u$ and $\sinpof{\strace} = v$, together with
  constraint $\constrof{\uttrace\conc\strace}$ on derived timed words.
\item[Initialization:]
    Let $z = \emptyword$, let $\constrof{z} = \true$;
\item[Return:]
    $\textsc{Infer-Traces}(\emptyword,\emptyword,\true)$
%%     \begin{itemize}
%% \item initialize $\vect{\remapinst}n$ by letting
%%     $\timerof jp \in \domof{\remapinst_l}$ iff $i_k$ is $\toevent{\timerof jp}$
%%     with $j \leq l < k$; let $\remapinst_j(\timerof jp) = p$ and
%%     $\remapinst_l(\timerof jp) = \timerof jp$ for $j < l < k$;
%% \item let $\constrof{\uttrace}$ be
%%     the conjunction of the constraints of form
%% $\delay{j}{k}=p$ for each $i_k$ of form $\toevent{\timerof jp}$.
%%     \end{itemize}
\end{description}
\medskip
\noindent{\bf Function} $\textsc{Infer-Traces}(z,\uttrace,\constrof{\uttrace})$;
\begin{enumerate}
\item
  let $z = i_1 \cdots i_m$;
  let $\uttrace = \tuple{i_1,o_1,\remapinst_1}\cdots\tuple{i_m,o_m,\remapinst_m}$;
\item
  {\bf for} each index $k = 1, \ldots , m$ {\bf do}:
     \qquad \textit{// find all timers in $\expirable(z)$}
  \begin{enumerate}
  \item
    find $\vect t{{m+1}}$ that maximizes $\delay{k}{(m+1)}$ given
    $\constrof{\uttrace}$;
  \item
    supply membership query $t_1i_2t_2i_2 \cdots t_mi_mt_{m+1}$;
  \item
    {\bf if} unexpected timeout $\toevent{\timerof jp}$ occurs within at most
    $t_{m+1}$ after $i_m$:
    \begin{enumerate}
    \item
      add $\timerof jp$ to $\domof{\remapinst_l}$ iff
$j \leq l \leq m$, where $\remapinst_j(\timerof jp) = p$ and
      $\remapinst_l(\timerof jp) = \timerof jp$ for $j < l \leq m$;
    \item
      Add conjunct 
      $\delay{j}{(m+1)} \leq p$ to $\constrof{\uttrace}$;
      \todobj{should it be $\delay{j}{(m+1)} < p$?}
    \item if $k \neq j$ go to (a);
    \end{enumerate}
    {\bf else} do nothing;
  \end{enumerate}
  {\bf od}
\item
  {\bf if} $\lengthof{z} = \lengthof{u} + \lengthof{v}$ {\bf return}
  $\tuple{\uttrace,\constrof{\uttrace}}$;
\item
  {\bf if} $i_{m+1} \in \extinputs$:
  \[  \textbf{return } \textsc{Infer-Traces}(zi_{m+1},\uttrace\tuple{i_{m+1},o_{m+1},\emptyset},\phi')
  \]
  where $\phi'$ is $\constrof{\uttrace}$ if $i_{m+1} \in I$, otherwise, if
  $i_{m+1} = \toevent{\timerof jp}$, it is obtained from $\constrof{\uttrace}$
  by replacing $\delay{j}{(m+1)} \leq p$ by $\delay{j}{(m+1)} = p$.
  \item
  {\bf else} \hspace*{4cm} \textit{// \ $i_{m+1}$ is the timeout symbol}
    \[
    \textbf{return} \mathop\bigcup_{\timerof jp \in \expirable(z)}
     \textsc{Infer-Traces}(z\toevent{\timerof jp},\uttrace\tuple{\toevent{\timerof jp},o_{m+1},\emptyset},\phi')
     \]
  where  $\phi_j^p$ is obtained from $\constrof{\uttrace}$
  by replacing $\delay{j}{(m+1)} \leq p$ by $\delay{j}{(m+1)} = p$.
\end{enumerate}
Intuitively, the function
$\textsc{Infer-Traces}(z,\uttrace,\constrof{\uttrace})$ finds
all timers in $\expirable(z)$ by testing for each $k$ whether 
it is possible
to set a timer at the $k$th transition that may expire after the
$m$th transition. Any such timer expiration adds a constraint to
$\constrof{\uttrace}$, which is included in the constraints used to find
further timers.
This is done by finding a timed word is found which maximises the
time within which it is able to expire, subject to the constraints under
which other, already detected, timers cannot expire. If no expiration is
observed, then it is ascertained that no such timer exists.
If such a timer may exist, then some expiration must be observed, either
by a timer set at the $k$th transition and expiring at the $m+1$st transition,
or by a so-far undetected timer that can expire because the membership
query exposes a new timing pattern. In either case, the new detected timer
is added to the word, and the procedure restarts.

\paragraph{Generating MMnTs}
Having solved the problem of inferring untimed traces, we can now
define our generalization of $L^*$, which is based on the
Nerode equivalence defined earlier
%% We can now present our algorithm for active learning of MMnTs.
%% The main structure of the algorithm is inherited from other active
%% algorithms: it builds on a Nerode equivalence, which is defined based on
%% the set of (untimed) traces of an MMnT.
%% Thus, the algorithm has two nontrivial parts.
%% \begin{itemize}
%%   \item
%%     One part is to infer untimed traces of am MMnT, essentially building
%%     a prefix tree of untimed traces. This is done by supplying membership
%%     queries, following the ideas of the previous section.
%% \item
%%   One part is to ``fold'' the prefix tree developed in the first part
%%   into a MMnt. This can be done by adapting the suitable tools for active
%%   automata learning, as found in $L^*$
%%   (adapted for Mealy machines in~\cite{Nie03} and to automata with
%%   data registers in~\cite{CasselHJS16}).
%% \end{itemize}
%% We assume that we can observe
%% exact timing of events and that we can observe whenever a timeout eMMvent
%% happens. On the other hand, we cannot observe the setting and resetting of
%% timers, and we cannot observe which timer caused an observed timeout event.


\begin{definition}[Observation Table]
  An \emph{observation table} is a tuple
  $\tuple{U,U^+,V,\smap}$, where
  \begin{itemize}
  \item $U$ is a set of (feasible) {\em input prefixes},
  \item $U^+$ is a set of {\em extended input prefixes}, each of form $ui$ for
    $i \in \extinputs$ (note that $U^+$ is in general not disjoint from $U$),
  \item
    $V$ is a set of input suffixes,
  \item
    $\smap$ maps each extended input prefix $u$ in $U^+$ and input suffix $v$
    to a set of pairs $\tuple{\uttrace,\strace}$ such that 
    $\uttrace\conc\strace$ is an untimed trace with
$\pinpof{\uttrace} = u$ and $\sinpof{\strace} = v$.
   We use $\smappre(u,v)$ for $\uttrace$ and 
   $\smapsuf(u,v)$ for $\strace$.
  \end{itemize}
\end{definition}
For an observation table and $u \in U^+$,
let $\obspre(u)$ denote
\(
\displaystyle\mathop\sqcup_{v \in V}\smappre(u,v)
\),
i.e., $\obspre(u)$ includes manipulation of all timers that may be
exposed by any of the suffixes in $V$.
The observation table defines the following equivalence
Let $u \equiv_{\smap}^{\timerequiv} u'$ denote that
$\timerequiv: \domof{\obspre(u)} \rightarrow \domof{\obspre(u')}$
is a bijection such that there is a bijection $\timerbij$ on $X$ with
$\timerequiv \extendedby\timerbij$ such that for each $v \in V$
\[
\strace \in \smapsuf(u,v)
\qquad \mbox{if and only if} \qquad
\timerequiv(\strace) \in \smapsuf(u',v)
\]
Let $u \equiv_{\smap} u'$ denote that 
$u \equiv_{\smap}^{\timerequiv} u'$ for some bijection
$\timerequiv: \domof{\obspre(u)} \rightarrow \domof{\obspre(u')}$.

Say that an observation table is {\em closed} if for each
$ui \in U^+$ there is a short prefix $u' \in U$ and a $\timerbij$ such that
$ui \equiv_{\smap}^{\timerequiv} u'$.

A closed observation table $\tuple{U,U^+,V,\smap}$ can be used to construct
a hypothesis  MMnT $\M = (I, O, Q, q_0, \vars, \delta, \lambda, \remap)$, where
\begin{itemize}
\item
  $Q = U$ and $q_0 = \emptyword$,
\item
  $\vars$ maps each state $u \in Q$ to $\domof{\obspre(u)}$,
\item
  $\delta(u,i)$ is the unique short prefix $u'$ such that
  $ui \equiv_{\smap}^{\timerequiv} u'$. If so, then
  $\remap(u,i) = \remap\circ\timerequiv^{-1}$,
  where $\remap$ is the remapping of the last transition in
  $\obspre(ui)$.
  \item
$\lambda$ is obtained easily from the work (To be done)
\end{itemize}









