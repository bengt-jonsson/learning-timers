
\section{Conclusions and Future Work}
\label{conclusions}

We have presented a new automaton-based model for timed systems, MMTs,
which aims to
be sufficiently simple to allow tractable learning algorithms, and sufficiently
expressive to model common network protocols. For the MMT model we have
developed a Nerode congruence, allowing to define canonical forms, and used it
as the basis for an active learning algorithm, which generalizes $L^*$.
A key technical result is the equivalence between the timed semantics,
which is suitable to represent practical learning scenarios, and the
untimed semantics, which is suitable for formulating automata learning
algorithms. This equivalence is embodied by an adaptor, which transforms
queries in one model to queries in the other.

The query complexity of our learning algorithm is polynomial in the number of
states of the learned MMT, but doubly exponential in the number of simultaneously
active timers. Since practical network protocols have at most a couple of
simultaneously active timers, this leads us to believe that our work will
be a suitable theoretical basis for practical learning algorithms for timed
protocols.

%% Our work consitutes a major step towards a practical approach for active learning of timed systems.
%% Such an approach would greatly enhance the applicability of active learning for reverse engineering of models of
%% software and hardware systems.
Future work includes to implement equivalence queries for MMTs.
In the untimed case, equivalence queries for Mealy machines are approximated using conformance testing algorithms,
for which a rich theory exists \cite{LeeY96}.
Our equivalence result between timed and untimed semantics may help to
lift such algorithms to the setting of MMTs.
A challenge is also to deal with the timing uncertainties that occur in real applications due to nondeterminism of
implementations and imprecise measurements. In a realistic
setting we may need more than one experiment to figure out which event causes a timeout. We may observe, for instance,
that slight changes in the timing of certain inputs lead to corresponding changes in the timing of certain timeouts.

%% Nevertheless, there are still major challenges that need to be addressed.
%% %
%% A first challenge is to implement equivalence queries for MMTs.
%% In the untimed case, equivalence queries for Mealy machines are approximated using conformance testing algorithms,
%% for which a rich theory exists \cite{LeeY96}.
%% We hope that our equivalence result for the timed and untimed semantics will help to lift existing conformance
%% testing algorithms to the setting of MMTs. For MMTs with a single timer this is easy, but we do not yet have good
%% conformance testing algorithms for MMTs with multiple timers.
%% %
%% A second challenge is to deal with the timing uncertainties that occur in real applications due to nondeterminism of
%% implementations and imprecise measurements. In the idealized setting of this paper we are able to
%% identify the triggering event for each timeout by providing inputs at different fractional times. In a realistic
%% setting we may need more than one experiment to figure out which event causes a timeout. We may observe, for instance,
%% that slight changes in the timing of certain inputs lead to corresponding changes in the timing of certain timeouts.
%
%% A third important challenge is to implement our algorithm and to apply it to practical case studies. In particular,
%% it is interesting to see how we can implement lookahead queries. For MMTs with a single timer this is trivial 
%% if we assume an upper bound $t$ on the value of timers (just  wait for $t+1$ time units).
%% But we still do not know whether we can always come up with efficient implementations of the lookahead oracle for applications with multiple timers.

%% \iflong
%% Just like the theory of timed automata \cite{AD94} paved the way to extend untimed model checking tools to a timed
%% setting, we hope that our work on MMTs will make it possible to lift untimed model learning tools to a timed setting.
%% MMTs can be viewed as a restricted subclass of timed automata in which clocks run backward instead of forward.
%% Many theoretical questions about MMTs, such as the complexity of equivalence checking and model checking, are still open.
%% \fi
