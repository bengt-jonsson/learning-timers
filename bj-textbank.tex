\section{Another Try on the Precise Nerode Equivalence}

Let us introduce notation to make the subsequent development more convenient. 
Recall that a canonical untimed behavior is \emph{canonical} when, for each $j$,
the timer that is updated in the $j$-th event (if any) is equal to $x_j$.
It is {\em lean} if it is canonical and
includes only timers that expire during the behavior. 
Since timers in lean behaviors are uniquely determined by the labels on
transition, we can denote lean behaviors simply as sequences
$i_1/o_1,\rho_1 \cdots i_n/o_n,\rho_n$ of input/output/assignment triples.
For behaviors that are to be appended as suffixes, we need to distinguish
timers that are assigned in the suffix from timers that are assigned in a prefix.
Therefore, extend the set of timers by the set $Y = \set{y_1,y_2,\ldots}$ of
{\em suffix timers}, which is disjoint from $X$.
Define a {\em suffix behavior} to be a sequence
$i_1/o_1,\rho_1 \cdots i_m/o_m,\rho_m$ of input/output/assignment triples,
in which the inputs are in $I$, of form $\toevent{x_j}$ for $x_j \in X$, or
of form $\toevent{y_j}$ for $y_j \in Y$, but where assignments may assign only
to timers in $Y$. Such a suffix behavior is {\em lean} if
timer $y_j$ is assigned only by $\rho_j$, each timeout event occurs at most once, and all assigned timers in $Y$ expire sometime after their assignment.

For integer $k \geq 0$, let $\suffmap{k}$ be the injective mapping on
$Y$ which maps each $y_j$ to $y_{j+k}$.  We apply mappings $\suffmap{k}$ to
lean suffix behaviors in the natural way.

Let $\beta$ be a lean behavior. We say that 
a lean suffix behavior $\gamma$ is a {\em $\beta$-suffix} if
there is a canonical behavior $\beta'$ with $\beta \sqsubseteq \beta'$ such that
$\beta'\cdot\suffmap{|\beta|}(\gamma)$ is a lean behavior.

We can define general residual languages, by letting
$\suffixbehs{S}{\beta}$ be the set $\beta$-suffixes.
Let $\getmemorable{S}{\beta}$ be the set of timers $x_i$ in
$x_1 , \ldots x_{|\beta|}$ whose corresponding timeout event
(of form $\toevent{x_i}$) occurs (as an input) in some $\beta$-suffix.
Let $\getassignment{S}{\beta}$ map each timer $x_i$ in
$\getmemorable{S}{\beta}$ to the unique positive integer to which it
is assigned in the $i$th transition of $\beta$.

We can now defined the Nerode equivalence.

\begin{definition}
  \label{def:bj-nerode}
Let $S$ be a timer language \todobj{Define this},
let $\beta$ and $\beta'$ be canonical untimed behaviors in $S$, 
and let $f : \getmemorable{S}{\beta} \to \getmemorable{S}{\beta'}$
be a bijection
from $\getmemorable{S}{\beta}$ to $\getmemorable{S}{\beta'}$.
Then $\beta$ and $\beta'$ are \emph{equivalent} under $f$, written
$\beta \equiv_{S}^f \beta'$ iff
\[
\gamma \in \suffixbehs{S}{\beta}
\qquad \mbox{iff} \qquad
f(\gamma) \in \suffixbehs{S}{\beta'}
\]
\end{definition}
Intuitively, $\beta \equiv_{S}^f \beta'$ means that $\beta$ and $\beta'$
allow the same suffixes, after renaming
timers assigned in $\beta$ by $f$.
We write $\beta \equiv_{S} \beta'$ to denote that
$\beta \equiv_{S}^f \beta'$ for some
$f : \getmemorable{S}{\beta} \to \getmemorable{S}{\beta'}$.

\begin{theorem}
  \label{thm:bj-nerode}
Let $S$ be a timer language.
Then there exists an MMT with feasible canonical untimed behaviors $S$ iff 
$\equiv_S$ has finitely many equivalence classes (finite index).
\end{theorem}
\begin{proof} 
``$\Leftarrow$'' This direction will produce the following MTT:
\begin{definition}[Constructed MMTautomaton]
\label{def:mmt-construction}
The constructed MMT is the MMT
$\M = (I, O, Q, q_0, \vars, \delta, \lambda, \remap)$, where
\begin{itemize}
\item $Q$ is the set of equivalence classes under $\equiv_S$. We let each
  equivalence class be represented by one of its members, called its
  {\em access string}. In the following, when we write $[\beta]$,
  this denotes that $\beta$ is the access string of the equivalence class $[\beta]$.
\item $q_0 = \emptyword$,
\item $\vars$ maps each equivalence  class $[\beta]\in U$ to
  $\getmemorable{S}{\beta}$,
\item $\lambda$ maps each $[\beta] \in U$ and $i \in \feasibleinputs{\beta}{S}$ to
  the unique $o$ such that $\extend{\beta}{i}{o} \in S$. 
  %% $i$, either in $I$ or of form $\toevent{x_i}$, to the output observed in
  %% reponse to $i$ after $\beta$.
\item $\delta$ maps each $\beta \in U$ and
  $i \in \feasibleinputs{\beta}{S}$ to
  to the unique $\beta' \in U$ such that there is an $f$ with
  $\extend{\beta}{i}{\lambda(\beta,i)} \equiv_{S}^f \beta'$.
\item $\remap$
 maps each $\beta \in U$ and $i \in \feasibleinputs{\beta}{S}$ to
  the function
  $\remap(\beta,i):\getmemorable{S}{\beta'} \mapsto (\getmemorable{S}{\beta} \cup \natplus)$ defined as $\rho \circ f^{-1}$, where $\rho$ maps 
  timer $x_{(|\beta|+1)}$ to
$\getassignment{S}{(\extend{\beta}{i}{\lambda(\beta,i)})}(x_{(|\beta|+1)})$.
\end{itemize}
\end{definition}
\todobj{The rest of this is just a sketch. More care is needed with timers}
``$\Rightarrow$'' Let $\M$ be an MMT, let $S$ be its set of feasible untimed
  behaviors. It follows from \todobj{fix} that $S$ is a timer language.
Suppose that $\beta, \beta' \in S$.
Then there exist (unique) $\delta, \delta$ of $T$ with
$\can{\delta}=\beta$ and $\can{\delta'}=\beta'$.
Suppose $\delta$ and $\delta'$ lead to the same state $q$ of $\M$.
We claim that $\beta \equiv_S \beta'$.
Since $\beta = \can{\delta}$, there exist isomorphisms
$g = g_0 ,\ldots, g_k$  $\beta$ to $\delta$.
Similarly, there exists isomorphisms $g' = g'_0 ,\ldots, g'_l$ from
$\beta'$ to $\delta'$.
Let $f = (g'_l)^{-1} \circ g_k$.
Suppose that, for some untimed behavior $\gamma$, $\beta \cdot \gamma \in S$.
Then there exists an untimed behavior $\zeta$ such that $\delta \cdot \zeta \in T$ and $\can{\delta \cdot \zeta} = \beta \cdot \gamma$.
Moreover, there exists an isomorphism $h = g_k ,\ldots, g_m$ from $\gamma$ to $\zeta$.
Since $\delta \cdot \zeta$ is a feasible behavior in $T$, it follows that
$\delta' \cdot \zeta$ is a feasible behavior in $T$
\todobj{This should follow from the observation that the same paths are possible
  after $\delta$ and $\delta'$ (we have the assumption that all paths are
  feasible}
Thus there exists an untimed behavior $\gamma'$ such that $\can{\delta' \cdot \zeta} = \beta' \cdot \gamma' \in S$.
Moreover, we may construct an isomorphism $h' = g'_l ,\ldots, g'_n$ from $\gamma'$ to $\zeta$.
Then $f = (g'_l)^{-1} \circ g_k \cdots (g'_n)^{-1} \circ g_m$ is an isomorphism from $\gamma$ to $\gamma'$ that extends $f_0$.
We then have $\beta' \cdot f(\gamma) \in S$, as required.
\todobj{To be continued}
\end{proof}


\section{Realizing untimed queries by timed queries}
\todobj{This section has been untouched for a while. Ignore it for now.}

In order to use the above approximated equivalence for learning, we need to
present a procedure for constructing the set
$\apprsuffixbehs{S}{\beta}{V}$ by a sequence of timed membership queries.

We should also construct an automaton from a set of results of such
suffix queries, provided they satisfy some conditions
\todobj{closedness, etc, which we must define}

Let us first present a procedure for answering suffix queries. It is
sufficient to perform them for one generic suffix at a time.
The input to a suffix query is a (feasible) untimed behavior $\beta$ an a
generic untimed input behavior $v$. It returns
$\apprsuffixbehs{S}{\beta}{\set{v}}$. To compute this, it must be determined
\begin{itemize}
\item which instantiations of generic timeouts in $v$ are feasible after $\beta$, and
\item what is the output in response to an input.
\end{itemize}
The second item is trivially answered by observing the output in response to
each relevant input. Note that if a behavior $\beta$ is feasible, then
any extension of $\beta$ by a non-timeout input is trivially feasible, simply
by supplying the input as fast as possible. Thus the only non-trivial part
of answering a suffix query is to determine which instantiations of a generic
timeout in $v$ are feasible. This is easily reduced to the problem of
determining, for a feasible behavior $\beta$, which timeouts of form
$\toevent{x_i}$ with $i \leq |\beta|$ are feasible (note that $\beta$ here in
general denotes an extension of the $\beta$ supplied as prefix in a
suffix query). Let us now describe how this can be done.

In order to determine which behaviors of form $\beta \cdot \toevent{x_i}$
are feasible, we perform a timed membership query, which provide the longest
possible time delay between the $i$th transition in $\beta$ and the time
of any other timeout that can feasibly expire after $\beta$. In order to
maximize this time, we need to know which other timeouts may expire while
performing $\beta \cdot \toevent{x_i}$. Such timeouts are either expiring
as part of $\beta$, or disturb the execution of $\beta \cdot \toevent{x_i}$,
in which case our membership query must know how to prevent them from occuring.

Les us introduce notation. Let $\beta$ be of form
\(
  {i_1/o_1, \rho_1} \ \cdot \ {i_n/o_n, \rho_n}
\)
We are looking for a corresponding timed behavior of form 
\[
\tvals_0 \dtrans{d_1} \tvals'_0 \uttrans{i_1}{o_1}{\rho_1}
\quad \cdots \quad
\dtrans{d_n} \tvals'_n \uttrans{i_n}{o_n}{\rho_n} \tvals_n
\dtrans{d_{n+1}} \tvals'_{n+1} \uttrans{\toevent{x_i}}{o_{n+1}}{\rho_n}
\]
The timed membership queries need to select values
of $\vect{d}{n+1}$, and gradually adds new timers to the word together
with constraints on $\vect{d}{n+1}$ induced by these timers.
Let $\delay{j}{k}$ denote $d_{j+1} + d_{j+2} + \cdots + d_k$.

The algorithm maintains an increasing constraint $\constr$ on delays of form
$\delay jk$. Initially, the constraint contains the conjunct
$\delay jk = D$ whenever $i_k$ is of form $\toevent{x_j}$ and
$\rho_j$ sets $x_j$ to $D$.
Thereafter, the following procedure is performed
  \begin{enumerate}
  \item
    find $\vect d{{n+1}}$ that maximizes $\delay{i}{(n+1)}$ given
    $\constr$ (note that $d_{n+1} = \infty$ is allowed)
  \item
    supply membership query with delay values $\vect d{{n+1}}$.
\begin{itemize}
  \item
    {\bf if} the output is of form $\beta$ followed by $\toevent{x_i}$, then
    report that $\toevent{x_i}$ is feasible and stop
  \item
    {\bf else if} no timeout occurs within $d_{n+1}$ after $\beta$, report
    that $\toevent{x_i}$ is not feasible and stop
  \item
    {\bf else if} another timeout of form $\toevent{x_j}$ occurs before
    the $k$th input, this means that timer $x_j$ is assigned a value
    $D_j$ at the $j$ transition and may expire before the $k$th transition
    in the untimed trace. In order to prevent this from happening,
    add the constraint $\delay jk < D_j$ to $\constr$ and start again
    from step 1.
\end{itemize}
  \end{enumerate}
  Intuitively, the procedure aims to find a timed membership query which maximise
  the time within which $\toevent{x_i}$ is able to expire,
  subject to the constraints under which timers that expire in $\beta$ expire
  as scheduled, and under which any other timers do not expire. These
  other timers may not be known when the procedure starts, and so their
  corresponding constraints are added when they are detected: this happens
  when they expire unexpectectly.

  Note that conjunctions of constraints on delays $\delay{j}{k}$ can be represented as DBM, as is standard for timed automata.




