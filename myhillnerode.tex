\subsection{A Myhill Nerode Theorem for MMTs}

\begin{definition}
Let $S$ be a set of feasible untimed behaviors in canonical form over $I$ and $O$. Then $S$ is
\emph{prefix closed} if $\beta \beta' \in S \Rightarrow \beta \in S$,
\emph{behavior deterministic} if
$\beta \xrightarrow{i/o_1, \rho_1} X_1 \in S \wedge \beta \xrightarrow{i/o_2, \rho_2} X_2 \in S \Rightarrow o_1 = o_2 \wedge \rho_1 = \rho_2 \wedge X_1 = X_2$,
\emph{input complete} if
$\beta \in S \wedge i \in I \Rightarrow \exists o, \rho, Y : \beta \xrightarrow{i/o, \rho} Y \in S$,
and
\emph{timeout complete} if
$\beta \in S \wedge x \mbox{ expirable after } \beta \Rightarrow
\exists o, \rho, Y: \beta \xrightarrow{\toevent{x}/o, \rho} Y \in S$.

Suppose $\beta$ and $\beta'$ are behaviors in $S$ with $\Last{\beta} = X$ and $\Last{\beta'} = X'$.
Then $\beta$ and $\beta'$ are \emph{equivalent}, written $\beta \equiv_S \beta'$, iff there exists a bijection
$f_0 : X \to X'$ such that for any untimed behavior $\gamma$:
(a) if $\beta \cdot \gamma \in S$ then there exists an isomorphism $f$ that extends $f_0$ such that $\beta' \cdot f(\gamma) \in S$, and conversely,
(b) if $\beta' \cdot \gamma \in S$ then there exists an isomorphism $f$ that extends $f_0^{-1}$ such that $\beta \cdot f(\gamma) \in S$.
We write $[\beta]$ to denote the equivalence class of $\beta$ with respect to $\equiv_S$.
\end{definition}

\begin{theorem}
\label{Myhill Nerode}
Let $S$ be a set of feasible untimed behaviors in canonical form over finite sets of inputs $I$ and outputs $O$.
Then there exists an MMT with feasible untimed behaviors $T$ such that $S = \can{T}$ iff 
$S$ is nonempty, all untimed behaviors in $S$
start with the empty set of timers, $S$ is prefix closed, 
there is a bound on the number of timers that can be active simultaneously 
($\exists n \in \nat ~ \forall \beta \in S : \mid \Last{\beta} \mid \leq n$),
$S$ is behavior deterministic, input complete, timeout complete,
and $\equiv_S$ has only finitely many equivalence classes (finite index).
\end{theorem}
\iflong
\begin{proof} (Needs work)

``$\Rightarrow$'' Let $\M$ be an MMT, let $T$ be its set of feasible untimed behaviors, and let $S = \can{T}$.
Then it is immediate from the definitions that $T$ is nonempty, all untimed behaviors in $S$
start with the empty set of timers, 
$S$ is prefix closed, 
there is a bound on the number of timers that can be active simultaneously,
$S$ is behavior deterministic, input complete, and timeout complete.
Suppose that $\beta, \beta' \in S$.
Then there exist (unique) $\beta_1, \beta'_1 \in T$ with
$\can{\beta_1}=\beta$ and $\can{\beta'_1}=\beta'$.
Suppose $\beta_1$ and $\beta'_1$ lead to the same state $q$ and moreover $\Post_{\beta_1} = \Post_{\beta'_1}$.
Since $\beta$ and $\beta_1$ are isomorphic, and $\beta'$ and $\beta'_1$ are isomorphic,
there exists a bijection $f_0 : X \to X'$ such that $f_0(\Post_{\beta}) = \Post_{\beta'}$.
Suppose that, for some untimed behavior $\gamma$, $\beta \gamma \in S$.
Then there exists an untimed behavior $\gamma_1$ such that $\beta_1 \gamma_1 \in T$ and $\can{\beta_1 \gamma_1} = \beta \gamma$.
By Lemma~\ref{lemma: feasibility concatenation},
$\beta'_1 \cdot \gamma_1 \in T$.
This implies $\beta' f(\gamma) \in S$ (elaborate this).
Since $\M$ only has finitely many states and by Lemma~\ref{lemma finitely many zones} the set
$\{ \Post_{\beta} \mid \beta \mbox{ is a feasible untimed behavior of } \M \}$ is finite, this means that
$\equiv_S$ has finite index.

``$\Leftarrow$'' Suppose $S$ is nonempty, etc.
Let $n$ be a bound on the number of timers that are active in a behavior in $S$ at any point.
We can construct a function that maps each untimed behavior $\beta \in S$
to an isomorphic behavior $\uncan{\beta}$ in such a way that only timers $\{ x_1 ,\ldots, x_n \}$ occur 
in all the behaviors in the range of $\mathit{uncan}$.
For instance, if a transition occurs in a state and this transition
starts a new timer, then $\mathit{uncan}$ may compute the smallest index $j$
such that $x_j$ is not unaffected by this transition, and then choose $x_j$ as the name of the new timer.
Let MMT $\M = (I, O \cup \{ \perp \}, Q, q_0, \vars, \delta, \lambda, \remap)$ be defined as follows:
\begin{itemize}
\item
$\perp$ is a special, fresh output event.
\item
$Q = \{ ( [\beta], \Post_{\uncan{\beta}}) \mid \beta \in S \}$.
\item
$q_0 = ([\emptyset], Z_0)$. (Note that $\emptyset \in S$ since $S$ is nonempty, prefix closed, and all untimed behaviors in $S$ start
with the empty set of timers; $Z_0$ is the unique zone for the empty set of variables.)
\item
$\varsof{([\beta], Z)} = \domof{Z}$.
\item
(rest needs doing). Let $\beta \in S$ and $i \in I$. Then, since $S$ is both input complete and behavior deterministic, there exist unique
$o$, $\rho$ and $Y$ such that $\beta' = \beta \xrightarrow{i/o, \rho} Y \in S$.
We define $\delta([\beta],i) = [\beta']$, $\lambda([\beta],i) = o$, and $\remap([\beta],i) = \rho$.
\item
Assume $\beta \in S$ and $x\in \Last{\beta}$ expirable after $\beta$. 
Then, since $S$ is both timeout complete and behavior deterministic, there exist unique
$o$, $\rho$ and $Y$ such that $\beta' = \beta \xrightarrow{\toevent{x}/o, \rho} Y \in S$.
We define $\delta([\beta],\toevent{x}) = [\beta']$, $\lambda([\beta],\toevent{x}) = o$, and $\remap([\beta],\toevent{x}) = \rho$.
\item
Assume $\beta \in S$ and $x \in \Last{\beta}$ not expirable after $\beta$.
We define $\delta([\beta],\toevent{x}) = [\beta]$, $\lambda([\beta],\toevent{x}) = \perp$, and $\remap([\beta],\toevent{x}) = \rho_0$, where $\domof{\rho_0} = \emptyset$.
\end{itemize}
It is routine to verify that $\M$ is a well-defined MMT whose set of feasible untimed behaviors equals $S$.
\end{proof}
\fi
