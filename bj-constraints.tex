\section{Untimed Semantics: Alternative Exposition}
\label{section constraints}
In this section, we present an alternative presentation of the untimed
semantics, and of the
equivalence between untimed and timed semantics. The main idea is to concentrate
the exposition by minimizing the number of introduced concepts, and
by formulating the problem of checking feasibility as a problem
of finding a satisfying assignment to a constraint. We then treat the problem
of finding a transparent word in an analogous manner.

Note that in this exposition, we allow timer reassignments from the start.
We also do not need to restrict to timer-live MMTS.
We assume the definition of timed semantics ad in~\ref{sec:mmt},
and use $\twsof{\M}$ for the set of timed words of an MMT $\M$, meaning
that $\M \approx_{\mathit{untimed}} \N$ iff $\twsof{\M} = \twsof{\N}$.
%% \subsection{Timed Semantics}
%% \todobj{Replace this by the previous section on timed semantics}

%% Let us first slightly redefine the notion of timed run. We first redefine the
%% transition relation between configurations by the rules
%% \[
%% \frac{q = q' \quad \tvals \xrightarrow{d} \tvals'}{(q,\tvals) \xrightarrow{d} (q',\tvals')}
%% \quad\quad
%%   \frac{q \xrightarrow{i/o/\rho} q' \quad \tvals \xrightarrow{i/o/\rho} \tvals'}{(q,\tvals) \xrightarrow{i/o/\rho} (q',\tvals')}
%% \]
%% A \emph{timed run} of $\M$ over $w$ is a sequence 
%% \begin{eqnarray*}
%% \alpha & = & (q_0,\tvals_0) \xrightarrow{d_1} (q_0,\tvals_0') \xrightarrow{i_1/o_1/\rho_1} (q_1,\tvals_1) \xrightarrow{d_2} 
%% %C'_1 \xrightarrow{i_2/o_2} C_2 
%% \cdots
%% \\ && \qquad \cdots
%% \xrightarrow{d_k} (q_{k-1},\tvals_{k-1}') \xrightarrow{i_k/o_k/\rho_k} (q_k,\tvals_k)
%% \end{eqnarray*}
%% of transitions between configurations $(q_j,\tvals_j),(q_j,\tvals_j')$ of $\M$, where $(q_0,\tvals_0)$ is the initial configuration.
%% %Note that, since MMTs are deterministic (if we allow to observe the
%% %identities of timers in timeout events),
%% %for each timed word $w$ there exists at most one run over $w$.

%% A \emph{timed word} over inputs $I$ and outputs $O$ is a sequence
%% \begin{eqnarray*}
%% w & = &  d_1 ~ i_1 ~ o_1 ~ d_2 ~ i_2 ~ o_2 \cdots d_k ~ i_k ~ o_k,
%% \end{eqnarray*}
%% where $d_j \in \delays$, $i_j \in I \cup \{ \mathit{to} \}$, and $o_j \in O$.
%% To each timed run $\alpha$ we associate a \emph{timed word} by forgetting the configurations and the timers
%% in timeout events, and forgetting the timer updates in untimed events:
%% \begin{eqnarray*}
%% \timedword(\alpha) & = & d_1 ~ i'_1 ~ o_1 ~ d_2 ~ i'_2 ~ o_2 \cdots d_k ~ i'_k ~ o_k,
%% \end{eqnarray*}
%% where for all  $j$,
%% $i'_j   =   i_j$ if $i_j \in I$, and
%% $i'_j   = \mathit{to}$ if $i_j \in \toevents$.

\subsection{Untimed Semantics}
We will define an alternative semantics for MMTs, which essentially reflects the
possible feasible sequences of transitions of an MMT, i.e., the sequences of
transitions that give rise to timed runs.
We intend this semantics to be independent of names of particular states and
of particular timers. Therefore, sequences in this semantics will not include
names of states, and will employ a canonical way to give names to timers. Thus,
extend the set of timers to include $Y = \{ y_1, y_2,\ldots \}$. Intuitively,
a timer that is started in the $l$th transition is named $y_l$.
Let $\toinputs{Y}$ be $I \cup \toeventsof{Y}$.

Define an \emph{untimed behavior} (\emph{utb} for short)
over inputs $I$ and outputs $O$ to be a sequence 
\begin{eqnarray*}
  \beta & = & Y_0 \xrightarrow{\hat{i}_1/o_1/\varrho_1} Y_1  \xrightarrow{\hat{i}_2/o_2/\varrho_1} \cdots \xrightarrow{\hat{i}_k/o_k/\varrho_k} Y_{k}
  \ \ ,
\end{eqnarray*}
where $Y_0 = \emptyset$ and for $j = 1, \ldots , k$ we have
$Y_j \subseteq \set{y_1, \ldots , y_j}$, $\hat{i}_j \in \toinputs{Y}$, $o_j \in O$,
and $\varrho_j: Y_j \rightarrow (Y_{j-1} \cup \natplus)$ is an injection, such
that for $y_l \in Y_j$ we have $\varrho_j(y_l) \in \natplus$ if $l=j$, else $\varrho_j(y_l) = y_l$.
Thus, the only timer that may be started in the $j$th transition is $y_j$, and there are no reassignments between timers.

Intuitively, our goal is to define the untimed semantics of an MMT as the utbs
that correspond to sequences of transitions that are {\em feasible}, i.e.,
can give rise to timed runs. Moreoever, timers that are non-live (i.e., that will
never induce a timeout) will be removed from these utbs.
We start by 
defining a \emph{path} $\mmtpath$ of an MMT $\M$ to be a sequence of
transitions of $\M$ of form
\[
\mmtpath = q_0 \xrightarrow{i_1/o_1/\rho_1} q_1  \xrightarrow{i_2/o_2/\rho_2}
\cdots
\xrightarrow{i_k/o_k/\rho_k} q_k
\]
We will replace each state $q_j$ in a path by the corresponding set of timers
$\vars(q_j)$, but with timers renamed to satisfy the requirements for utbs.
Thus, if $\mmtpath$ is a path,  define for each $j = 0, \ldots , k$ the injection
%% where $\canmap{\mmtpath}{0}, \ldots, \canmap{\mmtpath}{k}$ is the
%% sequence of injections
$\canmap{\mmtpath}{j}: \vars(q_j) \rightarrow \set{y_1, \ldots ,y_j}$
by
\[
\canmap{\mmtpath}{j}(x_p) = \mbox{if} \ \ \rho_j(x_p) \in \natplus
\ \ \mbox{then} \ \ y_j
\ \ \mbox{else} \ \
\canmap{\mmtpath}{j-1}(\rho_j(x_p)).
\]
Intuitively, $\canmap{\mmtpath}{j}$ renames each timer $x_p$
in $\vars(q_j)$ to the timer $y_l$ whose index $l$ is the same as
the index of the transition in which $x_p$ was started.
For $\mmtpath$ as above, define $\can{\mmtpath}$ as
\begin{eqnarray*}
  \can{\mmtpath} & = & \canmap{\mmtpath}{0}(\vars(q_0)) \xrightarrow{\hat{i}_1/o_1/\varrho_1}
%%   \canmap{\mmtpath}{1}(\domof{\tvals_1})  
\cdots \xrightarrow{\hat{i}_k/o_k/\varrho_k} \canmap{\mmtpath}{k}(\vars(q_k))
\ ,
\end{eqnarray*}
where
\begin{itemize}
\item
  $\hat{i}_j   =  i_j$ if $i_j \in I$,
\item
  $\hat{i}_j   = \toevent{\canmap{\mmtpath}{j-1}(x_p)}$ if $i_j   = \toevent{x_p}$, and
\item
  $\varrho_j = (\canmap{\mmtpath}{j-1} \cup \mbox{\sl Id}_{\natplus}) \circ \rho_j \circ (\canmap{\mmtpath}{j})^{-1}$. 
\end{itemize}
We claim that $\can{\mmtpath}$ is an utb. For this, we must prove that
for $y_l \in \canmap{\mmtpath}{j}(\vars(q_j))$ we have $\varrho_j(y_l) \in \natplus$ if $l=j$, else $\varrho_j(y_l) = y_l$.
To se this, assume $y_l = \canmap{\mmtpath}{j}(x_p)$ with $x_p \in \vars(q_j)$,
and note that
\begin{itemize}
\item if $l=j$ then by the definition of $\canmap{\mmtpath}{j}$ we have
  $\rho_j(x_p) \in \natplus$, whence by $x_p = (\canmap{\mmtpath}{j})^{-1}(y_l)$
 we have $\varrho_j(y_l) = \rho_j((\canmap{\mmtpath}{j})^{-1}(y_l)) \in \natplus$,
\item
  if $l \neq j$ then by definition of $\canmap{\mmtpath}{j}$ we have
$\rho_j(x_p) \not\in \natplus$ and
  $\canmap{\mmtpath}{j}(x_p) = \canmap{\mmtpath}{j-1}(\rho_j(x_p))$, which
  implies that
  \\
  $\varrho_j(y_l) =
  \left[\canmap{\mmtpath}{j-1} \circ \rho_j \circ (\canmap{\mmtpath}{j})^{-1}\right](y_l) = y_l$.
\end{itemize}
Thus, $\can{\mmtpath}$ satisfies the constraints for being an untimed behavior.
%% If $\alpha$ is a timed run as defined above,  define
%% for each $j = 0, \ldots , k$ the injection
%% %% where $\canmap{\alpha}{0}, \ldots, \canmap{\alpha}{k}$ is the
%% %% sequence of injections
%% $\canmap{\alpha}{j}: \domof{\tvals_j} \rightarrow \set{y_1, \ldots ,y_j}$
%% by
%% \[
%% \canmap{\alpha}{j}(x_p) = \mbox{if} \ \ \rho_j(x_p) \in \natplus
%% \ \ \mbox{then} \ \ y_j
%% \ \ \mbox{else} \ \
%% \canmap{\alpha}{j-1}(\rho_j(x_p)).
%% \]
%% Intuitively, $\canmap{\alpha}{j}$ renames each timer $x_p$
%% in $\domof{\tvals_j}$ to the timer $y_l$ whose index $l$ is the same as
%% the index of the transition in which $x_p$ was started.
%% For $\alpha$ as above, define $\can{\alpha}$ as
%% \begin{eqnarray*}
%%   \can{\alpha} & = & \canmap{\alpha}{0}(\domof{\tvals_0}) \xrightarrow{i_1'/o_1/\varrho_1}
%% %%   \canmap{\alpha}{1}(\domof{\tvals_1})  
%% \cdots \xrightarrow{i_k'/o_k/\varrho_k} \canmap{\alpha}{k}(\domof{\tvals_k})
%% \ ,
%% \end{eqnarray*}
%% where
%% \begin{itemize}
%% \item
%%   $i_j'   =  i_j$ if $i_j \in I$, and
%% $i_j'   = \toevent{\canmap{\alpha}{j-1}(x_l)}$ if $i_j   = \toevent{x_l}$, and
%% \item
%%   $\varrho_j = (\canmap{\alpha}{j-1} \cup \mbox{\sl Id}_{\natplus}) \circ \rho_j \circ (\canmap{\alpha}{j})^{-1}$. 
%% \end{itemize}
%% The last expression makes $\varrho_j$ confom to the constraints of utbs
%% (i.e., $\varrho_j(y_l) \in \natplus$ if $l=j$, else $\varrho_j(y_l) = y_l$).
%% To se this, note that
%% for $y_l = \canmap{\alpha}{j}(x_p)$ with $x_p \in \domof{\tvals_j}$,
%% if $l=j$ then $\varrho_j(y_l) = \rho_j \circ x_p \in \natplus$, otherwise,
%% since $\rho_j(x_p) \not\in \natplus$ it follows from
%% $\canmap{\alpha}{j}(x_p) = \canmap{\alpha}{j-1}(\rho_j(x_p))$ that
%% $\varrho_j(y_l) = (\canmap{\alpha}{j-1}) \circ \rho_j \circ (\canmap{\alpha}{j})^{-1}(y_l) = y_l$, i.e.,
%% $\can{\alpha}$ satisfies the constraints for being an untimed behavior.

Since each set $Y_j$ can be obtained as $\domof{\varrho_j}$, we can write
an utb simply as a sequence of labels
$\beta  =  {\hat{i}_1/o_1/\varrho_1}  \cdots {\hat{i}_k/o_k/\varrho_k}$.

We will next provide a characterization of which utbs are feasible, meaning
that the timing constraints induced by the timers can be satisfied in a
corresponding timed run. 
For a utb $\beta  =  {\hat{i}_1/o_1/\varrho_1}  \cdots {\hat{i}_k/o_k/\varrho_k}$,
we introduce the relations $\sdelay{\beta}$ and $\wdelay{\beta}$
on the set of indices $\set{1, \ldots , k}$, defined by
\begin{itemize}
\item
  $l \sdelay{\beta} j$ if $\hat{i}_j = \toevent{y_l}$,
  \item
    $l \wdelay{\beta} j$ if $y_l$ is started but does not timeout, and
    $j$ is the largest index in $\set{1, \ldots , k}$
    such that $y_l \in \domof{\varrho_{j}}$.
\end{itemize}
For $j = 1, \ldots, k$ we introduce $t_j$ to denote the time of occurrence of
the $j$th discrete transition in a timed run, and $t_0 = 0$ to denote
the starting time of a timed run. Thus, for a timed run $\alpha$ as
in Section~\ref{sec:mmt}, $t_j$ would be obtained as
$t_j = d_1 + \cdots + d_j$ for $j = 0 , \ldots, k$.
We associate with $\beta$ a conjunction of difference constraints over
$t_0, \ldots, t_{k}$, denoted $\Constraints{\beta}$, consisting of
%\todobj{We assume that delays are always non-zero. Check how this is said in the paper}
\begin{enumerate}
%% \item
%% $0 < t_1 \leq d_{\max}$,
%% \item
%% for each index $j < k$:  $0 <  t_{j+1} - t_j \leq d_{\max}$,
\item
$t_{j} - t_{j-1} > 0$ for each $j$ in $\set{1, \ldots , k}$,
\item
$t_j - t_l = \varrho_l(y_l)$ whenever $l \sdelay{\beta} j$,
\item
  $t_{j+1} -t_l \leq \varrho_l(y_l)$ whenever $l \wdelay{\beta} j$ for $j < k$, and
  \\
  $t_{k} -t_l \leq \varrho_l(y_l)$ whenever $l \wdelay{\beta} k$.
%% \item
%% for each pair of distinct indices $j$ and $l$ with $i_j, i_l \in I$: $\Frac{t_j} \neq \Frac{t_l}$ 
%% (to express that the fractional parts of $t_j$ and $t_l$ are different).
\end{enumerate}
Constraints of form 1 formalize that any two consecutive transitions must
be separated by a positive time delay,
constraints of form 2 formalize that a timer expires exactly $\varrho_l(y_l)$
time units after being initialized, and
constraints of form 3 formalize that a timer can still be alive (unexpired)
at most $\varrho_l(y_l)$ time units after being initialized.

\begin{definition}
  \label{def:feasible}
  An utb $\beta$ is \emph{feasible} if $\Constraints{\beta}$ is satisfiable.
\end{definition}
We will later show that this definition of feasible precisely captures
the existence of a corresponding timed run.
Let $\fcan{\M}$ denote the set of feasible utbs of form $\can{\mmtpath}$ for some
path $\mmtpath$ of $\M$.
As a last step in the definition of untimed semantics, we will
prune utb's by removing timers that will never expire.
For an utb
$\beta  =  {\hat{i}_1/o_1/\varrho_1}  \cdots {\hat{i}_j/o_j/\varrho_j}$ in $\fcan{\M}$,
and index $l \leq j$,
we say that $y_l$ is \emph{live after $\beta$ in $\M$}
if there is an utb $\beta'$ in $\fcan{\M}$ of form
${\hat{i}_1/o_1/\varrho_1}  \cdots {\hat{i}_j/o_j/\varrho_j} \cdots {\toevent{y_l}/o_k/\varrho_k}$ which extends $\beta$, and in which 
$y_l$ expires in some transition after $\beta$.
For a utb $\beta  =  {\hat{i}_1/o_1/\varrho_1}  \cdots {\hat{i}_k/o_k/\varrho_k}$ of $\M$,
let $\timerlive{\M}{\beta}$ be the utb obtained by removing 
$y_l$ from $\domof{\varrho_j}$ whenever $0 < l \leq j \leq k$ and
$y_l$ is not live after ${\hat{i}_1/o_1/\varrho_1}  \cdots {\hat{i}_j/o_j/\varrho_j}$ in $\M$.
Let
$\timerlivebehs{\M}$ be $\setcomp{\timerlive{\M}{\beta}}{\beta \in \fcan{\M}}$.
We say that $\M \approx_{\mathit{untimed}} \N$ if
$\timerlivebehs{\M} = \timerlivebehs{\N}$.

\subsection{Equivalence Between Timed and Untimed Semantics}

We will now prove the equivalence of the timed and untimed semantics.
We will establish that $\timerlivebehs{\M}$ can
be uniquely derived from $\twsof{\M}$ and vice versa.
This will imply that the untimed semantics coincides with the timed
one, i.e., that
\(
\M \approx_{\mathit{untimed}} \N
\)
iff
%% \quad \mbox{iff} \quad
\(
\M \approx_{\mathit{timed}} \N
\).

We start by introducing a weakening of timed runs, which does not care
whether timers expire or become negative. Define an \emph{extended valuation}
as a partial function $\xtvals : X \hookrightarrow \reals$, which can assign
negative values to timers. 
Extend the delay transition relation to extended valuations by
$\xtvals \xrightarrow{d} \xtvals'$ iff
\[
\domof{\xtvals} = \domof{\xtvals'} \wedge \forall x \in\domof{\xtvals} : \xtvals'(x) = \xtvals(x) - d .
\]
Define an \emph{extended configuration} as a pair $(q,\xtvals)$, where $q \in Q$ is a state and $\xtvals$ is an extended valuation with domain $\varsof{q}$.
Then, for $\xtvals, \xtvals'$ being extended valuations, $i \in \extinputs$, $o \in O$ and $\rho \in X \hookrightarrow (X \cup \natplus)$,
we define a discrete transition relation by: $\xtvals \xrightarrow{i/o/\rho}  \xtvals'$ iff
\begin{eqnarray*}
&& \domof{\rho} = \domof{\xtvals'} ~ \wedge \ranof{\rho} \subseteq \domof{\xtvals} \cup \natplus ~ \wedge\\
  &&\xtvals' = (\xtvals \cup \iota) \circ \rho ~ \wedge
  \\
  && \forall x \in X : i=\toevent{x} \Rightarrow x \not\in \ranof{\rho}.
\end{eqnarray*}
Transition relations $\xrightarrow{d}$ and $\xrightarrow{i/o/\rho}$ can be lifted to extended configurations.
For all extended configurations $(q, \xtvals)$, $(q', \xtvals')$ of MMT $\M$,
\[
\frac{q = q' \quad \xtvals \xrightarrow{d} \xtvals'}{(q,\xtvals) \xrightarrow{d} (q',\xtvals')}
\quad\quad
  \frac{q \xrightarrow{i/o/\rho} q' \quad \xtvals \xrightarrow{i/o/\rho} \xtvals'}{(q,\xtvals) \xrightarrow{i/o} (q',\xtvals')}
\]
A path $\mmtpath$, as above, and vector $\dvec = d_1, \ldots, d_k$ of positive real numbers, uniquely define an extended timed run
$\alpha(\mmtpath,\dvec)$, defined as
  \begin{eqnarray*}
\alpha(\mmtpath,\dvec) & = & (q_0,\xtvals_0) \xrightarrow{d_1} (q_0,\xtvals_0') \xrightarrow{i_1/o_1} (q_1,\xtvals_1) \xrightarrow{d_2} 
%C'_1 \xrightarrow{i_2/o_2} C_2 
\cdots
\\ && \qquad \cdots
\xrightarrow{d_k} (q_{k-1},\xtvals_{k-1}') \xrightarrow{i_k/o_k} (q_k,\xtvals_k)
\end{eqnarray*}
  We note that an extended timed run is also a timed run precisely if
  \begin{inparaenum}[(i)]
  \item no timer is ever assigned a negative value, and
  \item whenever $i_j = \toevent{x_p}$ then $\xtvals_{j-1}'(x_p) = 0$.
  \end{inparaenum}
  The following lemma establishes a useful relation between the delays in
  $\alpha(\mmtpath,\dvec)$
and the valuations of timers in $\can{\mmtpath}$.

\begin{lemma}
  \label{lem:timerval}
  Let $\mmtpath$ be a path, let $\dvec$ be a vector $\dvec = d_1, \ldots, d_k$
  of positive real numbers. 
  Let $\alpha = \alpha(\mmtpath,\dvec)$ and $\beta = \can{\mmtpath}$ be as above.
  Then 
for $l \leq j < k$,
whenever 
$y_l = \canmap{\mmtpath}{j}(x_p)$ with $x_p \in \domof{\xtvals_j}$ 
(implying $y_l \in \domof{\varrho_{j}}$) we have
\begin{eqnarray*}
 \xtvals_{j}'(x_p) & = &  \varrho_l(y_l) - (d_{l+1} + \cdots + d_{j+1})
 \ \ .
\end{eqnarray*}
\end{lemma}

\begin{proof}
We prove the lemma by induction on $j$.
  For the base case $j = l$, we have
 by the definition of $\canmap{\mmtpath}{l}$ that
 $\varrho_l(y_l) = \rho_l(x_p) \in \natplus$. By the definition of extended timed runs,
 $\xtvals_l(x_p) =  \rho_l(x_p) = \varrho_l(y_l)$ and
 $\xtvals_l'(x_p) =  \varrho_l(y_l) - d_{l+1}$.
 For the inductive step where $l < j$, assume that
 $y_l = \canmap{\mmtpath}{j}(x_p)$ with $x_p \in \domof{\xtvals_j}$, and
 as inductive hypothesis that
 $\xtvals_{j-1}'(\invcanmap{\mmtpath}{j-1}(y_l)) =
 \varrho_l(y_l) - (d_{l+1} + \cdots + d_{j})$.
%%  Let $x_p = \invcanmap{\mmtpath}{j}(y_l)$.
 Then $\rho_{j}(x_p) \not\in \natplus$, whence
 $\xtvals_{j}'(x_p) =  \xtvals_{j}(x_p) -d_{j+1} = 
 \xtvals_{j-1}'(\rho_{j}(x_p)) - d_{j+1}$.
 From the definition of $\varrho_{j}$, we infer
 $\rho_{j}(x_p) = \rho_{j}(\invcanmap{\mmtpath}{j}(y_l)) =
 \invcanmap{\mmtpath}{j-1}(\varrho_{j}(y_l)) =
 \invcanmap{\mmtpath}{j-1}(y_l)$. Using the inductive hypothesis, we infer
 $\xtvals_{j}'(x_p) =
 \varrho_l(y_l) - (d_{l+1} + \cdots + d_{j+1})$, which proves the inductive
 step.
\end{proof}

We can now prove the correspondence between $\timerlivebehs{\M}$ and
$\twsof{\M}$.

\begin{theorem}
\label{thm:characterization}
Let $\M$ be an MMT. 
Then $w =  d_1 ~ i_1' ~ o_1 \cdots d_k ~ i_k' ~ o_k$ is in $\twsof{\M}$
if and only if there is a ubt
$\beta = {\hat{i}_1/o_1/\varrho_1}  \cdots {\hat{i}_k/o_k/\varrho_k}$ in $\timerlivebehs{\M}$
such that
\begin{itemize}
\item
  $i'_j   =   \hat{i}_j$ if $\hat{i}_j \in I$, and
  $i'_j   = \mathit{to}$ if $\hat{i}_j \in \toeventsof{Y}$, and
\item
$t_0, \ldots, t_k$ satisfies $\Constraints{\beta}$, where we let
$t_j = d_1 + \cdots + d_j$ for $j = 0 , \ldots, k$.
\end{itemize}
\end{theorem}

\begin{proof}
  We will first consider the direction from timed words to utb's. So, let
  $w =  d_1 ~ i_1' ~ o_1 \cdots d_k ~ i_k' ~ o_k$ be a timed word of $\M$.
  This means that there is a timed run
  \begin{eqnarray*}
\alpha & = & (q_0,\tvals_0) \xrightarrow{d_1} (q_0,\tvals_0') \xrightarrow{i_1/o_1} (q_1,\tvals_1) \xrightarrow{d_2} 
%C'_1 \xrightarrow{i_2/o_2} C_2 
\cdots
\\ && \qquad \cdots
\xrightarrow{d_k} (q_{k-1},\tvals_{k-1}') \xrightarrow{i_k/o_k} (q_k,\tvals_k)
\end{eqnarray*}
  such that $w = \timedword(\alpha)$, which is obtained uniquely as
$\alpha(\mmtpath,\dvec)$, where
  \[
  \mmtpath = q_0 \xrightarrow{i_1/o_1/\rho_1}
  %% q_1  \xrightarrow{i_2/o_2/\rho_2}
\cdots
\xrightarrow{i_k/o_k/\rho_k} q_k
\]
is the path of $\M$ which induces $\alpha$,
and $\dvec = d_1, \ldots d_k$.
Define the utb
$\beta  =  {\hat{i}_1/o_1/\varrho_1}  \cdots {\hat{i}_k/o_k/\varrho_k}$.
by $\beta=\can{\mmtpath}$. It is clear that
$i'_j   =   \hat{i}_j$ if $\hat{i}_j \in I$, and that
$i'_j   = \mathit{to}$ if $\hat{i}_j \in \toeventsof{Y}$. We must now establish
that $\beta$ is feasible. This will be done using Lemma~\ref{lem:timerval}.
By the lemma,
whenever $l \leq j < k$ and $y_l \in \domof{\varrho_{j}}$, we have
$\tvals_{j}'(\invcanmap{\mmtpath}{j}(y_l)) = \varrho_l(y_l) - (t_{j+1} -t_l)$.
 We must prove that $t_0, \ldots t_k$ satisfies the two types of
 constraints in $\Constraints{\beta}$. There are two cases.
 \begin{itemize}
 \item $l \sdelay{\beta} j$. This means that the $j$th input in $\beta$ is
   $\toevent{y_l}$. By the semantics of timed runs, we have
   $\tvals_{j-1}'(\invcanmap{\mmtpath}{j-1}(y_l)) = 0$.
   By Lemma~\ref{lem:timerval}, this means
   $\varrho_l(y_l) = t_j - t_l = 0$, i.e.,
   $t_0, \ldots, t_k$ satisfies the corresponding constraint.
 \item $l \wdelay{\beta} j$. This means that $\invcanmap{\mmtpath}{j}(y_l)$
   is defined, 
   By the semantics of timed runs, we must then have
   \begin{inparaenum}[(i)]
     \item
       for $j < k$ that  $\tvals_{j}'(\invcanmap{\mmtpath}{j}(y_l)) \geq 0$,
       which  by Lemma~\ref{lem:timerval} implies
       $\varrho_l(y_l) \geq t_{j+1} - t_l$, i.e.,
       $t_0, \ldots, t_k$ satisfies the corresponding constraint,
       and
     \item
       for $j = k$ that  $\tvals_{j-1}'(\invcanmap{\mmtpath}{{j-1}}(y_l)) \geq 0$,
       which  by Lemma~\ref{lem:timerval} implies
       $\varrho_l(y_l) \geq t_{j} - t_l$, i.e.,
       $t_0, \ldots, t_k$ satisfies the corresponding constraint.
   \end{inparaenum}
\end{itemize}
 To finalize the proof in this direction, we observe that  $t_0, \ldots, t_k$
 also satisfies
$\Constraints{\timerlive{\M}{\beta}}$, since its conjuncts
are included among those of $\Constraints{\beta}$.

For the other direction,
assume  that $\timerlive{\M}{\beta}\in\timerlivebehs{\M}$ where
$\beta = {\hat{i}_1/o_1/\varrho_1}  \cdots {\hat{i}_k/o_k/\varrho_k}$ is
obtained as $\can{\mmtpath}$,
and that $t_0, \ldots, t_k$ satisfies $\Constraints{\timerlive{\M}{\beta}}$. 
Define 
$d_1, \ldots , d_k$ by $t_j = d_1 + \cdots + d_j$ for $j = 0 , \ldots, k$.
Let $\alpha =\alpha(\mmtpath,\dvec)$ be the extended timed run
\begin{eqnarray*}
\alpha & = & (q_0,\xtvals_0) \xrightarrow{d_1} (q_0,\xtvals_0') \xrightarrow{i_1/o_1} (q_1,\xtvals_1) \xrightarrow{d_2} 
%C'_1 \xrightarrow{i_2/o_2} C_2 
\cdots
\\ && \qquad \cdots
\xrightarrow{d_k} (q_{k-1},\xtvals_{k-1}') \xrightarrow{i_k/o_k} (q_k,\xtvals_k)
\ \ .
\end{eqnarray*}
We will prove that $\alpha$
satisfies the requirements for being a timed run of $\M$, i.e.,
  \begin{inparaenum}[(i)]
  \item no timer is ever assigned a negative value, and
  \item whenever $i_j = \toevent{x_p}$ then $\xtvals_{j-1}'(x_p) = 0$.
  \end{inparaenum}
  This will de done using Lemma~\ref{lem:timerval}. We first consider a
  valuation $\xtvals_{j-1}'(x_p)$ such that the clock
  $y_l = \canmap{\mmtpath}{j-1}(x_p)$ occurs in $\timerlive{\M}{\beta}$,
  i.e., $y_l$ is live after $\beta = {\hat{i}_1/o_1/\varrho_1}  \cdots {\hat{i}_{j-1}/o_{j-1}/\varrho_{j-1}}$. We check the two above constraints:
\begin{itemize}
  \item[(i)] If $i_j = \toevent{x_p}$, we must show
    that $\xtvals_{j-1}'(x_p) = 0$. Then $l \sdelay{\timerlive{\M}{\beta}} j$, and by definition
    definition of     $\Constraints{\timerlive{\M}{\beta}}$ we have
    $t_j - t_l = \varrho_l(y_l)$.
   By Lemma~\ref{lem:timerval}, this means
   $\xtvals_{j-1}'(x_p) = \varrho_l(y_l) - (t_j - t_l) = 0$, which proves
   this case
\item[(ii)] If $x_p \in \domof{\xtvals_{j-1}'}$ we must show
    that $\xtvals_{j-1}'(x_p) \geq 0$. 
Then $l \wdelay{\timerlive{\M}{\beta}} {j-1}$, and by
    definition of     $\Constraints{\timerlive{\M}{\beta}}$ we have
    $t_j - t_l = \varrho_l(y_l)$.
   By Lemma~\ref{lem:timerval}, this means
   $\xtvals_{j-1}'(x_p) = \varrho_l(y_l) - (t_j - t_l) \geq 0$, which proves
   this case
\end{itemize}
   We must finally consider the case when
   $y_l = \canmap{\mmtpath}{j-1}(x_p)$ does not occur in $\timerlive{\M}{\beta}$.
   Assume, to get a contradiction, that $\xtvals_{j-1}'(x_p) < 0$.
   Let $j-1$ be the smallest index where this can occur for any $x_p$. But
   this means that there is a timed run
\[
(q_0,\xtvals_0) \xrightarrow{d_1} (q_0,\xtvals_0') \xrightarrow{i_1/o_1}
\cdots \quad \cdots
\xrightarrow{d_j} (q_{j-1},\xtvals_{j-1}'') \xrightarrow{\toevent{x_p}/o_j'} (q_j',\xtvals_j'')
\]
in which $x_p$ times out at the $j$th transition
(recall that timeout transitions must be enabled for active timers).
This implies that $y_l$ should not have been removed when forming
$\timerlive{\M}{\beta}$.
\end{proof}

Theorem~\ref{thm:characterization} immediately implies
that untimed equivalence is finer than timed equivalence.

\begin{theorem}
\label{untimedimpliestimed}
$\M \approx_{\mathit{untimed}} \N$
implies
$\M \approx_{\mathit{timed}} \N$.
\end{theorem}

In the other direction, the proof is not so immediate, since timed words
hide the identity of timers. To prove the correspondence in the other direction,
we need to find a way to reveal the identity of each expiring timer.
We will show that this can be done for MMTs in which at most one timer is
(re)started on each transition. 
In order prove this result we need to prove a stronger version of
Theorem~\ref{thm:characterization}.

We let $\rtcsdelay{\beta}$ be the reflexive transitive closure of
$\sdelay{\beta}$.

\begin{lemma}
\label{lem:transparent}
  For an utb $\beta$, if $\Constraints{\beta}$ is satisfiable, then
  it has a solution $t_0, \ldots , t_k$ such that
  $t_l$ and $t_j$ have the same fractional part only if
  $l \rtcsdelay{\beta} j$.
\end{lemma}

\begin{proof}
  Consider an utb $\beta$.
The restriction that each transition can start at most one timer, together with
the restriction that the timer update function is injective, implies that
for each $l$ there is at most one $j$ such that
$l \sdelay{\beta} j$ or $l \wdelay{\beta} j$.

By standard DBM techniques, we can check whether $\Constraints{\beta}$ is satisfiable by normalizing it (i.e., closing it under all consequences of
pairs of constraints),
and then checking whether each resulting constraints still
admit a satisfying assigment (i.e., that the lower bound for
a difference does not exceed the upper bound). In fact, for
any specific difference $t_j - t_l$, and any value of
$d$ between its lower and upper bounds, there is a solution
of $\Constraints{\beta}$ such that $t_j-t_l=d$.

For the particular form of constraints of form $\Constraints{\beta}$,
its normalization $\nrml{\Constraints{\beta}}$ satisfies
the following properties, whenever $l \leq j$.
\begin{itemize}
\item
A constraint of form $t_j -t_l = N$ can be included only if
  $l \rtcsdelay{\beta} j$,
\item
A constraint of form $t_j -t_l \leq N$ can be included only if
  $l \rtcswdelay{\beta} j$,
\item
  A constraint of form $t_j -t_l \geq N$ can be included only if there is a $k$
  such that
  $l \rtcsdelay{\beta} k$ and
  $j \rtcswdelay{\beta} k$
\end{itemize}
These properties can be established by proving that they are preserved by
each normalization step, relying on the requirement that
for each $l$ there is at most one $j$ such that
$l \sdelay{\beta} j$ or $l \wdelay{\beta} j$.
Note that $\nrml{\Constraints{\beta}}$ may also include strict
inequialities, but these will not interest us here.
It follows that if $\nrml{\Constraints{\beta}}$ include both
a constraint of form $t_j -t_l \leq N$ and of form $t_j -t_l \geq N$
then $l \rtcsdelay{\beta} j$ (again relying on the requirement that
for each $l$ there is at most one $j$ such that
$l \sdelay{\beta} j$ or $l \wdelay{\beta} j$).

Assume now that $\Constraints{\beta}$ is satisfiable.
%% Assume now that $\beta$ is feasible, and that $t_0, \ldots, t_{k+1}$ is a
%% solution to $\Constraints{\beta}$ is satisfiable.
We shall prove that
there is a solution to  $\Constraints{\beta}$ with the property that
$t_l$ and $t_j$ have the same fractional part only if
$l \rtcsdelay{\beta} j$. We do this by iteratively selecting
values $v_0, \ldots , v_{k+1}$ of $t_0, \ldots , t_{k+1}$ such that this
property is satisfied.
We start by letting $v_0 = 0$ (or some other arbitrary value). Thereafter
we replace $t_o$ by its value ($v_0$) in $\nrml{\Constraints{\beta}}$ and
simplify all constraints that contain $v_0$ so that they become 
constraints of form $t_j \sim N$, where $\sim \in \set{<,\leq,\geq,>}$.
We note that all $N$ is still always an integer in these constraints.
Using the above properties of $\nrml{\Constraints{\beta}}$, it follows that
if we obtain both a constraint of form $t_j \leq N$ and of form $t_j \geq N$
then $0 \rtcsdelay{\beta} j$.
If not $0 \rtcsdelay{\beta} j$,
we can therefore choose a value $v_1$ for $t_1$ with a fractional part differen
from that of $v_0$ (i.e., $0$).

For $i = 1,2,\ldots, k+1$ we  now do the following.
\begin{enumerate}
\item choose a value $v_i$ for $t_i$ with a fractional part different
  from that of any $v_j$ with $j < i$ such that $j \rtcsdelay{\beta} i$
  does not hold.
\item
Replace $t_i$ by its value ($v_i$) in the current constraint, and
simplify all constraints that contain $v_i$ so that they become 
constraints of form $t_j \sim N$, where $\sim \in \set{<,\leq,\geq,>}$.
The resulting constraint has the property that for any $j > i$,
if $t_j \geq N$ and $t_j \leq N'$ are both included (here $N$ and $N'$ can
be real numbers), then $N$ and $N'$ have different fractional parts unless
$l \rtcsdelay{\beta} j$ for some $l$. This implies that for $i+1$,
unless $j \rtcsdelay{\beta} i+1$ for some $j$,
we can choose a value $v_{i+1}$ for $t_{i+1}$ with a fractional part different
  from that of any $v_j$ with $j < i+1$;
 thus we can repeat the procedure for $i+1$, etc.
\end{enumerate}
%% We use the known
%% property of normalized DBMs that for any $j$, we have that
%% $\exists t_{j+1}, \ldots , t_{k+1}\Constraints{\beta}$ is equivalent to the
%% constraint obtained by removing all constraints from $\Constraints{\beta}$ that
%% involve any of $t_{j+1}, \ldots , t_{k+1}$.
%% \todobj{citation}
%% We then inductively select values of $t_0, \ldots , t_{k+1}$ as follows:
%% \begin{itemize}
%% \item
%%   $t_1$ can be chosen arbitrarily.
%% \item Assume that  $t_0, \ldots , t_j$ have been selected to satisfy
%%   $\exists t_{j+1}, \ldots , t_{k+1}\Constraints{\beta}$, and such that
%%   $t_l$ and $t_j$ have the same fractional part only if
%%   $l \rtcsdelay{\beta} j$.
%%   We must then find a $t_{j+1}$ so that
%%   $\exists t_{j+2}, \ldots , t_{k+1}\Constraints{\beta}$ is satisfied.
%%   There are two cases
%%   \begin{itemize}
%%   \item
%%     $l \rtcsdelay{\beta} j+1$ for some $l \leq j$. In this case, there is only
%%     one choice for $t_{j+1}$. This value will maintain the property.
%%   \item
%%     $l \rtcsdelay{\beta} j+1$ for no $l \leq j$. In this case, by the
%%     above property, the constraints for choosing the possible values for
%%     $t_{j+1}$ may not have non-strict upper and lower bounds that are numbers
%%     with the same fractional parts. This implies that we can choose a value
%%     for $t_{j+1}$ with a fractional part different from those of
%%      $t_0, \ldots , t_k$.
%%   \end{itemize}
%% \end{itemize}
This procedure results in an assignment of values such that
  $t_l$ and $t_j$ have the same fractional part only if
  $l \rtcsdelay{\beta} j$, and the lemma is proven.
\end{proof}

Lemma~\ref{lem:transparent}  immediately implies the following
theorem.

%% \begin{theorem}
%%   \label{thm:race-free}
%%   If $\beta$ is a utb in $\timerlivebehs{\M}$, then there is
%%   a race-free and transparent timed run $\alpha$ such that
%%   $\beta = \timerlive{\M}{\can{\alpha}}$.
%% \end{theorem}

\begin{theorem}
  \label{thm:race-free}
  If $\alpha(\mmtpath,\dvec)$ is a timed run of $\M$, then there is
  a race-free and transparent timed run $\alpha(\mmtpath,\dvec')$ derived
  from the same path of $\M$.
\end{theorem}

Theorem~\ref{thm:race-free} also makes it easy to establish the equivalence
between timed and untimed semantics.

\begin{theorem}
\label{timedimpliesuntimed}
$\M \approx_{\mathit{timed}} \N$
implies
$\M \approx_{\mathit{untimed}} \N$.
\end{theorem}

\begin{proof}
  The proof uses analogous reasoning as in the proof of
  Theorem~\ref{thm:characterization}, but using transparent and race-
  free timed runs. For such runs, the identity of timers are uniquely
  determined from a run, which is what is needed for this reversal.
\end{proof}




%% Let $\beta$ and $\beta'$ be two untimed behaviors with the same length and outputs:
%% \begin{eqnarray*}
%% \beta & = & Y_0 \xrightarrow{i_1/o_1/\rho_1} Y_1  \xrightarrow{i_2/o_2/\rho_2} Y_2 \cdots \xrightarrow{i_k/o_k/\rho_k} Y_{k},\\
%% \beta' & = & Y_0 \xrightarrow{i'_1/o_1/\rho'_1} Y_1  \xrightarrow{i'_2/o_2/\rho'_2} Y_2 \cdots \xrightarrow{i'_k/o_k/\rho'_k} Y_{k}.
%% \end{eqnarray*}
%% An \emph{isomorphism} from $\beta$ to $\beta'$ is a list $f = f_0 ,\ldots, f_k$ of bijections $f_j : Y_j \rightarrow Y_j$ such that
%% for all $j>0$: (1) for all $x \in Y_j \setminus \domof{\rho_j})$, $f_j(x)=f_{j-1}(x)$,
%% (2) $i'_j = i_j$ if $i_j \in I$ and $i'_j = \toevent{f_{j-1}(x)}$ if $i_j = \toevent{x}$, for some $x \in Y_{j-1}$, and
%% (3) $\domof{\rho'_j} = f_j(\domof{\rho_j})$ and $\rho'_j(y) = \rho_j ( f_j^{-1}(y))$, for all $y\in\domof{\rho'_j}$.
%% In this case, since $\beta'$ is fully determined by $\beta$ and $f$, we write $\beta' = f(\beta)$.
%% We say that $\beta$ and $\beta'$ are \emph{isomorphic} if there exists an isomorphism $f$ from $\beta$ to $\beta'$.

%% Suppose $\beta$ is an untimed behavior in which transitions update at most one timer.
%% We say that $\beta$ is in \emph{canonical form} if, for each $j$, the timer that is updated in the $j$-th event
%% (if any) is equal to $y_j$.
%% For each untimed behavior $\beta$ in which transitions update at most one timer, there is a unique untimed behavior
%% $\beta'$ in canonical form that is isomorphic to $\beta$.
%% We write $\can{\beta}$ to denote this $\beta'$.







