\section{Untimed Semantics: Alternative Exposition}
\label{section constraints}
In this section, we present an alternative presentation of the untimed
semantics, and of the
equivalence between untimed and timed semantics. The main idea is to concentrate
the exposition by minimizing the number of introduced concepts, and
by formulating the problem of checking feasibility as a problem
of finding a satisfying assignment to a constraint. We then treat the problem
of finding a transparent word in an analogous manner.

Note that in this exposition, we allow timer reassignments from the start.
We also do not need to restrict to timer-live MMTS.


\subsection{Timed Semantics}
\todobj{Replace this by the previous section on timed semantics}

Let us first slightly redefine the notion of timed run. We first redefine the
transition relation between configurations by the rules
\[
\frac{q = q' \quad \tvals \xrightarrow{d} \tvals'}{(q,\tvals) \xrightarrow{d} (q',\tvals')}
\quad\quad
  \frac{q \xrightarrow{i/o/\rho} q' \quad \tvals \xrightarrow{i/o/\rho} \tvals'}{(q,\tvals) \xrightarrow{i/o/\rho} (q',\tvals')}
\]
A \emph{timed run} of $\M$ over $w$ is a sequence 
\begin{eqnarray*}
\alpha & = & (q_0,\tvals_0) \xrightarrow{d_1} (q_0,\tvals_0') \xrightarrow{i_1/o_1/\rho_1} (q_1,\tvals_1) \xrightarrow{d_2} 
%C'_1 \xrightarrow{i_2/o_2} C_2 
\cdots
\\ && \qquad \cdots
\xrightarrow{d_k} (q_{k-1},\tvals_{k-1}') \xrightarrow{i_k/o_k/\rho_k} (q_k,\tvals_k)
\end{eqnarray*}
of transitions between configurations $(q_j,\tvals_j),(q_j,\tvals_j')$ of $\M$, where $(q_0,\tvals_0)$ is the initial configuration.
%Note that, since MMTs are deterministic (if we allow to observe the
%identities of timers in timeout events),
%for each timed word $w$ there exists at most one run over $w$.

A \emph{timed word} over inputs $I$ and outputs $O$ is a sequence
\begin{eqnarray*}
w & = &  d_1 ~ i_1 ~ o_1 ~ d_2 ~ i_2 ~ o_2 \cdots d_k ~ i_k ~ o_k,
\end{eqnarray*}
where $d_j \in \delays$, $i_j \in I \cup \{ \mathit{to} \}$, and $o_j \in O$.
To each timed run $\alpha$ we associate a \emph{timed word} by forgetting the configurations and the timers
in timeout events, and forgetting the timer updates in untimed events:
\begin{eqnarray*}
\timedword(\alpha) & = & d_1 ~ i'_1 ~ o_1 ~ d_2 ~ i'_2 ~ o_2 \cdots d_k ~ i'_k ~ o_k,
\end{eqnarray*}
where for all  $j$,
$i'_j   =   i_j$ if $i_j \in I$, and
$i'_j   = \mathit{to}$ if $i_j \in \toevents$.

\subsection{Untimed Semantics}
We will define an alternative semantics for MMTs, which essentially reflects the
possible feasible sequences of transitions of an MMT, i.e., the sequences of
transitions that give rise to timed runs.
We intend this semantics to be independent of names of particular states and
of particular timers. The sequences in this semantics will not include
names of states, and will employ a canonical way to give names to timers. Thus,
extend the set of timers to include $Y = \{ y_1, y_2,\ldots \}$. Intuitively,
a timer that is started in the $l$th transition is named $y_l$.
Let $\toinputs{Y}$ be $I \cup \toeventsof{Y}$.

Define an \emph{untimed behavior} (\emph{utb} for short)
over inputs $I$ and outputs $O$ to be a sequence 
\begin{eqnarray*}
  \beta & = & Y_0 \xrightarrow{i_1/o_1/\varrho_1} Y_1  \xrightarrow{i_2/o_2/\varrho_1} \cdots \xrightarrow{i_k/o_k/\varrho_k} Y_{k}
  \ \ ,
\end{eqnarray*}
where $Y_0 = \emptyset$ and for $j = 1, \ldots , k$ we have
$Y_j \subseteq \set{y_1, \ldots , y_j}$, $i_j \in \toinputs{Y}$, $o_j \in O$,
and $\varrho_j: Y_j \rightarrow (Y_{j-1} \cup \natplus)$ is an injection, such
that for $y_l \in Y_j$ we have $\varrho_j(y_l) \in \natplus$ if $l=j$, else $\varrho_j(y_l) = y_l$.
Thus, the only timer that may be started in the $j$th transition is $y_j$, and there are no reassignments between timers.

Intuitively, our goal is to define the untimed semantics of an MMT as the utbs
that correspond to sequences of transitions that are {\em feasible}, i.e.,
can give rise to timed runs. Moreoever, timers that are non-live (i.e., that will
never induce a timeout) will be removed from these utbs.
We start by 
defining a \emph{path} $\mmtpath$ of an MMT $\M$ to be a sequence of
transitions of $\M$ of form
\[
\mmtpath = q_0 \xrightarrow{i_1/o_1/\rho_1} q_1  \xrightarrow{i_2/o_2/\rho_2}
\cdots
\xrightarrow{i_k/o_k/\rho_k} q_k
\]
We will replace each state in a path by the corresponding set of timers,
but with canonical names in $Y$. Thus,
if $\mmtpath$ is a path,  define for each $j = 0, \ldots , k$ the injection
%% where $\canmap{\mmtpath}{0}, \ldots, \canmap{\mmtpath}{k}$ is the
%% sequence of injections
$\canmap{\mmtpath}{j}: \vars(q_j) \rightarrow \set{y_1, \ldots ,y_j}$
by
\[
\canmap{\mmtpath}{j}(x_p) = \mbox{if} \ \ \rho_j(x_p) \in \natplus
\ \ \mbox{then} \ \ y_j
\ \ \mbox{else} \ \
\canmap{\mmtpath}{j-1}(\rho_j(x_p)).
\]
Intuitively, $\canmap{\mmtpath}{j}$ renames each timer $x_p$
in $\vars(q_j)$ to the timer $y_l$ whose index $l$ is the same as
the index of the transition in which $x_p$ was started.
For $\mmtpath$ as above, define $\can{\mmtpath}$ as
\begin{eqnarray*}
  \can{\mmtpath} & = & \canmap{\mmtpath}{0}(\vars(q_0)) \xrightarrow{i_1'/o_1/\varrho_1}
%%   \canmap{\mmtpath}{1}(\domof{\tvals_1})  
\cdots \xrightarrow{i_k'/o_k/\varrho_k} \canmap{\mmtpath}{k}(\vars(q_k))
\ ,
\end{eqnarray*}
where
\begin{itemize}
\item
  $i_j'   =  i_j$ if $i_j \in I$,
\item
  $i_j'   = \toevent{\canmap{\mmtpath}{j-1}(x_p)}$ if $i_j   = \toevent{x_p}$, and
\item
  $\varrho_j = (\canmap{\mmtpath}{j-1} \cup \mbox{\sl Id}_{\natplus}) \circ \rho_j \circ (\canmap{\mmtpath}{j})^{-1}$. 
\end{itemize}
We claim that $\can{\mmtpath}$ is an utb. For this, we must prove that
for $y_l \in \canmap{\mmtpath}{j}(\vars(q_j))$ we have $\varrho_j(y_l) \in \natplus$ if $l=j$, else $\varrho_j(y_l) = y_l$.
To se this, assume $y_l = \canmap{\mmtpath}{j}(x_p)$ with $x_p \in \vars(q_j)$,
and note that
\begin{itemize}
\item if $l=j$ then by the definition of $\canmap{\mmtpath}{j}$ we have
  $\rho_j(x_p) \in \natplus$, whence by $x_p = (\canmap{\mmtpath}{j})^{-1}(y_l)$
  we have $\varrho_j(y_l) = \rho_j(\canmap{\mmtpath}{j})^{-1}(y_l) \in \natplus$,
\item
  if $l \neq j$ then by definition of $\canmap{\mmtpath}{j}$ we have
$\rho_j(x_p) \not\in \natplus$; it follows from
$\canmap{\mmtpath}{j}(x_p) = \canmap{\mmtpath}{j-1}(\rho_j(x_p))$ that
  \\
  $\varrho_j(y_l) =
  \left[\canmap{\mmtpath}{j-1} \circ \rho_j \circ (\canmap{\mmtpath}{j})^{-1}\right](y_l) = y_l$.
\end{itemize}
Thus, $\can{\mmtpath}$ satisfies the constraints for being an untimed behavior.
%% If $\alpha$ is a timed run as defined above,  define
%% for each $j = 0, \ldots , k$ the injection
%% %% where $\canmap{\alpha}{0}, \ldots, \canmap{\alpha}{k}$ is the
%% %% sequence of injections
%% $\canmap{\alpha}{j}: \domof{\tvals_j} \rightarrow \set{y_1, \ldots ,y_j}$
%% by
%% \[
%% \canmap{\alpha}{j}(x_p) = \mbox{if} \ \ \rho_j(x_p) \in \natplus
%% \ \ \mbox{then} \ \ y_j
%% \ \ \mbox{else} \ \
%% \canmap{\alpha}{j-1}(\rho_j(x_p)).
%% \]
%% Intuitively, $\canmap{\alpha}{j}$ renames each timer $x_p$
%% in $\domof{\tvals_j}$ to the timer $y_l$ whose index $l$ is the same as
%% the index of the transition in which $x_p$ was started.
%% For $\alpha$ as above, define $\can{\alpha}$ as
%% \begin{eqnarray*}
%%   \can{\alpha} & = & \canmap{\alpha}{0}(\domof{\tvals_0}) \xrightarrow{i_1'/o_1/\varrho_1}
%% %%   \canmap{\alpha}{1}(\domof{\tvals_1})  
%% \cdots \xrightarrow{i_k'/o_k/\varrho_k} \canmap{\alpha}{k}(\domof{\tvals_k})
%% \ ,
%% \end{eqnarray*}
%% where
%% \begin{itemize}
%% \item
%%   $i_j'   =  i_j$ if $i_j \in I$, and
%% $i_j'   = \toevent{\canmap{\alpha}{j-1}(x_l)}$ if $i_j   = \toevent{x_l}$, and
%% \item
%%   $\varrho_j = (\canmap{\alpha}{j-1} \cup \mbox{\sl Id}_{\natplus}) \circ \rho_j \circ (\canmap{\alpha}{j})^{-1}$. 
%% \end{itemize}
%% The last expression makes $\varrho_j$ confom to the constraints of utbs
%% (i.e., $\varrho_j(y_l) \in \natplus$ if $l=j$, else $\varrho_j(y_l) = y_l$).
%% To se this, note that
%% for $y_l = \canmap{\alpha}{j}(x_p)$ with $x_p \in \domof{\tvals_j}$,
%% if $l=j$ then $\varrho_j(y_l) = \rho_j \circ x_p \in \natplus$, otherwise,
%% since $\rho_j(x_p) \not\in \natplus$ it follows from
%% $\canmap{\alpha}{j}(x_p) = \canmap{\alpha}{j-1}(\rho_j(x_p))$ that
%% $\varrho_j(y_l) = (\canmap{\alpha}{j-1}) \circ \rho_j \circ (\canmap{\alpha}{j})^{-1}(y_l) = y_l$, i.e.,
%% $\can{\alpha}$ satisfies the constraints for being an untimed behavior.

Since each set $Y_j$ can be obtained as $\domof{\varrho_j}$, we can write
an utb simply as a sequence of labels
$\beta  =  {i_1/o_1/\varrho_1}  \cdots {i_k/o_k/\varrho_k}$.

We will next provide a characterization of which utbs are feasible, meaning
that the timing constraints induced by the timers can be satisfied in a
corresponding timed run. 
For a utb $\beta  =  {i_1/o_1/\varrho_1}  \cdots {i_k/o_k/\varrho_k}$,
we start by introducing relations $\sdelay{\beta}$ and $\wdelay{\beta}$
on the set of indices $\set{1, \ldots , k}$, defined by
\begin{itemize}
\item
  $l \sdelay{\beta} j$ if $i_j = \toevent{y_l}$,
  \item
    $l \wdelay{\beta} j$ if $y_l$ is started but does not timeout, and
    $j$ is the largest index in $\set{1, \ldots , k}$
    such that $y_l \in \domof{\varrho_{j}}$.
\end{itemize}
Let $t_j = d_1 + \cdots + d_j$ for $j = 0 , \ldots, k$, i.e.,
$t_j$ is the time of occurrence of the $j$th untimed transition in $\alpha$
(assuming $t_0=0$).
We associate with $\beta$ a conjunction of difference constraints over
$t_0, \ldots, t_{k}$, denoted $\Constraints{\beta}$, consisting of
%\todobj{We assume that delays are always non-zero. Check how this is said in the paper}
\begin{enumerate}
%% \item
%% $0 < t_1 \leq d_{\max}$,
%% \item
%% for each index $j < k$:  $0 <  t_{j+1} - t_j \leq d_{\max}$,
\item
$t_{j} - t_{j-1} > 0$ for each $j$ in $\set{1, \ldots , k}$,
\item
$t_j - t_l = \varrho_l(y_l)$ whenever $l \sdelay{\beta} j$,
\item
  $t_{j+1} -t_l \leq \varrho_l(y_l)$ whenever $l \wdelay{\beta} j$ for $j < k$, and
  \\
  $t_{k} -t_l \leq \varrho_l(y_l)$ whenever $l \wdelay{\beta} k$.
%% \item
%% for each pair of distinct indices $j$ and $l$ with $i_j, i_l \in I$: $\Frac{t_j} \neq \Frac{t_l}$ 
%% (to express that the fractional parts of $t_j$ and $t_l$ are different).
\end{enumerate}
Constraints of form 1 formalize that any two consecutive transitions must
be separated by a positive time delay,
constraints of form 2 formalize that a timer expires exactly $\varrho_l(y_l)$
time units after being initialized, and
constraints of form 3 formalize that a timer can still be alive (unexpired)
at most $\varrho_l(y_l)$ time units after being initialized.

\begin{definition}
  \label{def:feasible}
  An utb $\beta$ is \emph{feasible} if $\Constraints{\beta}$ is
  satisfiable
\end{definition}
We will later show that this definition of feasible precisely captures
the existence of a corresponding timed run.
Let $\fcan{\M}$ denote the set of feasible utbs of form $\can{\mmtpath}$ for some
path $\mmtpath$ of $\M$.
As a last step in the definition of untimed semantics, we will
prune utb's by removing timers that will never expire.
For an utb
$\beta  =  {i_1/o_1/\varrho_1}  \cdots {i_j/o_j/\varrho_j}$ in $\fcan{\M}$,
and index $l \leq j$,
we say that $y_l$ is \emph{live after $\beta$ in $\M$}
if there is an utb $\beta'$ in $\fcan{\M}$ of form
${i_1/o_1/\varrho_1}  \cdots {i_j/o_j/\varrho_j} \cdots {\toevent{y_l}/o_k/\varrho_k}$ which extends $\beta$, and in which 
$y_l$ expires in some transition after $\beta$.
For a utb $\beta  =  {i_1/o_1/\varrho_1}  \cdots {i_k/o_k/\varrho_k}$ of $\M$,
let $\timerlive{\M}{\beta}$ be the utb obtained by removing 
$y_l$ from $\domof{\varrho_j}$ whenever $0 < l \leq j \leq k$ and
$y_l$ is not live after ${i_1/o_1/\varrho_1}  \cdots {i_j/o_j/\varrho_j}$ in $\M$.
Let
$\timerlivebehs{\M}$ be $\setcomp{\timerlive{\M}{\beta}}{\beta \in \fcan{\M}}$.
We say that $\M \approx_{\mathit{untimed}} \N$ if
$\timerlivebehs{\M} = \timerlivebehs{\N}$.

\subsection{Equivalence Between Timed and Untimed Semantics}

We can now prove the equivalence of the timed and untimed semantics.
Our goal is to prove that 
\[
\M \approx_{\mathit{untimed}} \N \quad \mbox{iff} \quad \M \approx_{\mathit{timed}} \N
\]
We will establish this result by showing that $\timerlivebehs{\M}$ can
be uniquely derived from the timed words of $\M$ and vice versa.
We start with the following theorem

\begin{theorem}
Let $\M$ be an MMT. 
Then $w =  d_1 ~ i_1' ~ o_1 \cdots d_k ~ i_k' ~ o_k$ be a timed word of $\M$
if and only if
there is a ubt
$\beta = {i_1/o_1/\varrho_1}  \cdots {i_k/o_k/\varrho_k}$ in $\timerlivebehs{\M}$
such that
\begin{itemize}
\item
  $i'_j   =   i_j$ if $i_j \in I$, and
  $i'_j   = \mathit{to}$ if $i_j \in \toeventsof{Y}$, and
\item
$t_0, \ldots, t_k$ satisfies $\Constraints{\beta}$, where we let
$t_j = d_1 + \cdots + d_j$ for $j = 0 , \ldots, k$.
\end{itemize}
\end{theorem}

\begin{proof}
  We will first consider the direction from timed words to utb's. So, let
  $w =  d_1 ~ i_1' ~ o_1 \cdots d_k ~ i_k' ~ o_k$ be a timed word of $\M$.
  This means that there is a timed run
  \begin{eqnarray*}
\alpha & = & (q_0,\tvals_0) \xrightarrow{d_1} (q_0,\tvals_0') \xrightarrow{i_1/o_1/\rho_1} (q_1,\tvals_1) \xrightarrow{d_2} 
%C'_1 \xrightarrow{i_2/o_2} C_2 
\cdots
\\ && \qquad \cdots
\xrightarrow{d_k} (q_{k-1},\tvals_{k-1}') \xrightarrow{i_k/o_k/\rho_k} (q_k,\tvals_k)
\end{eqnarray*}
  such that $w = \timedword(\alpha)$, which is derived from a path $\mmtpath$
  \[
\mmtpath = q_0 \xrightarrow{i_1/o_1/\rho_1} q_1  \xrightarrow{i_2/o_2/\rho_2}
\cdots
\xrightarrow{i_k/o_k/\rho_k} q_k
\]
of $\M$. Define the utb $\beta$ by $\beta=\can{\tau}$. It is clear that
$i'_j   =   i_j$ if $i_j \in I$, and that
$i'_j   = \mathit{to}$ if $i_j \in \toeventsof{Y}$. Thus, we only need
to establish that $\beta$ is feasible. 

The following lemma establishes a useful relation between the delays in $\alpha$
and the valuations of timers in $\can{\tau}$.

\begin{lemma}
  \label{lem:timerval}
Let $\beta = \can{\alpha}$ be as above; then 
%% i.e., be a timer that is started in the $l$th transition, i.e.,
%% $\rho_l(x_p) \in \natplus$.
for $l \leq j < k$,
whenever 
$y_l = \canmap{\alpha}{j}(x_p)$ with $x_p \in \domof{\tvals_j}$, 
(i.e., $y_l \in \domof{\varrho_{j}}$) we have
\begin{eqnarray*}
 \tvals_{j}'(x_p) & = &  \varrho_l(y_l) - (d_{l+1} + \cdots + d_{j+1})
 \ \ .
\end{eqnarray*}
\end{lemma}

\begin{proof}
We prove the lemma by induction on $j$.
  For the base case $j = l$, we have
 by the definition of $\canmap{\alpha}{l}$ that
 $\varrho_l(y_l) = \rho_l(x_p) \in \natplus$. By the semantics of timed runs,
 $\tvals_l(x_p) =  \rho_l(x_p) = \varrho_l(y_l)$ and
 $\tvals_l'(x_p) =  \varrho_l(y_l) - d_{l+1}$.
 For the inductive step where $l < j$, assume that
 $y_l = \canmap{\alpha}{j}(x_p)$ with $x_p \in \domof{\tvals_j}$, and
 as inductive hypothesis that
 $\tvals_{j-1}'(\invcanmap{\alpha}{j-1}(y_l)) =
 \varrho_l(y_l) - (d_{l+1} + \cdots + d_{j})$.
%%  Let $x_p = \invcanmap{\alpha}{j}(y_l)$.
 Then $\rho_{j}(x_p) \not\in \natplus$, whence
 $\tvals_{j}'(x_p) =  \tvals_{j}(x_p) -d_{j+1} = 
 \tvals_{j-1}'(\rho_{j}(x_p)) - d_{j+1}$.
 From the definition of $\varrho_{j}$, we infer
 $\rho_{j}(x_p) = \rho_{j}(\invcanmap{\alpha}{j}(y_l)) =
 \invcanmap{\alpha}{j-1}(\varrho_{j}(y_l)) =
 \invcanmap{\alpha}{j-1}(y_l)$. Using the inductive hypothesis, we infer
 $\tvals_{j}'(x_p) =
 \varrho_l(y_l) - (d_{l+1} + \cdots + d_{j+1})$, which proves the inductive
 step.
\end{proof}

Using Lemma~\ref{lem:timerval}, we can now complete the first direction in the proof.
By the lemma
whenever $l \leq j < k$ and $y_l \in \domof{\varrho_{j}}$, we have
$\tvals_{j}'(\invcanmap{\alpha}{j}(y_l)) = \varrho_l(y_l) - (t_{j+1} -t_l)$.
 We must prove that $t_0, \ldots t_k$ satisfies the two types of
 constraints in $\Constraints{\beta}$. There are two cases.
 \begin{itemize}
 \item $l \sdelay{\beta} j$. This means that the $j$th input in $\beta$ is
   $\toevent{y_l}$. By the semantics of timed runs, we have
   $\tvals_{j-1}'(\invcanmap{\alpha}{j-1}(y_l)) = 0$.
   By Lemma~\ref{lem:timerval}, this means
   $\varrho_l(y_l) = t_j - t_l$, i.e.,
   $t_0, \ldots, t_k$ satisfies the corresponding constraint.
 \item $l \wdelay{\beta} j$. This means that $\invcanmap{\alpha}{j}(y_l)$
   is defined, 
   By the semantics of timed runs, we must then have
   \begin{inparaenum}[(i)]
     \item
       for $j < k$ that  $\tvals_{j}'(\invcanmap{\alpha}{j}(y_l)) \geq 0$,
       which  by Lemma~\ref{lem:timerval} implies
       $\varrho_l(y_l) \geq t_{j+1} - t_l$, i.e.,
       $t_0, \ldots, t_k$ satisfies the corresponding constraint,
       and
     \item
       for $j = k$ that  $\tvals_{j-1}'(\invcanmap{\alpha}{{j-1}}(y_l)) \geq 0$,
       which  by Lemma~\ref{lem:timerval} implies
       $\varrho_l(y_l) \geq t_{j} - t_l$, i.e.,
       $t_0, \ldots, t_k$ satisfies the corresponding constraint.
   \end{inparaenum}
\end{itemize}
 To finalize the proof in this direction, we observe that  $t_0, \ldots, t_k$
 also satisfies
$\Constraints{\timerlive{\M}{\beta}}$, since its conjuncts
are include amont those of $\Constraints{\beta}$.

For the other direction,
assume  $\beta = {i_1/o_1/\varrho_1}  \cdots {i_k/o_k/\varrho_k}$ 
   in $\timerlivebehs{\M}$,
and that
$t_0, \ldots, t_k$ satisfies $\Constraints{\beta}$. 
We will then prove that
\begin{eqnarray*}
\alpha & = & (q_0,\tvals_0) \xrightarrow{d_1} (q_0,\tvals_0') \xrightarrow{i_1/o_1/\rho_1} (q_1,\tvals_1) \xrightarrow{d_2} 
%C'_1 \xrightarrow{i_2/o_2} C_2 
\cdots
\\ && \qquad \cdots
\xrightarrow{d_k} (q_{k-1},\tvals_{k-1}') \xrightarrow{i_k/o_k/\rho_k} (q_k,\tvals_k)
\end{eqnarray*}
satisfies the requirements for being a timed run of $\M$. This is done
using by Lemma~\ref{lem:timerval}, by the same arguments
as the first direction of the proof. We must only check that timers that
are not in $\invcanmap{\alpha}{j}(\domof{\varrho_j})$ for some $j$ do not
interfere. This can happen only if such a timer blocks a delay transition,
by being smaller than any clock of form
$\invcanmap{\alpha}{l}(\domof{\varrho_l})$.
But this implies that there is also a timed run where such a timer expires,
which implies that it should not have been removed when forming
$\timerlivebehs{\M}$.
\end{proof}

Theorem~\ref{thm:characterization} immediately implies
that untimed equivalence is finer than timed equivalence.

\begin{theorem}
\label{untimedimpliestimed}
$\M \approx_{\mathit{untimed}} \N$
implies
$\M \approx_{\mathit{timed}} \N$.
\end{theorem}

In the other direction, the proof is not so immediate, since timed words
hide the identity of timers. To prove the correspondence in the other direction,
we need to find a way to reveal the identity of each expiring timer.
We will show that this can be done for MMTs in which at most one timer is
(re)started on each transition. 
In order prove this result we need to prove a stronger version of
Theorem~\ref{thm:characterization}.

We let $\rtcsdelay{\beta}$ be the reflexive transitive closure of
$\sdelay{\beta}$.

\begin{lemma}
\label{lem:transparent}
  For an utb $\beta$, if $\Constraints{\beta}$ is satisfiable, then
  it has a solution $t_0, \ldots , t_k$ such that
  $t_l$ and $t_j$ have the same fractional part only if
  $l \rtcsdelay{\beta} j$.
\end{lemma}

\begin{proof}
  Consider an utb $\beta$.
The restriction that each transition can start at most one timer, together with
the restriction that the timer update function is injective, implies that
for each $l$ there is at most one $j$ such that
$l \sdelay{\beta} j$ or $l \wdelay{\beta} j$.

By standard DBM techniques, we can check whether $\Constraints{\beta}$ is satisfiable by saturating it (i.e., closing it under implication),
and checking whether the induced constraint for 
$t_{j+1} - t_j$ allow it to be positive for each $j$.
When $\Constraints{\beta}$ is saturated, the following restrictions
will be maintained, where $l \leq j$.
\begin{itemize}
\item
A constraint of form $t_j -t_l = N$ can be included only if
  $l \rtcsdelay{\beta} j$,
\item
A constraint of form $t_j -t_l \leq N$ can be included only if
  $l \rtcswdelay{\beta} j$,
\item
  A constraint of form $t_j -t_l \geq N$ can be included only if there is a $k$
  such that
  $l \rtcsdelay{\beta} k$ and
  $j \rtcswdelay{\beta} k$
\end{itemize}
It follows from the above that if the constraints include both
a constraint of form $t_j -t_l \leq N$ and of form $t_j -t_l \geq N$
then $l \rtcsdelay{\beta} j$.

Assume now that $\Constraints{\beta}$ is satisfiable.
%% Assume now that $\beta$ is feasible, and that $t_0, \ldots, t_{k+1}$ is a
%% solution to $\Constraints{\beta}$ is satisfiable.
We shall now prove that
there is a solution to  $\Constraints{\beta}$ with the property that
$t_l$ and $t_j$ have the same fractional part only if
$l \rtcsdelay{\beta} j$. We can do this by iteratively selecting
values of $t_0, \ldots , t_{k+1}$ with this property. We use the known
property of saturated DBMs that for any $j$, we have that
$\exists t_{j+1}, \ldots , t_{k+1}\Constraints{\beta}$ is equivalent to the
constraint obtained by removing all constraints from $\Constraints{\beta}$ that
involve any of $t_{j+1}, \ldots , t_{k+1}$.
\todobj{citation}
We then inductively select values of $t_0, \ldots , t_{k+1}$ as follows:
\begin{itemize}
\item
  $t_1$ can be chosen arbitrarily.
\item Assume that  $t_0, \ldots , t_j$ have been selected to satisfy
  $\exists t_{j+1}, \ldots , t_{k+1}\Constraints{\beta}$, and such that
  $t_l$ and $t_j$ have the same fractional part only if
  $l \rtcsdelay{\beta} j$.
  We must then find a $t_{j+1}$ so that
  $\exists t_{j+2}, \ldots , t_{k+1}\Constraints{\beta}$ is satisfied.
  There are two cases
  \begin{itemize}
  \item
    $l \rtcsdelay{\beta} j+1$ for some $l \leq j$. In this case, there is only
    one choice for $t_{j+1}$. This value will maintain the property.
  \item
    $l \rtcsdelay{\beta} j+1$ for no $l \leq j$. In this case, by the
    above property, the constraints for choosing the possible values for
    $t_{j+1}$ may not have non-strict upper and lower bounds that are numbers
    with the same fractional parts. This implies that we can choose a value
    for $t_{j+1}$ with a fractional part different from those of
     $t_0, \ldots , t_k$.
  \end{itemize}
\end{itemize}
This procedure results in an assignment of values such that
  $t_l$ and $t_j$ have the same fractional part only if
  $l \rtcsdelay{\beta} j$, and the lemma is proven.
\end{proof}

Lemma~\ref{lem:transparent}  immediately implies the following
theorem.

%% \begin{theorem}
%%   \label{thm:race-free}
%%   If $\beta$ is a utb in $\timerlivebehs{\M}$, then there is
%%   a race-free and transparent timed run $\alpha$ such that
%%   $\beta = \timerlive{\M}{\can{\alpha}}$.
%% \end{theorem}

\begin{theorem}
  \label{thm:race-free}
  If $\alpha$ is a time run of $\M$, then there is
  a race-free and transparent timed run $\alpha'$ of $\M$ such that
  $\timerlive{\M}{\can{\alpha}} = \timerlive{\M}{\can{\alpha'}}$.
\end{theorem}

Theorem~\ref{thm:race-free} also makes it easy to establish the equivalence
between timed and untimed semantics.

\begin{theorem}
\label{timedimpliesuntimed}
$\M \approx_{\mathit{timed}} \N$
implies
$\M \approx_{\mathit{untimed}} \N$.
\end{theorem}

\begin{proof}
  The proof uses analogous reasoning as in the proof of
  Theorem~\ref{thm:characterization}, but using transparent and race-
  free timed runs. For such runs, the identity of timers are uniquely
  determined from a run, which is what is needed for this reversal.
\end{proof}




%% Let $\beta$ and $\beta'$ be two untimed behaviors with the same length and outputs:
%% \begin{eqnarray*}
%% \beta & = & Y_0 \xrightarrow{i_1/o_1/\rho_1} Y_1  \xrightarrow{i_2/o_2/\rho_2} Y_2 \cdots \xrightarrow{i_k/o_k/\rho_k} Y_{k},\\
%% \beta' & = & Y_0 \xrightarrow{i'_1/o_1/\rho'_1} Y_1  \xrightarrow{i'_2/o_2/\rho'_2} Y_2 \cdots \xrightarrow{i'_k/o_k/\rho'_k} Y_{k}.
%% \end{eqnarray*}
%% An \emph{isomorphism} from $\beta$ to $\beta'$ is a list $f = f_0 ,\ldots, f_k$ of bijections $f_j : Y_j \rightarrow Y_j$ such that
%% for all $j>0$: (1) for all $x \in Y_j \setminus \domof{\rho_j})$, $f_j(x)=f_{j-1}(x)$,
%% (2) $i'_j = i_j$ if $i_j \in I$ and $i'_j = \toevent{f_{j-1}(x)}$ if $i_j = \toevent{x}$, for some $x \in Y_{j-1}$, and
%% (3) $\domof{\rho'_j} = f_j(\domof{\rho_j})$ and $\rho'_j(y) = \rho_j ( f_j^{-1}(y))$, for all $y\in\domof{\rho'_j}$.
%% In this case, since $\beta'$ is fully determined by $\beta$ and $f$, we write $\beta' = f(\beta)$.
%% We say that $\beta$ and $\beta'$ are \emph{isomorphic} if there exists an isomorphism $f$ from $\beta$ to $\beta'$.

%% Suppose $\beta$ is an untimed behavior in which transitions update at most one timer.
%% We say that $\beta$ is in \emph{canonical form} if, for each $j$, the timer that is updated in the $j$-th event
%% (if any) is equal to $y_j$.
%% For each untimed behavior $\beta$ in which transitions update at most one timer, there is a unique untimed behavior
%% $\beta'$ in canonical form that is isomorphic to $\beta$.
%% We write $\can{\beta}$ to denote this $\beta'$.







