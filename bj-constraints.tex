\section{Untimed Semantics: Alternative Exposition}
\label{section constraints}
In this section, we present an alternative presentation of the untimed
semantics, and of the
equivalence between untimed and timed semantics. The main idea is to concentrate
the exposition by minimizing the number of introduced concepts, and
by formulating the problem of checking feasibility as a problem
of finding a satisfying assignment to a constraint. We then treat the problem
of finding a transparent word in an analogous manner.

Note that in this exposition, we allow timer reassignments from the start.
We also do not need to restrict to timer-live MMTS.

Let us first slightly redefine the notion of timed run.
A \emph{timed run} of $\M$ over $w$ is a sequence 
\begin{eqnarray*}
\alpha & = & (q_0,\tvals_0) \xrightarrow{d_1} (q_0,\tvals_0') \xrightarrow{i_1/o_1/\rho_1} (q_1,\tvals_1) \xrightarrow{d_2} 
%C'_1 \xrightarrow{i_2/o_2} C_2 
\cdots
\\ && \qquad \cdots
\xrightarrow{d_k} (q_{k-1},\tvals_{k-1}') \xrightarrow{i_k/o_k/\rho_k} (q_k,\tvals_k)
\end{eqnarray*}
of transitions between configurations $(q_j,\tvals_j),(q_j,\tvals_j')$ of $\M$, where $(q_0,\tvals_0)$ is the initial configuration.
%Note that, since MMTs are deterministic (if we allow to observe the
%identities of timers in timeout events),
%for each timed word $w$ there exists at most one run over $w$.
A \emph{timed word} over inputs $I$ and outputs $O$ is a sequence
\begin{eqnarray*}
w & = &  d_1 ~ i_1 ~ o_1 ~ d_2 ~ i_2 ~ o_2 \cdots d_k ~ i_k ~ o_k,
\end{eqnarray*}
where $d_j \in \delays$, $i_j \in I \cup \{ \mathit{to} \}$, and $o_j \in O$.
To each timed run $\alpha$ we associate a \emph{timed word} by forgetting the configurations and the timers
in timeout events:
\begin{eqnarray*}
\timedword(\alpha) & = & d_1 ~ i'_1 ~ o_1 ~ d_2 ~ i'_2 ~ o_2 \cdots d_k ~ i'_k ~ o_k,
\end{eqnarray*}
where for all  $j$,
$i'_j   =   i_j$ if $i_j \in I$, and
$i'_j   = \mathit{to}$ if $i_j \in \toevents$.

\subsection{Untimed Semantics}
We will define an alternative semantics for MMTs, which essentially reflects the
possible feasible sequences of transitions of an MMT, i.e., the sequences of
transitions that give rise to timed runs.
Mowever, we do not want this semantics to
depend on the identity of the timers. Therefore, we introduce a canonical way to
give names to timers in such runs. To avoid confusion between timers of MMTs
and timers in the untimed semantics, we will extend the set of timers
to include $Y = \{ y_1, y_2,\ldots \}$
\tofix{Extend $\extinputs$}

Define an \emph{untimed behavior} (\emph{utb} for short)
over inputs $I$ and outputs $O$ to be a sequence 
\begin{eqnarray*}
  \beta & = & Y_0 \xrightarrow{i_1/o_1/\varrho_1} Y_1  \cdots \xrightarrow{i_k/o_k/\varrho_k} Y_{k}
  \ \ ,
\end{eqnarray*}
Where $Y_0 = \emptyset$ and for each $j>0$, $Y_j \subseteq \set{y_1, \ldots , y_j}$  $i_j \in \extinputs$, $o_j \in O$,
and for $y_l \in Y_j$ we have $\varrho_j(y_l) \in \natplus$ if $l=j$, else $\varrho_j(y_l) = y_l$.
I.e., the only timer that may be started in the $j$th transition is $y_j$, and there are no
reassignments between timers.
For a timed run $\alpha$ as defined above,  define
$\can{\alpha}$ as
\begin{eqnarray*}
  \can{\alpha} & = & \canmap{\alpha}{0}(\domof{\tvals_0}) \xrightarrow{i_1'/o_1/\varrho_1}
%%   \canmap{\alpha}{1}(\domof{\tvals_1})  
\cdots \xrightarrow{i_k'/o_k/\varrho_k} \canmap{\alpha}{k}(\domof{\tvals_k})
  \ \ ,
\end{eqnarray*}
where for $j = 0, \ldots , k$, the injection
%% where $\canmap{\alpha}{0}, \ldots, \canmap{\alpha}{k}$ is the
%% sequence of injections
$\canmap{\alpha}{j}: \domof{\tvals_j} \rightarrow \set{y_1, \ldots ,y_j}$
is defined by
\[
\canmap{\alpha}{j}(x_l) = \mbox{if} \ \ \rho_j(x_l) \in \natplus
\ \ \mbox{then} \ \ y_j
\ \ \mbox{else} \ \
\canmap{\alpha}{j-1}(\rho_j(x_l))
\]
and where
\begin{itemize}
\item
  $i_j'   =   i_j$ if $i_j \in I$, and
$i_j'   = \toevent{\canmap{\alpha}{j-1}(x_l)}$ if $i_j   = \toevent{x_l}$, and
\item
  $\varrho_j = (\canmap{\alpha}{j-1} \cup \mbox{\sl Id}_{\natplus}) \circ \rho_j \circ (\canmap{\alpha}{j})^{-1}$.
\end{itemize}
%% It follows that whenever $x_l \in \canmap{\alpha}{j}(Y_j)$ then
%% $\rho_j'(x_l)\in \natplus$ if $l=j$ else $\rho_j'(x_l) = x_l$.
For $\rho_j(x_l) \not\in \natplus$ if follows from
$\canmap{\alpha}{j}(x_l) = \canmap{\alpha}{j-1}(\rho_j(x_l))$ that
$\varrho_j(x_l) = (\canmap{\alpha}{j-1}) \circ \rho_j \circ (\canmap{\alpha}{j})^{-1}(x_l) = x_l$, i.e.,
$\can{\alpha}$ satisfies the constraints for being an untimed behavior.


Since each set $Y_j$ can be obtained as $\domof{\varrho_j}$, we can write
an utb simply as a sequence of labels
$\beta  =  {i_1/o_1/\varrho_1}  \cdots {i_k/o_k/\varrho_k}$.

Two utbs may still have the same observable behavior, but
differ in timers that are started but never time out. We will introduce two
ways to consider this.
For a sequence $i_1/o_1 \ldots i_j/o_j$ of input-output pairs and
$l \leq j$, say that $y_l$ is \emph{live after $i_1/o_1 \ldots i_j/o_j$} in $\M$
if there is an utb of $\M$ of form
${i_1/o_1/\varrho_1}  \cdots {i_j/o_j/\varrho_j} \cdots {\toevent{y_l}/o_k/\varrho_k}$
with $j < k$,  i.e., which starts with the sequence
$i_1/o_1 \ldots i_j/o_j$, and in which $y_l$ expires at some transition after
this sequence.

For a utb $\beta  =  {i_1/o_1/\varrho_1}  \cdots {i_k/o_k/\varrho_k}$ of $\M$,
let $\timerlive{\M}{\beta}$ be the utb obtained by removing 
$y_l$ from $\domof{\varrho_j}$ whenever $0 < l \leq j \leq k$ and
$y_l$ is not live after $i_1/o_1 \ldots i_j/o_j$.
Let
$\timerlivebehs{\M}$ be $\setcomp{\timerlive{\M}{\beta}}{\beta \in \can{\M}}$.
We say that $\M \approx_{\mathit{untimed}} \N$ if
$\timerlivebehs{\M} = \timerlivebehs{\N}$.

For a utb $\beta  =  {i_1/o_1/\varrho_1}  \cdots {i_k/o_k/\varrho_k}$,
define the relations $\sdelay{\beta}$ and $\wdelay{\beta}$ by
\begin{itemize}
\item
  $l \sdelay{\beta} j$ if $i_j = \toevent{y_l}$,
  \item
    $l \wdelay{\beta} j$ if $y_l$ is started but does not timeout, and
$j$ is the largest index such that $y_l \in \domof{\varrho_{j-1}}$.
\end{itemize}
Let $t_j = d_1 + \cdots + d_j$ for $j = 1 , \ldots, k+1$.
We associate with $\beta$ a conjunction of difference constraints over
$t_1, \ldots, t_{k+1}$, denoted $\Constraints{\beta}$, consisting of
%\todobj{We assume that delays are always non-zero. Check how this is said in the paper}
\begin{enumerate}
%% \item
%% $0 < t_1 \leq d_{\max}$,
%% \item
%% for each index $j < k$:  $0 <  t_{j+1} - t_j \leq d_{\max}$,
\item
for each index $j$ with $1 \leq j \leq k$:  $0 <  t_{j+1} - t_j$,
\item
$t_j - t_l = \varrho_l(y_l)$ whenever $l \sdelay{\beta} j$,
\item
$t_j -t_l \leq \varrho_l(y_l)$ whenever $l \wdelay{\beta} j$.
%% \item
%% for each pair of distinct indices $j$ and $l$ with $i_j, i_l \in I$: $\Frac{t_j} \neq \Frac{t_l}$ 
%% (to express that the fractional parts of $t_j$ and $t_l$ are different).
\end{enumerate}
Constraints of form 1 formalize that each delay $d_j$ in a timed word is
positive,
constraints of form 2 formalize that a timer expires exactly $\varrho_l(y_l)$
time units after being initialize, and
constraints of form 3 formalize that a timer must have expired if it has
not been stopped or expired after  more then $\varrho_l(y_l)$ time units.

For a timer $y_l \in \domof{\varrho_k}$,
we say that $y_l$ is \emph{expirable} after $\beta$ if there is a
solution $t_1, \ldots , t_{k+1}$ of  $\Constraints{\beta}$ such that
$\varrho_l(y_l) - (t_k - t_l)$ is minimal among all
$\varrho_j(y_j) - (t_k - t_j)$ with $y_j \in \domof{\varrho_k}$.
To see that this formalizes the intuition that $y_l$ may be the next
timer to expire, note that
$\tvals_k(y_j) = \varrho_j(y_j) - (t_k - t_j)$ for each $y_j \in \domof{\varrho_k}$.
We formalize this as the following lemma.

\begin{lemma}
\label{expirable}
Let $\beta  =  {i_1/o_1/\varrho_1}  \cdots {i_k/o_k/\varrho_k} \in \timerlivebehs{\M}$.
Then $y_l$ is expirable after $\beta$ iff $\beta \toevent{y_l}/o/\rho \in \timerlivebehs{\M}$ for some $o$ and $\rho$.
\end{lemma}

We are now in the position to characterize the set of
timed words of $\M$ in terms of the set $\timerlivebehs{\M}$

\begin{theorem}
\label{thm:characterization}
$w =  d_1 ~ i_1' ~ o_1 ~ d_2 ~ i_2' ~ o_2 \cdots d_k ~ i_k' ~ o_k$
  is a timed word of $\M$ if and only if there is 
  a utb 
  $\beta = {i_1/o_1/\varrho_1}  \cdots {i_k/o_k/\varrho_k}$
   in $\timerlivebehs{\M}$,
such that $t_1, \ldots, t_k$ satisfies $\Constraints{\beta}$, and
  such that   for all  $j$,
$i'_j   =   i_j$ if $i_j \in I$, and
$i'_j   = \mathit{to}$ if $i_j \in \toeventsof{Y}$.
\end{theorem}
  
\begin{proof}
First assume that
$w =  d_1 ~ i_1' ~ o_1 ~ d_2 ~ i_2' ~ o_2 \cdots d_k ~ i_k' ~ o_k$
is a timed word of $\M$, i.e., there is a timed run of $\M$
\begin{eqnarray*}
\alpha & = & (q_0,\tvals_0) \xrightarrow{d_1} (q_0,\tvals_0') \xrightarrow{i_1/o_1/\rho_1} (q_1,\tvals_1) \xrightarrow{d_2} 
%C'_1 \xrightarrow{i_2/o_2} C_2 
\cdots
\\ && \qquad \cdots
\xrightarrow{d_k} (q_{k-1},\tvals_{k-1}') \xrightarrow{i_k/o_k/\rho_k} (q_k,\tvals_k)
\end{eqnarray*}
where for all  $j$,
$i'_j   =   i_j$ if $i_j \in I$, and
$i'_j   = \mathit{to}$ if $i_j \in \toevents$.
Let $\beta = \timerlive{\M}{\can{\alpha}}$.
Let $t_j = d_1 + \cdots + d_j$ for $j = 1 , \ldots, k+1$.
We must prove that $t_1, \ldots t_k$ is a solution to $\Constraints{\beta}$.

Let $y_l \in Y$,
%% i.e., be a timer that is started in the $l$th transition, i.e.,
%% $\rho_l(x_p) \in \natplus$.
Using the definition of $\can{\alpha}$, we establish by induction that 
 whenever $l < j \leq k$ and
 $y_l \in \domof{\varrho_{j-1}}$, then
\begin{eqnarray*}
 \tvals_{j-1}'(\invcanmap{\alpha}{j-1}(y_l)) & = &
 \varrho_l(y_l) - (d_{l+1} + \cdots + d_j)
 \ \ .
\end{eqnarray*}
 For the base case $j = l+1$, let $x_p = \invcanmap{\alpha}{l}(y_l)$.
 By the definition of $\canmap{\alpha}{l}$ we have that
 $\rho_l(x_p)  \in \natplus$ and  
 $\varrho_l(y_l) = \rho_l(x_p)$. By the semantics of timed runs,
 $\tvals_l(x_p) =  \rho_l(x_p) = \varrho_l(y_l)$ and
 $\tvals_l'(x_p) =  \varrho_l(y_l) - d_{l+1}$.
 For the inductive step, assume that $l+1 < j$ and 
 $y_l \in \domof{\varrho_{j-1}}$, and
 $\tvals_{j-2}'(\invcanmap{\alpha}{j-2}(y_l)) =
 \varrho_l(y_l) - (d_{l+1} + \cdots + d_{j-1})$.
 Let $x_p = \invcanmap{\alpha}{j-1}(y_l)$.
 Then $\rho_{j-1}(x_p) \not\in \natplus$, whence
 $\tvals_{j-1}'(x_p) =  \tvals_{j-1}(x_p) -d_j = 
 \tvals_{j-2}'(\rho_{j-1}(x_p)) - d_j$.
 From the definition of $\varrho_{j-1}$, we infer
 $\rho_{j-1}(x_p) = \rho_{j-1}(\invcanmap{\alpha}{j-1}(y_l)) =
 \invcanmap{\alpha}{j-2}(\varrho_{j-1}(y_l)) =
 \invcanmap{\alpha}{j-2}(y_2)$. Using the inductive hypothesis, we infer
 $\tvals_{j-1}'(\invcanmap{\alpha}{j-1}(y_l)) =
 \varrho_l(y_l) - (d_{l+1} + \cdots + d_j)$, which proves the inductive
 step.

 Let us now consider the two types of constraints in $\Constraints{\beta}$.
 \begin{itemize}
 \item $l \sdelay{\beta} j$. This means that the $j$th input in $\beta$ is
   $\toevent{y_l}$. By the semantics of timed runs, we must then have
   $\tvals_{j-1}'(\invcanmap{\alpha}{j-1}(y_l)) = 0$ By Property XXX, this means
   $\varrho_l(y_l) = (d_{l+1} + \cdots + d_j) = t_j - t_l$, i.e.,
   $t_1, \ldots, t_k$ satisfies the corresponding constraint.
 \item $l \wdelay{\beta} j$. This means that $\invcanmap{\alpha}{j-1}(y_l)$
   is defined, 
   By the semantics of timed runs, we must then have
   $\tvals_{j-1}'(\invcanmap{\alpha}{j-1}(y_l)) \geq 0$.
   By Property XXX, this means
   $\varrho_l(y_l) \geq (d_{l+1} + \cdots + d_j) = t_j - t_l$, i.e.,
   $t_1, \ldots, t_k$ satisfies the corresponding constraint.
 \end{itemize}
In the other direction, assume that 
 $\beta = {i_1/o_1/\varrho_1}  \cdots {i_k/o_k/\varrho_k}$
   in $\timerlivebehs{\M}$,
and that
$t_1, \ldots, t_k$ satisfies $\Constraints{\beta}$.
We will then prove that
\begin{eqnarray*}
\alpha & = & (q_0,\tvals_0) \xrightarrow{d_1} (q_0,\tvals_0') \xrightarrow{i_1/o_1/\rho_1} (q_1,\tvals_1) \xrightarrow{d_2} 
%C'_1 \xrightarrow{i_2/o_2} C_2 
\cdots
\\ && \qquad \cdots
\xrightarrow{d_k} (q_{k-1},\tvals_{k-1}') \xrightarrow{i_k/o_k/\rho_k} (q_k,\tvals_k)
\end{eqnarray*}
is a timed run. We must establish that $\alpha$
satisfies the requirements of a timed run. This is done
using property XXX, by the same arguments
as the first direction of the proof. We must only check that timers that
are not in $\invcanmap{\alpha}{j}(\domof{\varrho_j})$ for some $j$ do not
interfere. This can happen only if such a timer blocks a delay transition,
by being smaller than any clock of form
$\invcanmap{\alpha}{l}(\domof{\varrho_l})$.
But this implies that there is also a timed run where such a timer expires,
which implies that it should not have been removed when forming
$\timerlivebehs{\M}$.
\end{proof}

\subsection{Equivalence timed/untimed semantics}
\label{sec:bj-wiggling}

Theorem~\ref{thm:characterization} immediately implies
that untimed equivalence is finer than timed equivalence.

\begin{theorem}
\label{untimedimpliestimed}
$\M \approx_{\mathit{untimed}} \N$
implies
$\M \approx_{\mathit{timed}} \N$.
\end{theorem}

In the other direction, the proof is not so immediate, since timed words
hide the identity of timers. To prove the correspondence in the other direction,
we need to find a way to reveal the identity of each expiring timer.
We will show that this can be done for MMTs in which at most one timer is
(re)started on each transition. 
In order prove this result we need to prove a stronger version of
Theorem~\ref{thm:characterization}.

We let $\rtcsdelay{\beta}$ be the reflexive transitive closure of
$\sdelay{\beta}$.

\begin{lemma}
\label{lem:transparent}
  For an utb $\beta$, if $\Constraints{\beta}$ is satisfiable, then
  it has a solution $t_1, \ldots , t_k$ such that
  $t_l$ and $t_j$ have the same fractional part only if
  $l \rtcsdelay{\beta} j$.
\end{lemma}

\begin{proof}
  Consider an utb $\beta$.
The restriction that each transition can start at most one timer, together with
the restriction that the timer update function is injective, implies that
for each $l$ there is at most one $j$ such that
$l \sdelay{\beta} j$ or $l \wdelay{\beta} j$.

By standard DBM techniques, we can check whether $\Constraints{\beta}$ is satisfiable by saturating it (i.e., closing it under implication),
and checking whether the induced constraint for 
$t_{j+1} - t_j$ allow it to be positive for each $j$.
When $\Constraints{\beta}$ is saturated, the following restrictions
will be maintained, where $l \leq j$.
\begin{itemize}
\item
A constraint of form $t_j -t_l = N$ can be included only if
  $l \rtcsdelay{\beta} j$,
\item
A constraint of form $t_j -t_l \leq N$ can be included only if
  $l \rtcswdelay{\beta} j$,
\item
  A constraint of form $t_j -t_l \geq N$ can be included only if there is a $k$
  such that
  $l \rtcsdelay{\beta} k$ and
  $j \rtcswdelay{\beta} k$
\end{itemize}
It follows from the above that if the constraints include both
a constraint of form $t_j -t_l \leq N$ and of form $t_j -t_l \geq N$
then $l \rtcsdelay{\beta} j$.

Assume now that $\Constraints{\beta}$ is satisfiable.
%% Assume now that $\beta$ is feasible, and that $t_1, \ldots, t_{k+1}$ is a
%% solution to $\Constraints{\beta}$ is satisfiable.
We shall now prove that
there is a solution to  $\Constraints{\beta}$ with the property that
$t_l$ and $t_j$ have the same fractional part only if
$l \rtcsdelay{\beta} j$. We can do this by iteratively selecting
values of $t_1, \ldots , t_{k+1}$ with this property. We use the known
property of saturated DBMs that for any $j$, we have that
$\exists t_{j+1}, \ldots , t_{k+1}\Constraints{\beta}$ is equivalent to the
constraint obtained by removing all constraints from $\Constraints{\beta}$ that
involve any of $t_{j+1}, \ldots , t_{k+1}$.
\todobj{citation}
We then inductively select values of $t_1, \ldots , t_{k+1}$ as follows:
\begin{itemize}
\item
  $t_1$ can be chosen arbitrarily.
\item Assume that  $t_1, \ldots , t_j$ have been selected to satisfy
  $\exists t_{j+1}, \ldots , t_{k+1}\Constraints{\beta}$, and such that
  $t_l$ and $t_j$ have the same fractional part only if
  $l \rtcsdelay{\beta} j$.
  We must then find a $t_{j+1}$ so that
  $\exists t_{j+2}, \ldots , t_{k+1}\Constraints{\beta}$ is satisfied.
  There are two cases
  \begin{itemize}
  \item
    $l \rtcsdelay{\beta} j+1$ for some $l \leq j$. In this case, there is only
    one choice for $t_{j+1}$. This value will maintain the property.
  \item
    $l \rtcsdelay{\beta} j+1$ for no $l \leq j$. In this case, by the
    above property, the constraints for choosing the possible values for
    $t_{j+1}$ may not have non-strict upper and lower bounds that are numbers
    with the same fractional parts. This implies that we can choose a value
    for $t_{j+1}$ with a fractional part different from those of
     $t_1, \ldots , t_k$.
  \end{itemize}
\end{itemize}
This procedure results in an assignment of values such that
  $t_l$ and $t_j$ have the same fractional part only if
  $l \rtcsdelay{\beta} j$, and the lemma is proven.
\end{proof}

Lemma~\ref{lem:transparent}  immediately implies the following
theorem.

%% \begin{theorem}
%%   \label{thm:race-free}
%%   If $\beta$ is a utb in $\timerlivebehs{\M}$, then there is
%%   a race-free and transparent timed run $\alpha$ such that
%%   $\beta = \timerlive{\M}{\can{\alpha}}$.
%% \end{theorem}

\begin{theorem}
  \label{thm:race-free}
  If $\alpha$ is a time run of $\M$, then there is
  a race-free and transparent timed run $\alpha'$ of $\M$ such that
  $\timerlive{\M}{\can{\alpha}} = \timerlive{\M}{\can{\alpha'}}$.
\end{theorem}

Theorem~\ref{thm:race-free} also makes it easy to establish the equivalence
between timed and untimed semantics.

\begin{theorem}
\label{timedimpliesuntimed}
$\M \approx_{\mathit{timed}} \N$
implies
$\M \approx_{\mathit{untimed}} \N$.
\end{theorem}

\begin{proof}
  The proof uses analogous reasoning as in the proof of
  Theorem~\ref{thm:characterization}, but using transparent and race-
  free timed runs. For such runs, the identity of timers are uniquely
  determined from a run, which is what is needed for this reversal.
\end{proof}




%% Let $\beta$ and $\beta'$ be two untimed behaviors with the same length and outputs:
%% \begin{eqnarray*}
%% \beta & = & Y_0 \xrightarrow{i_1/o_1/\rho_1} Y_1  \xrightarrow{i_2/o_2/\rho_2} Y_2 \cdots \xrightarrow{i_k/o_k/\rho_k} Y_{k},\\
%% \beta' & = & Y_0 \xrightarrow{i'_1/o_1/\rho'_1} Y_1  \xrightarrow{i'_2/o_2/\rho'_2} Y_2 \cdots \xrightarrow{i'_k/o_k/\rho'_k} Y_{k}.
%% \end{eqnarray*}
%% An \emph{isomorphism} from $\beta$ to $\beta'$ is a list $f = f_0 ,\ldots, f_k$ of bijections $f_j : Y_j \rightarrow Y_j$ such that
%% for all $j>0$: (1) for all $x \in Y_j \setminus \domof{\rho_j})$, $f_j(x)=f_{j-1}(x)$,
%% (2) $i'_j = i_j$ if $i_j \in I$ and $i'_j = \toevent{f_{j-1}(x)}$ if $i_j = \toevent{x}$, for some $x \in Y_{j-1}$, and
%% (3) $\domof{\rho'_j} = f_j(\domof{\rho_j})$ and $\rho'_j(y) = \rho_j ( f_j^{-1}(y))$, for all $y\in\domof{\rho'_j}$.
%% In this case, since $\beta'$ is fully determined by $\beta$ and $f$, we write $\beta' = f(\beta)$.
%% We say that $\beta$ and $\beta'$ are \emph{isomorphic} if there exists an isomorphism $f$ from $\beta$ to $\beta'$.

%% Suppose $\beta$ is an untimed behavior in which transitions update at most one timer.
%% We say that $\beta$ is in \emph{canonical form} if, for each $j$, the timer that is updated in the $j$-th event
%% (if any) is equal to $y_j$.
%% For each untimed behavior $\beta$ in which transitions update at most one timer, there is a unique untimed behavior
%% $\beta'$ in canonical form that is isomorphic to $\beta$.
%% We write $\can{\beta}$ to denote this $\beta'$.







\subsection{Potentially useful Text Snippets}

\tofix{Do not read this Subsection}

Let $\lean{\beta}$ be the utb obtained by removing 
$y_l$ from all $\beta$ altogether (i.e., from each $\domof{\varrho_j}$)
whenever $y_l$ does not expire in $\beta$.
Let $\can{\M}$ be the set of utbs of $\M$,, and
let 
$\lean{S}$ be $\setcomp{\lean{\beta}}{\beta \in S}$.
For a lean utb $\beta \in S$
\begin{eqnarray*}
\beta & = & X_0 \xrightarrow{i_1/o_1/\rho_1} X_1  \cdots \xrightarrow{i_k/o_k/\rho_k} X_{k},
\end{eqnarray*}
define its \emph{saturation wrp.\ $S$}, denoted $\saturatebeh{\beta}{S}$,
as the utb obtained from $\beta$ by performing the operation
$\Addtimerop{l}{d}{j}$ 
for each $\beta' \in S$ and $l < j \leq k$, such that $\beta'$ has the same
$j$ first input and outputs as $\beta$, which sets $x_l$ to $d$ and does not stop
or expire $x_l$ up to and including the $j$th transition (i.e., $\beta'$ is of
form
\begin{eqnarray*}
\beta' & = & X_0 \xrightarrow{i_1/o_1/\rho_1'} \cdots \xrightarrow{i_j/o_j/\rho_j'} X_{j}' \xrightarrow{i'/o'/\rho'}  \cdots
\end{eqnarray*}
with $\rho_l'(x_l) = d$, and $x_l \in X_j'$).
For a lean timer language $S$ define its \emph{saturation} $\saturateset{S}$
as $\setcomp{\saturatebeh{\beta}{S}}{\beta \in S}$.

Consider a utb $\beta$
\begin{eqnarray*}
\beta & = & Y_0 \xrightarrow{i_1/o_1, \rho_1} X_1  \cdots \xrightarrow{i_k/o_k, \rho_k} Y_{k}.
\end{eqnarray*}
Let $1 \leq l \leq j \leq k$, $d \in \natplus$ and suppose that 
(1) $\rho_l$ either starts no timer or sets $y_l$ to $d$,
(2) $\beta$ does not contain a timeout event $\toevent{y_l}$.
Then $\Addtimer{l}{d}{j}{\beta}$ is the utb obtained from $\beta$ by
letting $\rho_l$ map $y_l$ to $d$, and adding $y_j$ to the sets $Y_l$ up to (and including) $Y_j$.
%In the utb $\emptyset \xrightarrow{i/o,~ x:=1 } \{ x\} \xrightarrow{i/o,~ y:=60 } \{ x, y\}$, for instance, we may remove timer $x$
%and obtain $\emptyset \xrightarrow{i/o } \emptyset \xrightarrow{i/o,~ y:=60 } \{ y\}$.
%Similarly, we may remove the first use of timer $x$ in the utb 
%$\emptyset \xrightarrow{i/o,~ x:=1 } \{ x\} \xrightarrow{i/o,~ x:=5 } \{ x\} \xrightarrow{\toevent{x}/o} \emptyset$ and obtain
%$\emptyset \xrightarrow{i/o } \emptyset \xrightarrow{i/o,~ x:=5 } \{ x\} \xrightarrow{\toevent{x}/o} \emptyset$.
Write $\beta \sqsubseteq \beta'$ if $\beta'$ can be obtained from $\beta$ by zero or more applications of function $\mathsf{addtimer}.$
Then $\sqsubseteq$ is a partial order whose minimal elements are utbs in which every timer that is started also times out.
We call such utbs \emph{lean}.

\todobj{Maybe scrap the following}
We say that $\beta$ is a \emph{partial} utb of MMT $\M$ iff $\M$ has an utb $\beta'$ such that $\beta \sqsubseteq \beta'$.


The set of feasible lean behaviors of an MMT has certain properties, that
we here collect into the definition of a \emph{timer language}.

\begin{definition}
\label{def:timer language}
A (lean) \emph{timer language} over $I$ and $O$ is a nonempty set 
$S$ of (feasible?) lean utbs in canonical form over $I$ and $O$ that satisfies the following five properties:
\begin{itemize}
\item
\emph{no initial timers}: $\beta \in S \Rightarrow \Head{\beta} = \emptyset$,
\item
\emph{prefix closed}: $\beta \cdot \gamma \in S \Rightarrow \beta \in S$,
\item
\emph{behavior deterministic}:
$\beta \xrightarrow{i/o_1/\rho_1} X_1 \in S \wedge \beta \xrightarrow{i/o_2/\rho_2} X_2 \in S \Rightarrow o_1 = o_2 \wedge \rho_1 = \rho_2 \wedge X_1 = X_2$,
%% \item
%% \emph{input complete}:
%% $\beta \in S \wedge i \in I \Rightarrow \exists o, \rho, Y : \beta \xrightarrow{i/o/\rho} Y \in S$,
\item
\emph{timer live}:
$\beta \in S \wedge x \in last(S) \Rightarrow \exists \gamma, o, \rho, Y : \beta \cdot \gamma \xrightarrow{\toevent{x}/o/\rho} Y \in S$,
and
\item
\emph{timeout complete}:
$\beta \in S \wedge x \mbox{ expirable after } \beta \Rightarrow
\exists o, \rho, Y: \beta \xrightarrow{\toevent{x}/o/\rho} Y \in S$.
\end{itemize}
\end{definition}
\todobj{Check again the above definition!!!}

From the saturated version of $\Constraints{\beta}$, we also derive a constraint,
denoted $\Zone{\beta}$,
which characterizes the possible timer valuations $\tvals_k$ in a timed run
$\alpha$ with $\can{\alpha} = \beta$.
The constraint $\Zone{\beta}$
contains for each pair of timers $x_i,x_j$ in $\domof{\rho_k}$ the conjunct
\[
(m + \rho_j(x_j) - \rho_i(x_i)) < \tvals_k(x_j) -\tvals_k(x_i) < (n + \rho_j(x_j) - \rho_i(x_i))
\]
whenever the saturated version of $\Constraints{\beta}$ contains the conjunct
\(
m < t_j - t_i < n
\).
(This can be derived using
$\tvals_k(x_j) = \rho_j(x_j) - (t_k - t_j)$
and
$\tvals_k(x_i) = \rho_i(x_i) - (t_k - t_i)$).

Let $\beta$ be a feasible untimed behavior and let $x \in X$ be a timer. Then we say that $x$ is \emph{expirable} after $\beta$
if there exists a valuation in $\Zone{\beta}$ in which $x$ is minimal.

