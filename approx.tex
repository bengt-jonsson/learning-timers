\section{Approximating the Nerode Equivalence}
\label{sec:approx}

In this section, we present a framework for approximating the Nerode equivalence
defined in Definition~\ref{def:nerode} using finite sets of suffixes. This
is analogous to the use of finite sets of suffixes in $L^*$ to approximate the
standard Nerode congruence on languages over finite words. A natural approach
would be to define an approximated Nerode equivalence on untimed behaviors,
parameterized by a finite set $V$ of suffixes, by letting 
untimed behaviors $\beta$ and $\beta'$ be equivalent
iff there exists a bijection 
$f_0 : X \to X'$ between some sets of timers \todobj{to be defined}
such that for any untimed behavior $\gamma \in V$:
if $\beta \cdot \gamma \in S$ then there exists an isomorphism $f$ that extends $f_0$ such that $\beta' \cdot f(\gamma) \in S$, and vice versa.
%% (b) if $\beta' \cdot \gamma \in S$ then there exists an isomorphism $f$ that extends $f_0^{-1}$ such that $\beta \cdot f(\gamma) \in S$.
In order to make this idea work, we must let the sets $V$ of suffixes be
closed under renaming of timers.

To illustrate this, consider a simple MMT, which can perform the untimed runs
\(
q_0 \xrightarrow{i_1/o_1, x_1:= 5} q_2
\)
or
\(
q_0 \xrightarrow{i_1/o_1} q_0 \xrightarrow{i_2/o_2, x_1:= 5} q_2
\),
and that from $q_2$ it can perform
\(
q_2 \xrightarrow{\toevent{x_1}/o_3} q_3
\).
The canonical untimed behaviors corresponding to the first two
untimed runs are
\(
\emptyset \xrightarrow{i_1/o_1, x_1:= 5} \set{x_1}
\)
and
\(
\emptyset \xrightarrow{i_1/o_1} \emptyset \xrightarrow{i_2/o_2, x_2:= 5} \set{x_2}
\).
In order to let these two behaviors be equivalent we must discover that
timer $x_1$ can expire after the first one, and that
timer $x_2$ can expire after the second one. By thinking slightly more about
this, one finds out that if the set $V$ of suffixes includes
the effect of a timeout $\toevent{x_1}$ for some timer $x_1$, then it must
include the effects of all timeout events of form $\toevent{x_i}$, otherwise we
may fail to overapproximate the Nerode equivalence of Definition~\ref{def:nerode}.

\todobj{We need to decide how to handle sets of timers in ``partial''
  untimed behaviors}

Let us introduce a formal {\em timer parameter}, denoted $p$, and the
{\em generic timeout} $\toevent{p}$,
which intuitively denotes the occurrence of any timout.
Let a {\em generic untimed behavior} be a sequence of symbols in
$I \cup \set{\toevent{p}}$.
A {\em instance} of a generic untimed behavior is obtained by replacing all
generic timeout $\toevent{p}$ by some concrete timeout $\toevent{x_i}$.
For a set $V$ of generic untimed behaviors, let $\instancesof{V}$ be the
set of instances of generic untimed behaviors in $V$. For a canonical untimed
behavior $\beta$, let $\suffixbehs{S}{\beta}{V}$ be the set of untimed behaviors
$\gamma$ with $\untimedinputword(\gamma) \in \instancesof{V}$ such that
$\beta\cdot\gamma$ is a canonical untimed behavior in $S$. Let
$\getmemorable{S}{\beta}{V}$ be the set of timers among
$x_1 , \ldots x_{|\beta|}$ that occur in some behavior in
$\suffixbehs{S}{\beta}{V}$.
Let $\suffbij{\beta}{\beta'}$ be the mapping on timers that map each
timer $x_i$ with $i > |\beta|$ to $x_{i + |\beta'| - |\beta|}$.

\todobj{An illustrating example would help the reader here}

The approximated Nerode equivalence can now be defined as follows.
Let $\beta$ and $\beta'$ be behaviors in $S$, and let  $V$ be a set of
generic untimed behaviors.
Let $f : \getmemorable{S}{\beta}{V} \to \getmemorable{S}{\beta'}{V}$
be a bijection
from $\getmemorable{S}{\beta}{V}$ to $\getmemorable{S}{\beta'}{V}$.
Then $\beta$ and $\beta'$ are \emph{equivalent wrp $V$} under $f$, written
$\beta \equiv_{S,V}^f \beta'$, 
\[
\gamma \in \suffixbehs{S}{\beta}{V}
\qquad \mbox{iff} \qquad
(\suffbij{\beta}{\beta'} \circ f) (\gamma) \in \suffixbehs{S}{\beta'}{V}
\]
Intuitively, $\beta \equiv_{S,V}^f \beta'$ means that $\beta$ and $\beta'$
have the same suffixes with inputs in $\instancesof{V}$, after performing the
obvious renamings of timers that rename timers in $\beta$ according to
$f$ and timers in each suffix according to $\suffbij{\beta}{\beta'}$.
We write $\beta \equiv_{S,V} \beta'$ to denote that
$\beta \equiv_{S,V}^f \beta'$ for some
$f : \getmemorable{S}{\beta}{V} \to \getmemorable{S}{\beta'}{V}$.

\todobj{We now need to prove theorems about this approximating equivalence.
  We should make the definition of Nerode use the same notation, whenever
  reasonable, to make this easier}

In order to use the above approximated equivalence for learning, we need to
present a procedure for constructing the set
$\suffixbehs{S}{\beta}{V}$ by a sequence of timed membership queries.

We should also construct an automaton from a set of results of such
suffix queries, provided they satisfy some conditions
\todobj{closedness, etc, which we must define}

Let us first present a procedure for answering suffix queries. It is
sufficient to perform them for one generic suffix at a time.
The input to a suffix query is a (feasible) untimed behavior $\beta$ an a
generic untimed input behavior $v$. It returns
$\suffixbehs{S}{\beta}{\set{v}}$. To compute this, it must be determined
\begin{itemize}
\item which instantiations of generic timeouts in $v$ are feasible after $\beta$, and
\item what is the output in response to an input.
\end{itemize}
The second item is trivially answered by observing the output in response to
each relevant input. Note that if a behavior $\beta$ is feasible, then
any extension of $\beta$ by a non-timeout input is trivially feasible, simply
by supplying the input as fast as possible. Thus the only non-trivial part
of answering a suffix query is to determine which instantiations of a generic
timeout in $v$ are feasible. This is easily reduced to the problem of
determining, for a feasible behavior $\beta$, which timeouts of form
$\toevent{x_i}$ with $i \leq |\beta|$ are feasible (note that $\beta$ here in
general denotes an extension of the $\beta$ supplied as prefix in a
suffix query). Let us now describe how this can be done.

In order to determine which behaviors of form $\beta \cdot \toevent{x_i}$
are feasible, we perform a timed membership query, which provide the longest
possible time delay between the $i$th transition in $\beta$ and the time
of any other timeout that can feasibly expire after $\beta$. In order to
maximize this time, we need to know which other timeouts may expire while
performing $\beta \cdot \toevent{x_i}$. Such timeouts are either expiring
as part of $\beta$, or disturb the execution of $\beta \cdot \toevent{x_i}$,
in which case our membership query must know how to prevent them from occuring.

Les us introduce notation. Let $\beta$ be of form
\(
  {i_1/o_1, \rho_1} \ \cdot \ {i_n/o_n, \rho_n}
\)
We are looking for a corresponding timed behavior of form 
\[
\tvals_0 \xrightarrow{d_1} \tvals'_0 \xrightarrow{i_1/o_1, \rho_1}
\quad \cdots \quad
\xrightarrow{d_n} \tvals'_n \xrightarrow{i_n/o_n, \rho_n} \tvals_n
\xrightarrow{d_{n+1}} \tvals'_{n+1} \xrightarrow{\toevent{x_i}/o_{n+1}, \rho_n}
\]
The timed membership queries need to select values
of $\vect{d}{n+1}$, and gradually adds new timers to the word together
with constraints on $\vect{d}{n+1}$ induced by these timers.
Let $\delay{j}{k}$ denote $d_{j+1} + d_{j+2} + \cdots + d_k$.

The algorithm maintains an increasing constraint $\constr$ on delays of form
$\delay jk$. Initially, the constraint contains the conjunct
$\delay jk = D$ whenever $i_k$ is of form $\toevent{x_j}$ and
$\rho_j$ sets $x_j$ to $D$.
Thereafter, the following procedure is performed
  \begin{enumerate}
  \item
    find $\vect d{{n+1}}$ that maximizes $\delay{i}{(n+1)}$ given
    $\constr$ (note that $d_{n+1} = \infty$ is allowed)
  \item
    supply membership query with delay values $\vect d{{n+1}}$.
\begin{itemize}
  \item
    {\bf if} the output is of form $\beta$ followed by $\toevent{x_i}$, then
    report that $\toevent{x_i}$ is feasible and stop
  \item
    {\bf else if} no timeout occurs within $d_{n+1}$ after $\beta$, report
    that $\toevent{x_i}$ is not feasible and stop
  \item
    {\bf else if} another timeout of form $\toevent{x_j}$ occurs before
    the $k$th input, this means that timer $x_j$ is assigned a value
    $D_j$ at the $j$ transition and may expire before the $k$th transition
    in the untimed trace. In order to prevent this from happening,
    add the constraint $\delay jk < D_j$ to $\constr$ and start again
    from step 1.
\end{itemize}
  \end{enumerate}
  Intuitively, the procedure aims to find a timed membership query which maximise
  the time within which $\toevent{x_i}$ is able to expire,
  subject to the constraints under which timers that expire in $\beta$ expire
  as scheduled, and under which any other timers do not expire. These
  other timers may not be known when the procedure starts, and so their
  corresponding constraints are added when they are detected: this happens
  when they expire unexpectectly.

  Note that conjunctions of constraints on delays $\delay{j}{k}$ can be represented as DBM, as is standard for timed automata.

\section{Algorithm for Learning of MMTs}

In this section, we provide a more specialized approach for learning MMTs based
on queries for untimed behaviors, outlined previously in this section.
It might potentially replace the text in Section~\ref{sect:learn-mm}.

The learning algorithm follows the standard pattern for automata learning
algorithms, such as $L^*$. It maintains
a set $U$ of {\em short prefixes}, and
  an overapproximation of the Nerode equivalence,
  represented by a set $V$ of generic untimed behaviors.
  Each short prefix is an untimed
behavior, which represent a state in the MMT to be constructed.
During hypothesis construction,
the approximation of the Nerode equivalence triggers the expansion of
$U$ until it is closed, so that a hypothesis MMT can be formed.
During hypothesis validation, returned counterexamples are used to refine
the Nerode equivalence by expanding $V$.

It is required that 
Let $U$ be a set of untimed behaviors, and let $V$ be a set of generic suffix
inputs, which contains all one-symbol words, including $\toevent{p}$.
\begin{itemize}
\item
$U$ is {\em closed} wrt.\ $V$ if 
  for each $\beta \in U$ and each
  $i$ in $I$ or of form $\toevent{x_i}$ such thas $x_i$ is expirable
  after $\beta$,
  there is an output $o$ and a $\beta' \in U$ such that
  $\beta \cdot i/o,\rho_{i+1} \equiv_{S,V} \beta'$.
\item
$U$ is {\em timer-consistent} wrt.\ $V$ if 
  for each $\beta \in U$ and each
  $i$ in $I$ or of form $\toevent{x_i}$ such thas $x_i$ is expirable
  after $\beta$, we have
  $\getmemorable{S}{\beta \cdot i/o,\rho_{i+1}}{V} \subseteq
  (\getmemorable{S}{\beta}{V} \cup \domof{\rho_{i+1}})$.
\end{itemize}
Closedness ensures 
that each transition in the MMT to be constructed has a target location. 
Timer-consistency states that each timer in $\beta \cdot i/o,\rho_{i+1}$
is either a timer in $\beta$ or is started by the transition $i/o,\rho_{i+1}$.
In the automaton to be constructed, it ensures that
any timer started during $\beta$ that is active in $\beta \cdot i/o,\rho_{i+1}$
is also active in $\beta$.

Whenever the set $U$ is not closed wrt.\ $V$, then $U$ is extended:
if there is some $\beta \in U$ and $i$ in $I$ or of form $\toevent{x_i}$ with
$x_i$ expirable, for which there is no $o$ and $\beta' \in U$ such that
$\beta \cdot i/o,\rho_{i+1} \equiv_{S,V} \beta'$, 
then $\beta \cdot i/o,\rho_{i+1}$ is added to $U$,
triggering new membership queries.
Whenever the set $U$ is not timer-consistent wrt.\ $V$, then $V$ is extended:
for any suffix of form $\gamma \cdot \toevent{x_j}$, where
$x_j$ is a ``missing timer'', we make sure that $i/o,\rho_{i+1} \cdot \gamma \cdot \toevent{p}$ is in $V$.
    This process of
    extending $U$ and $V$ is continued until $U$ is closed
    and timer consistent wrt.\ $V$.

\begin{definition}[Hypothesis automaton]
  Let $U$ be a prefix-closed set
  of untimed behaviors, which contains $\emptyword$,
  and $V$ a set of generic suffix inputs such that
$U$ is closed and timer consistent wrt.\ $V$. Then the
{\em hypothesis automaton} $\hypoof UV$ is the MMT
$\hypoof UV = (I, O, Q, q_0, \vars, \delta, \lambda, \remap)$, where
\begin{itemize}
\item $Q = U$ and $q_0 = \emptyword$,
\item $\vars$ maps each location $\beta\in U$ to $\getmemorable{S}{\beta}{V}$,
\item $\delta$ maps each $\beta \in U$ and input
  $i$, either in $I$ or of form $\toevent{x_i}$,
  to the unique $\beta' \in U$ such that
$\beta \cdot i/o,\rho_{i+1} \equiv_{S,V}^f \beta'$.
\item $\lambda$ maps each $\beta \in U$ and input
  $i$, either in $I$ or of form $\toevent{x_i}$, to the output observed in
  reponse to $i$ after $\beta$.
\item $\remap$
 maps each $\beta \in U$ and input
$i$, either in $I$ or of form $\toevent{x_i}$, to the function
  $\remap(\beta,i):\getmemorable{S}{\beta'}{V} \mapsto (\getmemorable{S}{\beta}{V} \cup \natplus)$ defined as $\rho_{i+1} \circ f^{-1}$, where $\rho_{i+1}$ and $f$
  are as in the definition of $\delta$.
\end{itemize}
\end{definition}

\paragraph{Counterexample processing}
We should also indicate a procedure for processing counterexamples.
