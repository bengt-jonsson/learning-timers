\section{Another Try on the Precise Nerode Equivalence}

Let us introduce notation to make the subsequent development more convenient. 
Recall that a canonical untimed behavior is \emph{canonical} when, for each $j$,
the timer that is updated in the $j$-th event (if any) is equal to $x_j$.
It is {\em lean} if it is canonical and
includes only timers that expire during the behavior. 
Since timers in lean behaviors are uniquely determined by the labels on
transition, we can denote lean behaviors simply as sequences
$i_1/o_1,\rho_1 \cdots i_n/o_n,\rho_n$ of input/output/assignment triples.
For behaviors that are to be appended as suffixes, we need to distinguish
timers that are assigned in the suffix from timers that are assigned in a prefix.
Therefore, extend the set of timers by the set $Y = \set{y_1,y_2,\ldots}$ of
{\em suffix timers}, which is disjoint from $X$.
Define a {\em suffix behavior} to be a sequence
$i_1/o_1,\rho_1 \cdots i_m/o_m,\rho_m$ of input/output/assignment triples,
in which the inputs are in $I$, of form $\toevent{x_j}$ for $x_j \in X$, or
of form $\toevent{y_j}$ for $y_j \in Y$, but where assignments may assign only
to timers in $Y$. Such a suffix behavior is {\em lean} if
timer $y_j$ is assigned only by $\rho_j$, each timeout event occurs at most once, and all assigned timers in $Y$ expire sometime after their assignment.

For integer $k \geq 0$, let $\suffmap{k}$ be the injective mapping on
$Y$ which maps each $y_j$ to $y_{j+k}$.  We apply mappings $\suffmap{k}$ to
lean suffix behaviors in the natural way.

Let $\beta$ be a lean behavior. We say that 
a lean suffix behavior $\gamma$ is a {\em $\beta$-suffix} if
there is a canonical behavior $\beta'$ with $\beta \sqsubseteq \beta'$ such that
$\beta'\cdot\suffmap{|\beta|}(\gamma)$ is a lean behavior.

We can define general residual languages, by letting
$\suffixbehs{S}{\beta}$ be the set $\beta$-suffixes.
Let $\getmemorable{S}{\beta}$ be the set of timers $x_i$ in
$x_1 , \ldots x_{|\beta|}$ whose corresponding timeout event
(of form $\toevent{x_i}$) occurs (as an input) in some $\beta$-suffix.
Let $\getassignment{S}{\beta}$ map each timer $x_i$ in
$\getmemorable{S}{\beta}$ to the unique positive integer to which it
is assigned in the $i$th transition of $\beta$.

We can now defined the Nerode equivalence.

\begin{definition}
  \label{def:bj-nerode}
Let $S$ be a timer language \todobj{Define this},
let $\beta$ and $\beta'$ be canonical untimed behaviors in $S$, 
and let $f : \getmemorable{S}{\beta} \to \getmemorable{S}{\beta'}$
be a bijection
from $\getmemorable{S}{\beta}$ to $\getmemorable{S}{\beta'}$.
Then $\beta$ and $\beta'$ are \emph{equivalent} under $f$, written
$\beta \equiv_{S}^f \beta'$ iff
\[
\gamma \in \suffixbehs{S}{\beta}
\qquad \mbox{iff} \qquad
f(\gamma) \in \suffixbehs{S}{\beta'}
\]
\end{definition}
Intuitively, $\beta \equiv_{S}^f \beta'$ means that $\beta$ and $\beta'$
allow the same suffixes, after renaming
timers assigned in $\beta$ by $f$.
We write $\beta \equiv_{S} \beta'$ to denote that
$\beta \equiv_{S}^f \beta'$ for some
$f : \getmemorable{S}{\beta} \to \getmemorable{S}{\beta'}$.

\begin{theorem}
  \label{thm:bj-nerode}
Let $S$ be a timer language.
Then there exists an MMT with feasible canonical untimed behaviors $S$ iff 
$\equiv_S$ has finitely many equivalence classes (finite index).
\end{theorem}
\begin{proof} 
``$\Leftarrow$'' This direction will produce the following MTT:
\begin{definition}[Constructed MMTautomaton]
\label{def:mmt-construction}
The constructed MMT is the MMT
$\M = (I, O, Q, q_0, \vars, \delta, \lambda, \remap)$, where
\begin{itemize}
\item $Q$ is the set of equivalence classes under $\equiv_S$. We let each
  equivalence class be represented by one of its members, called its
  {\em access string}. In the following, when we write $[\beta]$,
  this denotes that $\beta$ is the access string of the equivalence class $[\beta]$.
\item $q_0 = \emptyword$,
\item $\vars$ maps each equivalence  class $[\beta]\in U$ to
  $\getmemorable{S}{\beta}$,
\item $\lambda$ maps each $[\beta] \in U$ and $i \in \feasibleinputs{\beta}{S}$ to
  the unique $o$ such that $\extend{\beta}{i}{o} \in S$. 
  %% $i$, either in $I$ or of form $\toevent{x_i}$, to the output observed in
  %% reponse to $i$ after $\beta$.
\item $\delta$ maps each $\beta \in U$ and
  $i \in \feasibleinputs{\beta}{S}$ to
  to the unique $\beta' \in U$ such that there is an $f$ with
  $\extend{\beta}{i}{\lambda(\beta,i)} \equiv_{S}^f \beta'$.
\item $\remap$
 maps each $\beta \in U$ and $i \in \feasibleinputs{\beta}{S}$ to
  the function
  $\remap(\beta,i):\getmemorable{S}{\beta'} \mapsto (\getmemorable{S}{\beta} \cup \natplus)$ defined as $\rho \circ f^{-1}$, where $\rho$ maps 
  timer $x_{(|\beta|+1)}$ to
$\getassignment{S}{(\extend{\beta}{i}{\lambda(\beta,i)})}(x_{(|\beta|+1)})$.
\end{itemize}
\end{definition}
\todobj{The rest of this is just a sketch. More care is needed with timers}
``$\Rightarrow$'' Let $\M$ be an MMT, let $S$ be its set of feasible untimed
  behaviors. It follows from \todobj{fix} that $S$ is a timer language.
Suppose that $\beta, \beta' \in S$.
Then there exist (unique) $\delta, \delta$ of $T$ with
$\can{\delta}=\beta$ and $\can{\delta'}=\beta'$.
Suppose $\delta$ and $\delta'$ lead to the same state $q$ of $\M$.
We claim that $\beta \equiv_S \beta'$.
Since $\beta = \can{\delta}$, there exist isomorphisms
$g = g_0 ,\ldots, g_k$  $\beta$ to $\delta$.
Similarly, there exists isomorphisms $g' = g'_0 ,\ldots, g'_l$ from
$\beta'$ to $\delta'$.
Let $f = (g'_l)^{-1} \circ g_k$.
Suppose that, for some untimed behavior $\gamma$, $\beta \cdot \gamma \in S$.
Then there exists an untimed behavior $\zeta$ such that $\delta \cdot \zeta \in T$ and $\can{\delta \cdot \zeta} = \beta \cdot \gamma$.
Moreover, there exists an isomorphism $h = g_k ,\ldots, g_m$ from $\gamma$ to $\zeta$.
Since $\delta \cdot \zeta$ is a feasible behavior in $T$, it follows that
$\delta' \cdot \zeta$ is a feasible behavior in $T$
\todobj{This should follow from the observation that the same paths are possible
  after $\delta$ and $\delta'$ (we have the assumption that all paths are
  feasible}
Thus there exists an untimed behavior $\gamma'$ such that $\can{\delta' \cdot \zeta} = \beta' \cdot \gamma' \in S$.
Moreover, we may construct an isomorphism $h' = g'_l ,\ldots, g'_n$ from $\gamma'$ to $\zeta$.
Then $f = (g'_l)^{-1} \circ g_k \cdots (g'_n)^{-1} \circ g_m$ is an isomorphism from $\gamma$ to $\gamma'$ that extends $f_0$.
We then have $\beta' \cdot f(\gamma) \in S$, as required.
\todobj{To be continued}
\end{proof}

\section{Approximating the Nerode Equivalence}
\label{sec:approx}

In this section, we present an approximation the Nerode equivalence on
untimed behaviors
defined in Definition~\ref{def:nerode}, which is parameterized on a
finite sets of suffixes. This
approximated equivalence can be inferred using a finite set of membership
queries, and therefore be used as a basis for a learning algorithm,
analogously to the use of an approximated Nerode equivalence in
$L^*$~\cite{Ang87}.
A natural approach
would be to define an approximated Nerode equivalence on untimed behaviors,
parameterized by a finite set $V$ of untimed input words (hereafter often
called {\em suffixes}), by letting
untimed behaviors $\beta$ and $\beta'$ be equivalent
iff there exists a bijection 
$f_0 : X \to X'$ between the timers that are still active after
$\beta$ and $\beta'$, respectively,
such that for any untimed behavior $\gamma$ with $\untimedinputword(\gamma)
\in V$:
if $\beta \cdot \gamma \in S$ then there exists an isomorphism $f$ that extends $f_0$ such that $\beta' \cdot f(\gamma) \in S$, and vice versa.
%% (b) if $\beta' \cdot \gamma \in S$ then there exists an isomorphism $f$ that extends $f_0^{-1}$ such that $\beta \cdot f(\gamma) \in S$.

In order to make this idea work, we must let the sets $V$ of suffixes be
closed under renaming of timers that are active after a prefix $\beta$.
To illustrate why, consider a simple MMT, which can perform the untimed runs
\(
q_0 \xrightarrow{i_1/o_1, x_1:= 5} q_2
\)
or
\(
q_0 \xrightarrow{i_1/o_1} q_0 \xrightarrow{i_2/o_2, x_1:= 5} q_2
\),
and that from $q_2$ it can perform
\(
q_2 \xrightarrow{\toevent{x_1}/o_3} q_3
\).
The canonical untimed behaviors corresponding to the first two
untimed runs are
\(
\emptyset \xrightarrow{i_1/o_1, x_1:= 5} \set{x_1}
\)
and
\(
\emptyset \xrightarrow{i_1/o_1} \emptyset \xrightarrow{i_2/o_2, x_2:= 5} \set{x_2}
\).
In order to let these two behaviors be equivalent we must discover that
timer $x_1$ can expire after the first one, and that
timer $x_2$ can expire after the second one. By thinking slightly more about
this, one finds out that if the set $V$ of suffixes includes
the effect of a timeout $\toevent{x_1}$ for some timer $x_1$, then it must
include the effects of the timeout events of form $\toevent{x_i}$ for any
timer $x_i$ that is assigned in a prefix, otherwise we
may fail to overapproximate the Nerode equivalence of Definition~\ref{def:nerode}.

Let us introduce notation to make the subsequent development more convenient. 
Recall that a canonical untimed behavior is \emph{canonical} when, for each $j$,
the timer that is updated in the $j$-th event (if any) is equal to $x_j$.
It is {\em lean} if it is canonical and
includes only timers that expire during the behavior. 
Since timers in lean behaviors are uniquely determined by the labels on
transition, we can denote lean behaviors simply as sequences
$i_1/o_1,\rho_1 \cdots i_n/o_n,\rho_n$ of input/output/assignment triples.
For behaviors that are to be appended as suffixes, we need to distinguish
timers that are assigned in the suffix from timers that are assigned in a prefix.
Therefore, extend the set of timers by the set $Y = \set{y_1,y_2,\ldots}$ of
{\em suffix timers}, which is disjoint from $X$.
Define a {\em suffix behavior} to be a sequence
$i_1/o_1,\rho_1 \cdots i_m/o_m,\rho_m$ of input/output/assignment triples,
in which the inputs are in $I$, of form $\toevent{x_j}$ for $x_j \in X$, or
of form $\toevent{y_j}$ for $y_j \in Y$, but where assignments may assign only
to timers in $Y$. Such a suffix behavior is {\em lean} if
timer $y_j$ is assigned only by $\rho_j$, each timeout event occurs at most once, and all assigned timers in $Y$ expire sometime after their assignment.

For integer $k \geq 0$, let $\suffmap{k}$ be the injective mapping on
$Y$ which maps each $y_j$ to $y_{j+k}$.  We apply mappings $\suffmap{k}$ to
lean suffix behaviors in the natural way.

Let $\beta$ be a lean behavior. We say that 
a lean suffix behavior $\gamma$ is a {\em $\beta$-suffix} if
there is a canonical behavior $\beta'$ with $\beta \sqsubseteq \beta'$ such that
$\beta'\cdot\suffmap{|\beta|}(\gamma)$ is a lean behavior. In this case we
use $\beta;\gamma$ to denote $\beta'\cdot\suffmap{|\beta|}(\gamma)$.

We can define general residual languages, by letting
$\suffixbehs{S}{\beta}$ be the set $\beta$-suffixes.
Let $\getmemorable{S}{\beta}$ be the set of timers $x_i$ in
$x_1 , \ldots x_{|\beta|}$ whose corresponding timeout event
(of form $\toevent{x_i}$) occurs (as an input) in some $\beta$-suffix.
Let $\getassignment{S}{\beta}$ map each timer $x_i$ in
$\getmemorable{S}{\beta}$ to the unique positive integer to which it
is assigned in the $i$th transition of $\beta$.

Let an {\em input suffix} be a sequence of elements
in $I$, of form $\toevent{x_j}$ for $x_j \in X$, or
of form $\toevent{y_j}$ for $y_j \in Y$.
A set $V$ of input suffixes is {\em adequate} if it is closed under permutations
on $X$ and includes all input suffixes of length one.

For an adequate set $V$ of input suffixes, and a lean behavior $\beta$,
let $\apprsuffixbehs{S}{\beta}{V}$ be the set of $\beta$-suffixes $\gamma$
with $\untimedinputword(\gamma) \in V$.
Let $\apprgetmemorable{S}{\beta}{V}$ be the set of timers $x_i$ in
$x_1 , \ldots x_{|\beta|}$ whose corresponding timeout event
(of form $\toevent{x_i}$) occurs (as an input) in some untimed behavior in
$\apprsuffixbehs{S}{\beta}{V}$.
Let $\apprgetassignment{S}{\beta}{V}$ map each timer $x_i$ in
$\apprgetmemorable{S}{\beta}{V}$ to the unique positive integer to which it
is assigned in the $i$th transition of that behavior.


%% \todobj{Maybe we should point out that two prefixes $\beta$ and $\beta'$
%%   can be equivalent even if they assign corresponding timers to different
%%   values. However, timers assigned in corresponding suffixes must
%%   be assigned the same values.}

%% Let us introduce the {\em generic timeout event} $\toevent{p}$, where $p$w can
%% be regarded as a formal parameter,
%% which can be instantiated to an arbitrary timout event.
%% Let $(I \cup \set{\toevent{p}})^*$ be the set of
%% {\em generic untimed input words}.
%% An {\em instance} of a generic untimed input word is an untimed input
%% word obtained by replacing each generic timeout event
%% $\toevent{p}$ by a concrete timeout event.
%% For a set $V$ of generic untimed input words, let $\instancesof{V}$ be the
%% set of instances of words in $V$. For a canonical untimed
%% behavior $\beta$, let $\apprsuffixbehs{S}{\beta}{V}$ be the set of untimed behaviors
%% $\gamma$ with $\untimedinputword(\gamma) \in \instancesof{V}$ such that
%% $\beta\cdot\gamma$ is a canonical untimed behavior in $S$. Let
%% $\apprgetmemorable{S}{\beta}{V}$ be the set of timers $x_i$ in
%% $x_1 , \ldots x_{|\beta|}$ whose corresponding timeout event
%% (of form $\toevent{x_i}$) occurs (as an input) in some untimed behavior in
%% $\apprsuffixbehs{S}{\beta}{V}$.
%% Let $\apprgetassignment{S}{\beta}{V}$ map each timer $x_i$ in
%% $\apprgetmemorable{S}{\beta}{V}$ to the unique positive integer to which it
%% is assigned in the $i$th transition of that behavior.


%% Let $\suffbij{\beta}{\beta'}$ be the partial injective mapping
%% on timers that maps each
%% timer $x_i$ with $i > |\beta|$ to $x_{i + |\beta'| - |\beta|}$.
%% Intuitively, $\suffbij{\beta}{\beta'}$ maps a timer that is assigned 
%% in a transition of $\gamma$, where $\gamma$ is a suffix of the canonical
%% untimed behavior $\beta \cdot \gamma$, to the corresponding timer of
%% the canonical untimed behavior $\beta'\cdot\gamma$.
%% For two partial mappings $f$, $g$ on $X$, we let $f \sqcup g$ be their
%% union. \todobj{Does this need to be further clarified?}

%% \todobj{An illustrating example would help the reader here}

We can now defined the approximated Nerode equivalence, which is parameterized
on an adequate set of input suffixes.

\begin{definition}
  \label{def:approx-nerode}
Let $S$ be a timer language \todobj{Define this},
let $\beta$ and $\beta'$ be canonical untimed behaviors in $S$,
and let  $V$ be an adequate set of input suffixes.
Let $f : \apprgetmemorable{S}{\beta}{V} \to \apprgetmemorable{S}{\beta'}{V}$
be a bijection
from $\apprgetmemorable{S}{\beta}{V}$ to $\apprgetmemorable{S}{\beta'}{V}$.
Then $\beta$ and $\beta'$ are \emph{equivalent wrp $V$} under $f$, written
$\beta \equiv_{S,V}^f \beta'$ iff
\[
\gamma \in \apprsuffixbehs{S}{\beta}{V}
\qquad \mbox{iff} \qquad
f(\gamma) \in \apprsuffixbehs{S}{\beta'}{V}
\]
\end{definition}
Intuitively, $\beta \equiv_{S,V}^f \beta'$ means that $\beta$ and $\beta'$
allow the same suffixes with inputs in $V$, after renaming
timers assigned in $\beta$ by $f$.
We write $\beta \equiv_{S,V} \beta'$ to denote that
$\beta \equiv_{S,V}^f \beta'$ for some
$f : \apprgetmemorable{S}{\beta}{V} \to \apprgetmemorable{S}{\beta'}{V}$.


\todobj{What theorems should we prove about this approximating equivalence?
  We should make the definition of Nerode use the same notation, whenever
  reasonable, to make this easier}

\section{Algorithm for Learning of MMTs}
\label{sec:learning}

In this section, we present an algorithm for learning MMTs in then untimed
MAT of~\ref{sec:untimed-mat}, using the approximated Nerode equivalence
presented in Section~\ref{sec:approx}.
The learning algorithm follows the standard pattern for active automata learning
algorithms, such as $L^*$~\cite{Ang87}. It maintains
a set $U$ of {\em short prefixes}, and
  an overapproximation of the Nerode equivalence,
  parameterized by a set $V$ of input suffixes.
Each short prefix in $U$ is a feasible canonical untimed
behavior, which represents a state in the MMT to be constructed.
The learning algorithm iterates two phases: hypothesis construction and
hypothesis validation.
During hypothesis construction,
the approximation of the Nerode equivalence triggers the expansion of
$U$ and $V$ until two convergence conditions are satisfied that allow
a hypothesis automaton to be formed.
During hypothesis validation, the hypothesis automaton is submitted in an
equivalence query, and returned counterexamples are used to refine
the Nerode equivalence by expanding $V$.

\todobj{Say somewhere that we assume a fixed (un)timed language $S$.}

Let us introduce the two conditions for convergence of the construction phase.
For a feasible untimed word $\beta$, let $\feasibleinputs{\beta}{S}$ be the
set of $i \in \extinputs$ such that there is a $\beta$-suffix of form
$i/o,\rho$ for some $o$ and $\rho$.
For $i \in \feasibleinputs{\beta}{S}$, let $\lambda(\beta,i)$ be the unique
output $o$ such that $i/o,\rho$ is a $\beta$-suffix for some $\rho$.
%% A set $V$ of generic untimed input words is {\em adequate} if it includes
%% $I \cup \set{\toevent{p}}$.
Let $U$ be a prefix-closed set of feasible lean behaviors.
and let $V$ be an adequate set of input suffixes.
\begin{itemize}
\item
$U$ is {\em closed} wrt.\ $V$ if 
  for each $\beta \in U$ and
  $i \in \feasibleinputs{\beta}{S}$
%%   , which is either in $I$ or of form $\toevent{x_i}$ where $x_i$ is expirable after $\beta$ if
  there is a $\beta' \in U$ such that
  $\beta;i/\lambda(\beta,i) \equiv_{S,V} \beta'$.
\item
$U$ is {\em timer-consistent} wrt.\ $V$ if 
  for each $\beta \in U$ and
  $i \in \feasibleinputs{\beta}{S}$
  %% $i$, which is either in $I$ or
  %% of form $\toevent{x_i}$ for $x_i$ expirable after $\beta$,
  we have
  $\apprgetmemorable{S}{\beta;\lambda(\beta,i)}{V} \subseteq
  (\apprgetmemorable{S}{\beta}{V} \cup \set{x_{(|\beta|+1)}})$.
\end{itemize}
Closedness ensures that each transition in the MMT to be constructed has a target location. 
Timer-consistency states that each timer which is needed after such
a transition (i.e., a timer set during $\beta;7O\lambda(\beta,i)$)
is either a timer active after $\beta$ or is started by the last transition, thus
getting the name $x_{(|\beta|+1)}$.
Closedness and timer-consistency allow the construction of a hypothesis MMT.

\todobj{Note that lean behaviors of form $\beta;i/\lambda(\beta,i),\rho$
  always have an empty $\rho$, hence we can omit $\rho$}

\begin{definition}[Hypothesis automaton]
\label{def:hypo}
  Let $U$ be a non-empty prefix-closed set of untimed behaviors,
  and $V$ an adequate set of input suffixes such that
  $U$ is closed and timer consistent wrt.\ $V$. Then the
{\em hypothesis automaton} $\hypoof UV$ is the MMT
$\hypoof UV = (I, O, Q, q_0, \vars, \delta, \lambda, \remap)$, where
\begin{itemize}
\item $Q = U$ and $q_0 = \emptyword$,
\item $\vars$ maps each location $\beta\in U$ to $\apprgetmemorable{S}{\beta}{V}$,
\item $\lambda$ maps each $\beta \in U$ and $i \in \feasibleinputs{\beta}{S}$ to
  the unique $o$ such that $\extend{\beta}{i}{o} \in S$. 
  %% $i$, either in $I$ or of form $\toevent{x_i}$, to the output observed in
  %% reponse to $i$ after $\beta$.
\item $\delta$ maps each $\beta \in U$ and
  $i \in \feasibleinputs{\beta}{S}$ to
  to the unique $\beta' \in U$ such that there is an $f$ with
  $\extend{\beta}{i}{\lambda(\beta,i)} \equiv_{S,V}^f \beta'$.
\item $\remap$
 maps each $\beta \in U$ and $i \in \feasibleinputs{\beta}{S}$ to
  the function
  $\remap(\beta,i):\apprgetmemorable{S}{\beta'}{V} \mapsto (\apprgetmemorable{S}{\beta}{V} \cup \natplus)$ defined as $\rho \circ f^{-1}$, where $\rho$ maps 
  timer $x_{(|\beta|+1)}$ to
$\apprgetassignment{S}{(\extend{\beta}{i}{\lambda(\beta,i)})}{V}(x_{(|\beta|+1)})$.
\end{itemize}
\end{definition}


Our learning algorithm iterates two phases: hypothesis construction and
hypothesis validation.
During hypothesis construction, 
the approximation of the Nerode equivalence triggers the expansion of
$U$ and $V$ until $U$ is closed and timer-consistent wrpt.\ $V$,
so that a hypothesis MMT can be formed.
During hypothesis validation, returned counterexamples are used to refine
the approximated Nerode equivalence by expanding $V$.

During {\bf hypothesis construction}, membership queries are performed:
\begin{itemize}
\item for all untimed input words of form $\untimedinputword(\beta) \cdot i$
   for $\beta \in U$ and $i \in \extinputs$: this allows to determine
   $\feasibleinputs{\beta}{S}$ and $\lambda(\beta,i)$
   for $\beta \in U$ and $i \in \feasibleinputs{\beta}{S}$.
\item
  for all untimed input words of form $\untimedinputword(\beta) \cdot v$ and
  $\untimedinputword(\beta) \cdot i \cdot v$, where
  $\beta \in U$, $v \in \instancesof{V}$, and
  $i \in \feasibleinputs{\beta}{S}$.
    This allows to compute the approximated Nerode equivalence $\equiv_{S,V}$ on
    the set of untimed words of form $\beta$ and
    $\extend{\beta}{i}{\lambda(\beta,i)}$ with
    $\beta \in U$ and $i \in \feasibleinputs{\beta}{S}$.
    Whenever the set $U$ is not closed wrt.\ $V$, then it is extended:
if there is some $\beta \in U$ and $i$ in $\feasibleinputs{\beta}{S}$
for which there is no $o$ and $\beta' \in U$ such that
$\extend{\beta}{i}{\lambda(\beta,i)} \equiv_{S,V} \beta'$, 
then $\extend{\beta}{i}{\lambda(\beta,i)}$ is added to $U$,
triggering new membership queries.
Whenever the set $U$ is not timer-consistent wrt.\ $V$, then $V$ is extended:
for timer $x_j$ in
$\apprgetmemorable{S}{\extend{\beta}{i}{\lambda(\beta,i)}}{V} \setminus (\apprgetmemorable{S}{\beta}{V} \cup \set{x_{(|\beta|+1)}})$, find an 
untimed behavior of form $\extend{\gamma}{\toevent{x_i}}{o'}$ in
$\apprsuffixbehs{S}{\extend{\beta}{i}{\lambda(\beta,i)}}{V}$.
Then add $i\cdot\untimedinputword(\gamma)\cdot\toevent{p}$ to $V$.
\end{itemize}

When $U$ is closed and timer-consistent wrt.\ $V$, then
a hypothesis MMT $\hypoof UV$ is constructed and
validated by submitting it in an equivalence query.
If the query returns ``yes'', then
the learning is completed, and $\hypoof UV$ accepts $S$.
If the query returns a counterexample word $\alpha$, this is used to extend
$V$ as follows. We assume w.l.o.g.\ that no proper prefix of $\alpha$ is
a counterexample. 
By the fact that $\alpha$ is a counterexample, there is a suffix
$i/o,\rho \cdot \gamma$ of $\alpha$
such that $\beta\cdot i/o,\rho \equiv_{S,V} \beta'$ but
$\beta \cdot i/o,\rho \cdot \gamma \in S \notiff \beta' \cdot \gamma \in S$
for some $\beta,\beta' \in U$.
(To se this, let $\alpha = i_1/o_1,\rho_1 \cdots i_n/o_n,\rho_n$, and
define the sequence $\beta_0, \beta_1, \ldots ,\beta_{n}$ of short prefixes
in $U$ by $\beta_0 = \emptyword$ and
$\beta_{j-1}i_j/o_j,\rho_j \equiv_{S,V} \beta_{i}$ for
$j = 1, \ldots n$, i.e., $\beta_0 \ldots \beta_n$ is the sequence of states
visited when $\hypoof UV$ processes $\alpha$.
Let $\gamma_j$ be the suffix $i_{j+1}/o_{j+1}\rho_{j+1} \cdots i_n/o_n,\rho_n$
of $\alpha$ of length $n-j$. 
By the fact that $\alpha$ is a counterexample, we have
$\beta_0\cdot\gamma_0 \in S \notiff  \beta_{n} \in S$, which implies that
$\beta_{j-1}\cdot\gamma_{j-1} \in S \notiff \beta_j\cdot\gamma_j \in S$
for some $j$;
we can then take $\beta_{j-1}$ as $\beta$ and $\beta_j$ as $\beta'$.)
This means that $\gamma$ is a new separating suffix, and that $V$ should be
extended with the $v$ such that
$\untimedinputword(\gamma) \in \instancesof{\set{v}}$
should be added to
$V$. After adding $v$ to $V$, $U$ is no longer closed
wrt.\ $V$, so the algorithm can resume a next round of
hypothesis construction, which will eventually generate a new hypothesis,
etc.

\todobj{Should we insert a pseudocode description of the algorithm?}

The algorithm enjoys the following properties, which are similar to those
enjoyed by active automata learning algorithms for regular languages,
e.g.,~\cite{Ang87}.

\begin{theorem}
  \label{thm:alg:termination}
Given an MMT $\M$ whose normal form has $n$ states, 
algorithm XXX terminates and produces an equivalence MMT in normal form
using at most $n$ equivalence queries, and
$Y$ untimer membership queries.
\end{theorem}
   
\begin{proof}
Starting from some initial approximations (e.g., the singleton
set consisting of the empty word), the sets $U$ and $V$ are
successively extended, until $U$ contains one element of each equivalence
class of $\equiv_{S}$, and $\equiv_{S,V}$ coincides with
$\equiv_{S}$. 
At termination the hypothesis is correct, by definition of equivalence query.
During the construction, $U$ will never contain two elements that are
equivalent wrp. $\M$.
Since each round of hypothesis construction and validation adds at least one
word to $U$, there can be at most $n$ equivalence queries.
\end{proof}

\todobj{This is to be fixed}
Since each equivalence query adds only one
word to $V$, this means that $|V| \leq n$ when the algorithm finishes,
implying that in total, at most $n^2|I|$ membership queries
will be performed during hypothesis construction.
During hypothesis
validation, at most $2\log(m)$ membership queries need be performed
(in addition to the equivalence query), where
$m$ is the length of the largest counterexample word returned.


\section{Realizing untimed queries by timed queries}
\todobj{This section has been untouched for a while. Ignore it for now.}

In order to use the above approximated equivalence for learning, we need to
present a procedure for constructing the set
$\apprsuffixbehs{S}{\beta}{V}$ by a sequence of timed membership queries.

We should also construct an automaton from a set of results of such
suffix queries, provided they satisfy some conditions
\todobj{closedness, etc, which we must define}

Let us first present a procedure for answering suffix queries. It is
sufficient to perform them for one generic suffix at a time.
The input to a suffix query is a (feasible) untimed behavior $\beta$ an a
generic untimed input behavior $v$. It returns
$\apprsuffixbehs{S}{\beta}{\set{v}}$. To compute this, it must be determined
\begin{itemize}
\item which instantiations of generic timeouts in $v$ are feasible after $\beta$, and
\item what is the output in response to an input.
\end{itemize}
The second item is trivially answered by observing the output in response to
each relevant input. Note that if a behavior $\beta$ is feasible, then
any extension of $\beta$ by a non-timeout input is trivially feasible, simply
by supplying the input as fast as possible. Thus the only non-trivial part
of answering a suffix query is to determine which instantiations of a generic
timeout in $v$ are feasible. This is easily reduced to the problem of
determining, for a feasible behavior $\beta$, which timeouts of form
$\toevent{x_i}$ with $i \leq |\beta|$ are feasible (note that $\beta$ here in
general denotes an extension of the $\beta$ supplied as prefix in a
suffix query). Let us now describe how this can be done.

In order to determine which behaviors of form $\beta \cdot \toevent{x_i}$
are feasible, we perform a timed membership query, which provide the longest
possible time delay between the $i$th transition in $\beta$ and the time
of any other timeout that can feasibly expire after $\beta$. In order to
maximize this time, we need to know which other timeouts may expire while
performing $\beta \cdot \toevent{x_i}$. Such timeouts are either expiring
as part of $\beta$, or disturb the execution of $\beta \cdot \toevent{x_i}$,
in which case our membership query must know how to prevent them from occuring.

Les us introduce notation. Let $\beta$ be of form
\(
  {i_1/o_1, \rho_1} \ \cdot \ {i_n/o_n, \rho_n}
\)
We are looking for a corresponding timed behavior of form 
\[
\tvals_0 \xrightarrow{d_1} \tvals'_0 \xrightarrow{i_1/o_1, \rho_1}
\quad \cdots \quad
\xrightarrow{d_n} \tvals'_n \xrightarrow{i_n/o_n, \rho_n} \tvals_n
\xrightarrow{d_{n+1}} \tvals'_{n+1} \xrightarrow{\toevent{x_i}/o_{n+1}, \rho_n}
\]
The timed membership queries need to select values
of $\vect{d}{n+1}$, and gradually adds new timers to the word together
with constraints on $\vect{d}{n+1}$ induced by these timers.
Let $\delay{j}{k}$ denote $d_{j+1} + d_{j+2} + \cdots + d_k$.

The algorithm maintains an increasing constraint $\constr$ on delays of form
$\delay jk$. Initially, the constraint contains the conjunct
$\delay jk = D$ whenever $i_k$ is of form $\toevent{x_j}$ and
$\rho_j$ sets $x_j$ to $D$.
Thereafter, the following procedure is performed
  \begin{enumerate}
  \item
    find $\vect d{{n+1}}$ that maximizes $\delay{i}{(n+1)}$ given
    $\constr$ (note that $d_{n+1} = \infty$ is allowed)
  \item
    supply membership query with delay values $\vect d{{n+1}}$.
\begin{itemize}
  \item
    {\bf if} the output is of form $\beta$ followed by $\toevent{x_i}$, then
    report that $\toevent{x_i}$ is feasible and stop
  \item
    {\bf else if} no timeout occurs within $d_{n+1}$ after $\beta$, report
    that $\toevent{x_i}$ is not feasible and stop
  \item
    {\bf else if} another timeout of form $\toevent{x_j}$ occurs before
    the $k$th input, this means that timer $x_j$ is assigned a value
    $D_j$ at the $j$ transition and may expire before the $k$th transition
    in the untimed trace. In order to prevent this from happening,
    add the constraint $\delay jk < D_j$ to $\constr$ and start again
    from step 1.
\end{itemize}
  \end{enumerate}
  Intuitively, the procedure aims to find a timed membership query which maximise
  the time within which $\toevent{x_i}$ is able to expire,
  subject to the constraints under which timers that expire in $\beta$ expire
  as scheduled, and under which any other timers do not expire. These
  other timers may not be known when the procedure starts, and so their
  corresponding constraints are added when they are detected: this happens
  when they expire unexpectectly.

  Note that conjunctions of constraints on delays $\delay{j}{k}$ can be represented as DBM, as is standard for timed automata.




