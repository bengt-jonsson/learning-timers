\section{Algorithm for Learning of MMTs}
\label{sec:learning}

In this section, we present an algorithm for learning MMTs in then untimed
MAT of~\ref{sec:untimed-mat}, using the approximated Nerode equivalence
presented in Section~\ref{sec:approx}.
The learning algorithm follows the standard pattern for active automata learning
algorithms, such as $L^*$~\cite{Ang87}. It maintains
a set $U$ of lean behaviors, called {\em short prefixes}, which
represent states in the MMT to be constructed,
and an overapproximation of the Nerode equivalence,
parameterized by a set $V$ of input suffixes.
%% Each short prefix in $U$ is a feasible canonical untimed
%% behavior, which represents a state in the MMT to be constructed.
The learning algorithm iterates two phases: hypothesis construction and
hypothesis validation.
During hypothesis construction,
the approximation of the Nerode equivalence triggers the expansion of
$U$ and $V$ until two convergence conditions are satisfied that allow
a hypothesis automaton to be formed.
During hypothesis validation, the hypothesis automaton is submitted in an
equivalence query, and returned counterexamples are used to refine
the Nerode equivalence by expanding $V$.

\todobj{Say somewhere that we assume a fixed (un)timed language $S$.
  Make sure we make clear that it is the set of {\bf feasible} lean
behaviors of the SUL}

Let us introduce the two conditions for convergence of the construction phase.
For a lean behavior $\beta\in S$ and $\gamma \in \suffixbehs{S}{\beta}$, let
$\beta\compose{S}\gamma$ be the (unique) lean behavior $\beta'\cdot\suffmap{|\beta|}(\gamma)$ in $S$ with $\beta \sqsubseteq \beta'$.
Let $\feasibleinputs{\beta}{S}$ be the set of $i \in \extinputs$ such that
$\beta \compose{S} \simpleutlabel{i}{o} \in S$ for some $o$ (recall that the
last assignment of a lean behavior is always empty).
For $i \in \feasibleinputs{\beta}{S}$, let $\lambda(\beta,i)$ be the unique
output $o$ such that $\beta \compose{S} \simpleutlabel{i}{o} \in S$.
%% A set $V$ of generic untimed input words is {\em adequate} if it includes
%% $I \cup \set{\toevent{p}}$.
Let $U$ be a prefix-closed set of lean behaviors.
and let $V$ be an adequate set of input suffixes.
\begin{itemize}
\item
$U$ is {\em closed} wrt.\ $V$ if 
  for each $\beta \in U$ and
  $i \in \feasibleinputs{\beta}{S}$
%%   , which is either in $I$ or of form $\toevent{x_i}$ where $x_i$ is expirable after $\beta$ if
  there is a $\beta' \in U$ such that
  $\beta\compose{S}\simpleutlabel{i}{\lambda(\beta,i)} \equiv_{S,V} \beta'$.
\item
$U$ is {\em timer-consistent} wrt.\ $V$ if 
  for each $\beta \in U$ and
  $i \in \feasibleinputs{\beta}{S}$
  %% $i$, which is either in $I$ or
  %% of form $\toevent{x_i}$ for $x_i$ expirable after $\beta$,
  we have
  $\apprgetmemorable{S}{\beta\compose{S}\simpleutlabel{i}{\lambda(\beta,i)}}{V} \subseteq
  (\apprgetmemorable{S}{\beta}{V} \cup \set{x_{(|\beta|+1)}})$.
\end{itemize}
Closedness ensures that each transition in the MMT to be constructed has a target state. 
Timer-consistency states that each timer which is needed after such
a transition (i.e., a timer set during $\beta\compose{S}\simpleutlabel{i}{\lambda(\beta,i)}$)
is either a timer active after $\beta$ or is started by the last transition, thus
getting the name $x_{(|\beta|+1)}$.
Closedness and timer-consistency allow the construction of a hypothesis MMT.

\begin{definition}[Hypothesis automaton]
\label{def:hypo}
  Let $U$ be a non-empty prefix-closed set of lean behaviors,
  and $V$ an adequate set of input suffixes such that
  $U$ is closed and timer consistent wrt.\ $V$. Then the
{\em hypothesis automaton} $\hypoof UV$ is the MMT
$\hypoof UV = (I, O, Q, q_0, \vars, \delta, \lambda, \remap)$, where
\begin{itemize}
\item $Q = U$ and $q_0 = \emptyword$,
\item $\vars$ maps each location $\beta\in U$ to $\apprgetmemorable{S}{\beta}{V}$,
\item Let $\beta \in U$ and $i \in \feasibleinputs{\beta}{S}$. Then
  \begin{itemize}
    \item $\lambda(\beta,i)$ is
     the unique $o$ such that $\extend{\beta}{i}{o} \in S$. 
    \item $\delta(\beta,i)$ is is
        the unique $\beta' \in U$ such that there is an $f$ with
  $\extend{\beta}{i}{\lambda(\beta,i)} \equiv_{S,V}^f \beta'$.
      \item $\remap(\beta,i):\apprgetmemorable{S}{\beta'}{V} \mapsto (\apprgetmemorable{S}{\beta}{V} \cup \natplus)$ is defined as $f^{-1}$ on
        $\apprgetmemorable{S}{\beta'}{V}$, except that it maps
        $f(x_{(|\beta|+1)})$ to
        $\apprgetassignment{S}{\beta'}{V}(f(x_{(|\beta|+1)}))$
        if $f(x_{(|\beta|+1)}) \in \apprgetmemorable{S}{\beta'}{V}$.
  \end{itemize}
\item When $\beta \in U$ and $i$ is of form $\toevent{x_j}$ with
  $x_i \apprgetmemorable{S}{\beta}{V}$ but
  $\toevent{x_j} \not\in\feasibleinputs{\beta}{S}$, we let
  $\lambda(\beta,i) = \bot$, 
  $\delta(\beta,i) = \beta$, and let
  $\remap(\beta,i)$ be the identity mapping on $\apprgetmemorable{S}{\beta}{V}$.
\end{itemize}
\end{definition}
The last case (where  $\toevent{x_j} \not\in\feasibleinputs{\beta}{S}$)
constructs a transition that is not feasible,
but which must anyway be syntactically present, since $x_j$
may expire after some continuation if $\beta$ and is hence live.

{\bf Hypothesis construction}, performs membership queries in order
 to construct the sets of form $\apprsuffixbehs{S}{\beta}{V}$ and
 $\apprsuffixbehs{S}{\extend{\beta}{i}{\lambda(\beta,i)}}{V}$
 for $\beta \in U$ and $i \in \feasibleinputs{\beta}{S}$.
 Moreoever, the sets $U$ and $V$ are expanded if needed to construct
 a hypothesis automaton.

 More precisely,
membership queries are first performed
  for all untimed input words of form $\untimedinputword(\beta) \cdot i$
   for $\beta \in U$ and $i \in \extinputs$: this allows to determine
   $\feasibleinputs{\beta}{S}$ and $\lambda(\beta,i)$
   for $\beta \in U$ and $i \in \feasibleinputs{\beta}{S}$.
   Thereafter, membership queries are performed
  for all untimed input words of form $\untimedinputword(\beta) \cdot v$ and
  $\untimedinputword(\beta) \cdot i \cdot v$, where
  $\beta \in U$, $v \in V$, and
  $i \in \feasibleinputs{\beta}{S}$. Note that one need only consider
  $v$ in which timeouts from $\toevents$ concern timers in
  $\set{x_1, \ldots, x_{|\beta|}}$ (or in  $\set{x_1, \ldots, x_{(|\beta|+1)}}$
  for $\untimedinputword(\beta) \cdot i$).
  This allows to construct the sets of form $\apprsuffixbehs{S}{\beta}{V}$ and
  $\apprsuffixbehs{S}{(\extend{\beta}{i}{\lambda(\beta,i)})}{V}$
  for $\beta \in U$.
  It then allows to compute the approximated Nerode equivalence $\equiv_{S,V}$ on
  the set of lean behaviors of form $\beta$ and
  $\extend{\beta}{i}{\lambda(\beta,i)}$ with
  $\beta \in U$ and $i \in \feasibleinputs{\beta}{S}$.
  We then check whether $U$ and $V$ meet the convergence criteria.
  \begin{itemize}
    \item
  Whenever the set $U$ is not closed wrt.\ $V$, then it is extended:
if there is some $\beta \in U$ and $i$ in $\feasibleinputs{\beta}{S}$
for which there is no $\beta' \in U$ such that
$\extend{\beta}{i}{\lambda(\beta,i)} \equiv_{S,V} \beta'$, 
then $\extend{\beta}{i}{\lambda(\beta,i)}$ is added to $U$,
triggering new membership queries.
\item
  Whenever the set $U$ is not timer-consistent wrt.\ $V$, then $V$ is extended:
for timer $x_j$ in
$\apprgetmemorable{S}{\extend{\beta}{i}{\lambda(\beta,i)}}{V} \setminus (\apprgetmemorable{S}{\beta}{V} \cup \set{x_{(|\beta|+1)}})$, find a lean
suffix $\gamma$ in
$\apprsuffixbehs{S}{(\extend{\beta}{i}{\lambda(\beta,i)})}{V}$, whose last input
is $\toevent{x_j}$. This input is
obviously missing from $\apprsuffixbehs{S}{\beta}{V}$. But, by adding
the concatenation of $i$ with $\untimedinputword(\gamma)$ (where
timers are  appropriately renamed), the input $\toevent{x_j}$
will also appear in $\apprgetmemorable{S}{\beta}{V}$.
This extension of $V$ will subsequently trigger new membership queries.
\end{itemize}
When $U$ is closed and timer-consistent wrt.\ $V$, then
a hypothesis MMT $\hypoof UV$ is constructed and
validated by submitting it in an equivalence query.
If the query returns ``yes'', then
the learning is completed, and $\hypoof UV$ accepts $S$.
If the query returns a counterexample in the form of a behavior
$\alpha$ on which $\hypoof UV$ and $S$ disagree,
a procedure for counterexample processing, found in the appendix, is used to extend $V$ by a new input suffix $v$
such that
$U$ is no longer closed wrt.\ $V$.
\ifshort
\else
as follows.
We assume w.l.o.g.\ that no proper prefix of $\alpha$ is a counterexample. 
By the fact that $\alpha$ is a counterexample, we can find a lean
suffix $\gamma$ of $\alpha$, such that
$\beta\cdot \simpleutlabel{i}{o} \equiv_{S,V}^f \beta'$ but
$\gamma \in \suffixbehs{S}{(\extend{\beta}{i}{\lambda(\beta,i)})}
\smallnotiff \gamma \in \suffixbehs{S}{\beta'}$
for some $\beta,\beta' \in U$ such that 
$\beta\compose{S}\simpleutlabel{i}{o} \equiv_{S,V}^f \beta'$ is used to construct
the transition triggered by $i$ from $\beta$ in $\hypoof UV$.
To se this, let $\alpha = \utlabel{i_1}{o_1}{\rho_1} \cdots \utlabel{i_n}{o_n}{\rho_n}$, and for $j = 0 , \ldots, n$, define
a lean behavior $\beta_j$ and a $\beta_j$-suffix as follows.
\begin{itemize}
\item $\beta_0 = \emptyword$, and
%%   , $f_0$ is arbitrary, and 
  $\gamma_0$ is obtained by making $\alpha$ a lean suffix (i.e.,
  renaming each timer $x_l$ to $y_l$).
\item
  Let $\gamma_{j-1}$ be of form
  $\utlabel{\hat{i_j}}{\hat{o_j}}{\hat{\rho_j}} \cdot \gamma_{j-1}'$,
  let $\beta_j$ be $\lambda(\beta_{j-1},\hat{i_j})$,
  let $f_j$ be the mapping used to establish
  $\beta_{j-1}\compose{S}\simpleutlabel{\hat{i_j}}{\hat{o_j}} \equiv_{S,V}^{f_j} \beta_{j}$
  in the construction of $\hypoof UV$,
  and let $\gamma_j$ be obtained from $\gamma_{j-1}'$ by
  (i) applying $f_j$ to timers in $\set{x_1, \ldots , x_{\beta_{j-1}}}$,
  (ii) replacing $y_1$ by $f_j(x_{\beta_{j-1}+1})$, and
  (ii) replacing each $y_l$ by $y_{l-1}$.
\end{itemize}
The result is that
$\beta_0 \ldots \beta_n$ is the sequence of states 
visited when $\hypoof UV$ processes $\alpha$, and
$\gamma_j$ is lean suffix that can be composed with $\beta_j$,
which corresponds to a suffix of $\alpha$.
%% Let $\gamma_j$ be the suffix
%% $\utlabel{i_{j+1}}{o_{j+1}}{\rho_{j+1}} \cdots \utlabel{i_n}{o_n}{\rho_n}$
%% of $\alpha$ of length $n-j$. 
By the fact that $\alpha$ is a counterexample, we have
$\gamma_0 \in\suffixbehs{S}{\beta_0} \smallnotiff \gamma_n \in\suffixbehs{S}{\beta_n}$ (since $\gamma_n$ is the empty sequence), which implies that
$\gamma_{j-1} \in\suffixbehs{S}{\beta_{j-1}}
\smallnotiff
\gamma_{j} \in\suffixbehs{S}{\beta_{j}}$
for some $j$;
we can then take $\beta_{j-1}$ as $\beta$ and $\beta_j$ as $\beta'$, and
$\gamma$ as $\gamma_j$.
This means that $\gamma$ is a new separating suffix, and that $V$ should be
extended with $\untimedinputword(\gamma)$.
After adding $v$ (and its permutations) to $V$, $U$ is no longer closed
wrt.\ $V$,
\fi
The algorithm can then resume a next round of
hypothesis construction, which will eventually generate a new hypothesis, etc.

The algorithm enjoys properties analogous
to those of, e.g., $L^*$~\cite{Ang87}.
The additional complexity caused by timers is analogous to 
that  caused by registers in learning of register
automata~\cite{HowarSJC12,CasselHJS16}.

\begin{theorem}
  \label{thm:alg:termination}
  Given an MMT $\M$ whose canonical form has $n$ states, each of which has
  at most $r$ active timers, the
  procedure of this section terminates and produces an equivalent MMT in
  canonical form, using a number of queries that is at most
  polynomial in $n$ and doubly exponential in $r$.
\end{theorem}

At termination the hypothesis is correct, by definition of equivalence query.
During the construction, $U$ and $V$ will expand until they represent
$\equiv_S$, at which time the hypothesis will be the desired one.
Let us analyze the number of queries that may be required in the worst case
to learn an MMT whose canonical form has $n$ states and at most $r$ active
timers in any state. In the below, we let $|V|$ be the number of unique
elements of $V$, before adding permutations.
\begin{itemize}
\item
  Hypothesis construction may need $n\cdot(|I|\!+\!r\!+\!1)\cdot |V| \cdot r!$
  membership queries in total. The factor $r!$ arises as the number of
  permutations of prefix-timers in each suffix.
\item
  Processing a counterexample may need $\log(m)$ membership queries,
  using binary search, where $m$ is the length of the counterexample.
\item
  Each equivalence query will result in refuting an equivalence of form
  $\beta\cdot \simpleutlabel{i}{o} \equiv_{S,V}^f \beta'$, and extending
  $V$. There are at most
  $r !$ possible permutations $f$ for each $\beta\cdot \simpleutlabel{i}{o}$,
  implying at most $n \cdot r!$ equivalence queries.
\item Since each equivalence query adds at most one element
 to $V$, we have $|V| \leq n\cdot r!$ when the algorithm finishes.
\end{itemize}


%% implying that in total, at most $n^2|I|$ membership queries
%% will be performed during hypothesis construction.
%% During hypothesis
%% validation, at most $2\log(m)$ membership queries need be performed
%% (in addition to the equivalence query), where
%% $m$ is the length of the largest counterexample word returned.

