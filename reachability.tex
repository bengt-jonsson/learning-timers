\newcommand{\Zone}[1]{\mathsf{Zone}({#1})}
\iflong
\subsection{Reachability analysis}

In order to implement an untimed MMT teacher using a timed teacher, we need some basic algorithms to
analyse and manipulate untimed behaviors and timed words that we will discuss in this subsection.

It is not difficult to translate an MMTs to a timed automaton \cite{AD94,BengtssonY03} that accepts the same timed words.
Through such a translation, verification and analysis tools for timed automata, such as Uppaal \cite{Uppaal4.0}
become available for MMTs.
MMTs are less expressive than timed automata. MMTs, for instance, do not contain time deadlocks or zeno loops that
may prevent time from progressing.

MMTs have infinitely (in fact even uncountably) many configurations and it is thus necessary to use symbolic representations
for reachability analysis. Given an untimed behavior $\beta$ we want to compute its effect on a set of valuations. 
If $\beta =X_0$ and $K \subseteq\Vals{X_0}$ then we define $\Post_{\beta}(K) = K$.
If $\beta = X_0 \xrightarrow{i_1/o_1, \rho_1} X_1$ and
$K \subseteq\Vals{X_0}$ then $\Post_{\beta}(K)$ is equal to the set of all valuations $\tvals'' \in\Vals{X_1}$ such that
\[
  \exists \tvals \in K \exists d >0 \exists \tvals' \in\Vals{X_0} :
 \tvals \xrightarrow{d} \tvals' \xrightarrow{i_1/o_1, \rho_1} \tvals'' .
\]
The definition of $\Post_{\beta}(K)$ is extended inductively to arbitrary $\beta$ by 
$\Post_{\gamma \cdot \gamma'}(K) = \Post_{\gamma'} (\Post_{\gamma}(K))$, where $\gamma \cdot \gamma'$ is the decomposition of
$\beta$ into an untimed behavior $\gamma$ of length one and an untimed behavior $\gamma'$.
We write $\Zone{\beta}$ as abbreviation for $\Post_{\beta}(\Vals{\Head{\beta}})$.

For any untimed behavior $\beta$, $\Zone{\beta}$ can be symbolically represented and computed using a Difference Bound Matrices (DBMs)
 \cite{Di89}, as the transition relations $\xrightarrow{d}$ and $\xrightarrow{i_1/o_1, \rho_1}$ can be decomposed 
into elementary operations on DBMs such as reset, conjunction, and delay successors \cite{BengtssonY03}.
%
It is easy to see that an untimed behavior $\beta$ is feasible iff $\Zone{\beta} \neq \emptyset$.
Since emptiness of DBMs is decidable, this allows us to compute whether or not $\beta$ is feasible.

The next technical lemma's are needed further on:

\begin{lemma}
\label{lemma: feasibility concatenation}
Suppose $\beta, \beta'$ are untimed behaviors such that
$\Zone{\beta} = \Zone{\beta'}$. Let $\gamma$ be any untimed behavior.
Then $\beta \cdot \gamma$ is feasible iff $\beta' \cdot \gamma$ is feasible.
\end{lemma}

\begin{lemma}
\label{lemma finitely many zones}
$\{ \Zone{\beta} \mid \beta \mbox{ feasible untimed behavior of } \M \}$ is finite.
\end{lemma}
\begin{proof}
All the sets $\Zone{\beta}$ can be represented using DBMs. An MMT only has a finite number of timers that can only be set to a finite number of integer values. Since in an MMT the values of timers can only decrease, only finitely many numbers may
appear in the DBM's that represent the sets $\Zone{\beta}$. Thus all the sets $\Zone{\beta}$ can be represented by a finite
collection of DBMs.
\end{proof}

\begin{lemma}
\label{feasible plus input is feasible}
Suppose $\beta$ is a feasible untimed behavior with $\Last{\beta} = Y$ and 
suppose $Y \xrightarrow{i/o, \rho} Y'$ is an untimed behavior with $i \in I$.
Then $\beta \xrightarrow{i/o, \rho} Y'$ is a feasible untimed behavior.
\end{lemma}

Let $\beta$ be a feasible untimed behavior and let $x \in X$ be a timer. Then we say that $x$ is \emph{expirable} after $\beta$
if there exists a valuation in $\Zone{\beta}$ in which $x$ is minimal.

\begin{lemma}
Suppose $\beta$ is a feasible untimed behavior with $\Last{\beta} = Y$ and $Y \xrightarrow{\toevent{x}/o, \rho} Y'$ is an untimed behavior.
Then $x$ is expirable after $\beta$ iff $\beta \xrightarrow{\toevent{x}/o, \rho} Y'$ is feasible.
\end{lemma}

Suppose that $\beta$ is an untimed behavior in canonical form:
\begin{eqnarray*}
\beta & = & \emptyset \xrightarrow{i_1/o_1, \rho_1} X_1  \cdots X_{k-1} \xrightarrow{i_k/o_k, \rho_k} X_k.
\end{eqnarray*}
We associate to $\beta$ a set of constraints with real-valued variables $t_1 ,\ldots t_k$ that denote the time of
occurrence of events $i_1 ,\ldots, i_k$ respectively. We add the following constraints to this set $\Constraints{\beta}$:
\begin{itemize}
\item
$0 < t_1$,
\item
for each index $j < k$:  $0 <  t_{j+1} - t_j \leq d_{\max}$,
\item
for each timeout event $i_j = \toevent{x_l}$: $t_j = t_l + \rho_j(x_l)$,
\item
for each clock $x_l$ that is started but does not timeout: $t_j \leq t_l + \rho_j(x_l)$,
where $j$ is the largest index such that $x_l \in X_j$, and
\item
for each pair of distinct indices $j$ and $l$ with $i_j, i_l \in I$: $\Frac{t_j} \neq \Frac{t_l}$ 
(to express that the fractional parts of $t_j$ and $t_l$ are different).
\end{itemize}
Note that $\beta$ is feasible iff the set of constraints $\Constraints{\beta}$ is satisfiable.
We may use an SMT solver to decide whether $\Constraints{\beta}$ is satisfiable and to compute\footnote{If an SMT solver
does not support fractions, we may compute a solution without these constraints and then use the wiggling approach
of Section~\ref{section untimed semantics} to ensure that the fractional times of inputs are different.} a solution for
$t_1 ,\ldots t_k$.

\todofv{Next two lemmas belong in different section}

\begin{lemma}
\label{not untimed}
Suppose $\M$ and $\N$ are MMTs with $\M \not\approx_{\mathit{untimed}} \N$.
Then there exists a feasible untimed behavior $\beta$ of $\M$ that is not isomorphic to any feasible untimed
behavior of $\N$.
\end{lemma}

\begin{lemma}
\label{not timed}
Suppose $\M$ and $\N$ are MMTs with $\M \not\approx_{\mathit{timed}} \N$.
Then there exists a transparent timed word $w$ of $\M$ that is not a timed word of $\N$.
\end{lemma}

%\marginpar{What is the complexity of deciding eg timed equivalence or reachability of MMTs?}
\fi
